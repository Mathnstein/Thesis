Dynamical systems is the study of the possible states an observable solution may experience and is important in most engineering, biological or even chemical systems to name a few. This approach allows conditions to be given for when a solution can be found or when there is stable behavior. Often we find that parameters inherent in the model play huge roles in the dynamical behavior and they can be the difference between a system having an equilibrium or not. When we find a parameter that has this effect, we call it a bifurcation parameter since there is some value that changes the qualitative behavior of the system.

\indent For example, the Hodgkin-Huxley model for neurons contains a parameter for injected current $I$ which turns out to be a bifurcation parameter with a Hopf bifurcation. This bifurcation is responsible for the actual firing of a neuron in the brain. In epidemiological modeling, the basic SIR model with an additional transition function between the infected and recovered population causes the reproduction number $R_0$ to become a bifurcation parameter with a backward bifurcation. This causes a temporary equilibrium to form in the infected population that usually would never see an equilibrium. Even in activation potentials of neural networks, using a hyperbolic tangent function causes a bifurcation to occur in the synaptic feedback parameter $w$ which results in a pitchfork bifurcation. This causes wildly different equilibria for learned parameters in a machine learning setting. The canonical example is the \textit{saddle-node} bifurcation and was the first to be found within a complex system studied from a dynamical perspective. The saddle-node bifurcation has the locally topological equivalent form

\begin{equation}\label{eq:intro_saddlenode}
\dot{x}=a-x^2,
\end{equation}

where by locally topological equivalence we mean the behavior near the bifurcation may be represented in this form. This property is critical to reducing most complex problems and models to simpler local problems that can be studied individually.

\begin{figure}[H]
\centering
\includegraphics[width = .7\linewidth]{intro/saddlenode.jpg}
\caption{Vector field of a saddle-node bifurcation $a=0$.}
\label{fig:intro_saddlenode}
\end{figure}

\begin{figure}[H]
\centering
\includegraphics[width=.7\linewidth]{intro/saddlenode_bif_diagram.jpg}
\caption{Bifurcation diagram of saddle-node bifurcation $a=0$.}
\label{fig:intro_saddlenode_bif_diagram}
\end{figure}

\indent In figure~\ref{fig:intro_saddlenode} we show the vector field of the system that contains a saddle-node bifurcation. The equilibria of this system is $x=\pm \sqrt{a}$ for any $a\ge0$ where stable equilibrium points are marked with red filled points and unstable with unfilled points. Notice that at $a=0$ we are no longer able to find a stable equilibrium and when $a<0$ there are no equilibria at all. Thus $a=0$ is a simple example of a bifurcation where two equilibria annihilate. This behavior is why the bifurcation is often referred to as a \textit{fold} bifurcation, although we refer to this as a saddle-node bifurcation in this thesis. In figure~\ref{fig:intro_saddlenode_bif_diagram} we plot the same system against the parameter $a$, which we call the bifurcation diagram. Here we see the region with two equilibrium $a>0$, the bifurcation $a=0$ and the region of no stability $a<0$. For more on the saddle-node bifurcation see \cite{kuznetsov2006saddle}.

\indent There are many types of bifurcations that appear in different systems that each have their own key properties. Studying these properties leads to a deeper understanding of the system on both a global and a local scale. Work has been done on systems that have smooth bifurcations due to how commonly these appear, but non-smooth dynamics still are present in the physical world.

\indent Non-smooth bifurcations are a topic that arise in special systems and for how frequent they appear, they have not been studied nearly as much as their smooth counter parts. This thesis discusses the role of the non-smooth saddle-node bifurcation in a simplified one component system in \autoref{chap:oneD} as well as in the classic Stommel model for thermohaline circulation dynamics in \autoref{chap:twoD}. Many interesting ocean and weather mechanisms may be incorporated into the Stommel model to provide more realistic predictions for weather patterns. We choose to study slowly varying bifurcation parameters and their effect on the stability of a system while contrasting this with non-autonomous oscillatory forcing. The interaction of these features causes complex dynamics around the standard bifurcations and can lead to an advanced bifurcation or delayed tipping. For the one component system, a detailed analysis of these features is done on the smooth bifurcation in \cite{zhu2015tipping}.

\section*{Tipping in a Slowly Varying System}
A system with a parameter known to cause a bifurcation will no longer admit a bifurcation in the standard sense when the parameter slowly varies. Instead, these conditions give rise to a smooth but rapid change in the system's equilibria. The point in which this behavior occurs is then called a \textit{tipping point}.

\indent More formally, a tipping point is a point that causes an abrupt smooth transition in dynamical behavior as the system moves into a qualitatively different state. This is usually caused by some exterior control system that pushes change towards a different state once a critical point has been passed, for example with biological systems seen in \cite{angeli2004detection}. These are known to be caused by changes in one or more parameters in the system. An analysis that lays the theoretical backing of slowly varying parameters with algebraic bifurcations is found in \cite{haberman1979slowly}.

\indent Tipping points have been discovered to occur in a wide variety of systems and have become a big staple in the study of areas like catastrophe theory and dynamical systems. They aid in predicting the future of a system and even could be a warning for irreversible change like in the case of the Stommel model. A tipping point thus shares similar characteristics to a bifurcation and typically occurs close to the static bifurcation location.

\indent In this thesis we use the results from \cite{zhu2015tipping} where the system

\begin{equation}\label{eq:intro_Zhueq}
\begin{aligned}
\dot{x} =& Da + k_0 +k_1 x + k_2 x^2,\\
\dot{a} =& -\epsilon,
\end{aligned}
\end{equation}

where $\epsilon\ll 1$ was studied. This model is a slowly varying quadratic differential equation containing a smooth saddle-node bifurcation and appears in many physical models, for example \cite{erneux1989jump}. A key result from \cite{zhu2015tipping} is that the solution and tipping point for \eqref{eq:intro_Zhueq} have the form

\begin{equation}\label{eq:intro_Zhuairy}
x\sim \frac{1}{|k_2|}\left(\frac{k_1}{2}+\left(\frac{D|k_2|}{\epsilon}\right)^{1/3}\right)\frac{Ai'\left([D|k_2|\epsilon]^{-2/3}\left(\frac{k_1^2}{4}+k_0|k_2|+D|k_2|a\right)\right)}{Ai\left([D|k_2|\epsilon]^{-2/3}\left(\frac{k_1^2}{4}+k_0|k_2|+D|k_2|a\right)\right)}
\end{equation}

\begin{equation}\label{eq:intro_Zhuresult}
a_{\text{tip}}=(D|k_2|)^{-1/3}a_{\text{Airy}}-\frac{a_s}{D}\quad \text{for} \quad a_s = k_0+\frac{k_1^2}{4|k_2|},
\end{equation}

with $Ai(\cdot)$ being the Airy function and $a_{\text{Airy}}=\epsilon^{2/3}\cdot(-2.33810\ldots)$ corresponding to the first zero of the Airy function. The singularity found in \eqref{eq:intro_Zhuresult} is a recurring tool for the work presented in this thesis, even though we deal with a version of \eqref{eq:intro_Zhueq} that has a non-smooth bifurcation.

\begin{figure}[H]
\centering
\begin{subfigure}{.5\textwidth}
 \centering
 \includegraphics[width=\linewidth]{intro/saddlenode_tipping.jpg}
 \caption{}
\end{subfigure}%
\begin{subfigure}{.5\textwidth}
 \centering
 \includegraphics[width=\linewidth]{intro/saddlenode_tipping_zoom.jpg}
 \caption{}
\end{subfigure}
\caption{The saddle-node bifurcation. In (a) an example of tipping occurring around the bifurcation for two sizes of slow variation, $\epsilon=\{.01,.1\}$. In (b) a zoom in closer to the bifurcation. The dashed ($\epsilon = .01$) and dash-dotted ($\epsilon=.1$) black lines are the numerical solutions, we overlay the bifurcation diagram for reference.}
\label{fig:intro_tipping}
\end{figure}


\indent In figure~\ref{fig:intro_tipping} we show a numerical solution to the simple saddle-node system with tipping \eqref{eq:intro_Zhueq}.  Here we have $D=1$, $k_0=k_1=0$ and $k_2=-1$ which is the model from \eqref{eq:intro_saddlenode}. The solution follows closely to the stable branch even after the bifurcation for the static model, which is an example of this delayed behavior. From here on we refer to the numerical tipping point to be when the numerical solution has passed a threshold away from the equilibrium such that it is reasonable to say the solution is transitioning to a new state. We call this threshold the tipping criterion and it is specified whenever we compare our estimates  to the numerical solution.

The task of finding where tipping points occur depends on the situation, but in general the approach is to search for when a solution to a model fails or becomes large. Examples could be when the solution fails to be real or when an exponential term grows large, both of which are seen throughout this thesis.


\section*{The Stommel Model}

Global circulation models have primarily focused on three different categories: 

\begin{itemize}
\setlength\itemsep{1em}
\item Atmospheric components - the effect greenhouse gases have on the atmosphere,
\item Oceanic components - the effect of tides and interaction of temperature with salinity in the oceans,
\item Sea ice and land surface components.
\end{itemize}

These categories all contribute significantly to the overall prediction of weather and climate for the planet, which has importance to just about every industry and economy. Failure to adhere to and prepare for sudden changes in the climate has led to drastic situations like severe droughts or ocean acidification. Atmospheric models have been vastly studied but far less work has been done on the contribution from the ocean and the dynamics that drive the tides and currents.

\indent A key feature of oceanic models is when patterns form around regions of bi-stability of temperature and salinity. An example of this is the thermohaline circulation (THC) which has abrupt qualitative changes at certain points, see \cite{alley2003abrupt,marotzke2000abrupt,rahmstorf2000thermohaline,rahmstorf2002ocean}. Just earlier this year evidence was found of weakening occurring around these abrupt changes in a system of ocean patterns known as the Atlantic meridional overturning circulation (AMOC) \cite{caesar2018AMOC}. This is the first evidence of ocean dynamics responding to temperature change on the surface and can help further predict the future of the system. It is imperative that appropriate action is taken to prepare for the future of these type of systems as they are outside our realm of control. 

\indent To study these phenomena we create parametric models to replicate the dynamics we observe. Initially, Henry Stommel proposed the two box model in 1961 to understand the physics of the THC, shown in figure~\ref{fig:stommel_boxes}. In \cite{stommel1961thermohaline}, it is suggested that there are actually two different stability regimes which even overlap in the system that is proposed and concluded that oceanic dynamics behave very similarly about these equilibria. These type of systems have since been a heavily studied area for both climatology due to the wide ranging applications and dynamical systems for its generalization into bi-stability.

\begin{figure}[H]
\centering
\includegraphics[width=0.7\textwidth]{intro/Box.jpg}
\caption{The Stommel Two Box Model: Differing volume boxes with a temperature and salinity, $T_i$ and $S_i$. The boxes are connected by an overflow and capillary tube that has a circulation rate $V$. There is also a surface temperature and salinity for each box, ${T_i}^s$ and ${S_i}^s$. We assume that there is some stirring to give a well mixed structure.}
\label{fig:stommel_boxes}
\end{figure}

\indent With emphasis on mathematics, the focus of this thesis is on developing an effective approach to models with bi-stability and additional mechanisms. Thus the physical quantities are brushed aside in favor of their non-dimensional alternatives, see \autoref{app:stommel} for the derivation. The non-dimensionalized Stommel model is represented with the system

\begin{equation}
 \begin{aligned}
  \dot{T} & = \eta_1-T(1+|T-S|), \\
  \dot{S}   & = \eta_2-S(\eta_3+|T-S|). 
 \end{aligned}
\end{equation}

The variables $T$ and $S$ are the temperature and salinity respectively where the non-smoothness is seen directly from the $|T-S|$ term. The parameters $\eta_1$, $\eta_2$, and $\eta_3$ are all dimensionless quantities that each have physical interpretation to the relaxation times and volumes of the box. Here $\eta_1$ is thought of as the thermal variation, $\eta_2$ as the saline variation otherwise known as the freshwater flux, and $\eta_3$ as the ratio of relaxation times of temperature and salinity. It also is a physical restriction for both $\eta_1$ and $\eta_3$ to be positive quantities that take any value. The parameter $\eta_3$ has the additional property to determine the orientation of the equilibria. We denote a standard orientation to be when $\eta_3<1$, reverse orientation for $\eta_3>1$, and $\eta_3=1$ a special case. The different orientations are shown in figure~\ref{fig:Stommel_bif_plots}. Recall that $\eta_3$ is the ratio of relaxation rates and when $\eta_3=1$ the relaxation rates for both the thermal and salinity variables are the same. Under these conditions we lose bi-stability and instead see a single stable equilibrium.

\indent The parameter $\eta_2$ is the most interesting as different values cause major qualitative and quantitative changes in the dynamics of the system. Bifurcations have been discovered at two different points in the system, each being called either a smooth or a non-smooth saddle-node bifurcation. In the Stommel model, it is convenient to view the system in terms of the circulation rate $V=T-S$, see \autoref{app:stommel} for the derivation. This leads to the system

\begin{equation}\label{eq:basic_stommel}
 \begin{aligned}
  \dot{T} & = \eta_1-T(1+|V|), \\
  \dot{V}   & = (\eta_1-\eta_2)-V|V|-T+\eta_3(T-V).
 \end{aligned}
\end{equation}

\begin{figure}[H]
\centering
\begin{subfigure}{.5\textwidth}
 \centering
 \includegraphics[width=\linewidth]{intro/T_equil.jpg}
 \caption{$V$ vs. $T$}
 \label{fig:Tequil}
\end{subfigure}%
\begin{subfigure}{.5\textwidth}
 \centering
 \includegraphics[width=\linewidth]{intro/V_bif.jpg}
 \caption{$\eta_2$ vs. $V$}
 \label{fig:Vbif}
\end{subfigure}
\caption{The equilibria of the non-dimensionalized system \eqref{eq:basic_stommel}. Parameters values are $\eta_1=4$ and $\eta_3=.375$. The above plots are two-dimensional projections of the full 3-dimensional system in ($\eta_2$,$V$,$T$). We see non-smooth behavior happening in both plots when $V=0$. The red line indicates a stable branch where the dashed dotted line is for an unstable branch.}
\label{fig:systemequil}
\end{figure}

\indent As shown in figure~\ref{fig:systemequil}, the equilibrium curves reveal much about the dynamics. In (a) the graph of the equilibria for $V$ vs. $T$ shows non-smooth behavior occurring at $V=0$ and in (b) the two types of bifurcation appear clearly in the graph of equilibria for $\eta_2$ vs. $V$. In this plot, both the upper and lower branches of the equilibrium are stable with the middle branch being unstable. The stable branches relate to which variable is dominant. For the lower branch, we call this the saline branch, and the upper branch the thermal branch. The location of the non-smooth bifurcation is found analytically, $({\eta_2}_{\text{ns}},V_{\text{ns}},T_{\text{ns}})=(\eta_1\eta_3,0,\eta_1)$, and the smooth bifurcation, $({\eta_2}_{\text{smooth}},V_{\text{smooth}},T_{\text{smooth}})$, is the only real solution to a cubic polynomial. The smoothness of each bifurcation is apparent and arises from the absolute value term in the defining dynamics of \eqref{eq:basic_stommel}, which is non-smooth only at $V=0$.

\begin{figure}[H]
\centering
\begin{subfigure}{.5\textwidth}
 \centering
 \includegraphics[width=\linewidth]{intro/V_bif.jpg}
 \caption{$\eta_3=.375$}
\end{subfigure}%
\begin{subfigure}{.5\textwidth}
 \centering
 \includegraphics[width=\linewidth]{intro/V_bif_collapse.jpg}
 \caption{$\eta_3=1$}
\end{subfigure}
\begin{subfigure}{.5\textwidth}
 \centering
 \includegraphics[width=\linewidth]{intro/V_bif_reverse.jpg}
 \caption{$\eta_3=1.875$}
\end{subfigure}
\caption{The choice in $\eta_3$ dictates the orientation of the problem, in each plot we have fixed $\eta_1=4$. The case for $\eta_3=1$ is special due to the two bifurcations overlapping and the unstable equilibrium vanishing.}
\label{fig:Stommel_bif_plots}
\end{figure}

\indent Much is known about the Stommel model in the case where $\eta_2$ is fixed but realistically this is not the case. In \cite{rahmstorf2000thermohaline}, this parameter is described as the influx of freshwater into the Atlantic and the changing nature of $\eta_2$ is justified by a positive feedback loop for salinity that drives the THC to move high-salinity water towards deep oceans. This loop causes the abrupt smooth bifurcation but then afterwards, a salinity deficit causes the parameter to decrease back towards the non-smooth bifurcation.

\indent This type of behavior is known as hysteresis, where there is some bi-stability region that the solution cycles through and observes both states of the equilibria. A similar analysis to the Stommel model's hysteresis can be found in \cite{roberts2017relaxation}. The phenomena of hysteresis appears in many physical systems, for example \cite{jung1990scaling,hohl1995scaling,joshi2005dynamical}. The smooth component of the hysteresis curve has been studied in a reduced one component model, see \cite{zhu2015tipping}. In this thesis we complement these results with an analysis of the non-smooth component.

\section*{Numerical Methods}

To obtain numerical solutions to the ordinary differential equations studied in this thesis, we choose to use both 2nd order and 4th order Runge-Kutta methods. The 2nd order method are used for the one component model and the 4th order method for the two component model. The choice in these methods comes from using the simplest scheme since numerical sensitivity is not present in our problem. We use the numerical solutions to compare our approximations to the observed state transition.