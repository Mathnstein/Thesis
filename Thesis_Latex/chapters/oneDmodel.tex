We consider a simpler system to give insight into the more complex two dimensional Stommel model, here we choose to look at a one-dimensional model with

\begin{equation}\label{eq:oneD_canonical}
\begin{aligned}
\dot{x}=&-\mu+2|x|-x|x|+A\sin(\Omega t),\\
\dot{\mu}=&-\epsilon,\quad \epsilon\ll 1,
\end{aligned}
\end{equation}

\begin{equation*}
x(0)=x^0,\quad\mu(0)=\mu^0,
\end{equation*}

where the constants are the slow variation rate $\epsilon$, the amplitude of oscillation $A$ and the frequency of oscillation $\Omega$. We also assume the initial conditions to be ${x^0=1-\sqrt{1+\mu^0}}$ and $\mu^0>0$ while focuses our calculations on the lower equilibrium branch where $x<0$ and search for nearby behavior.\\
The system \eqref{eq:oneD_canonical} is generalized from a basic model that contains both a smooth and non-smooth saddle-node bifurcation. This structure gives the similarity to the Stommel model and hence a good place to test generalizations like slow variation or oscillatory forcing. In each case, emphasis is put on the non-smooth component of the model to study the non-smooth bifurcation and it's role in the hysteresis curve we anticipate in the Stommel model.

\section{Static Bifurcations}
\label{sec:oneD_static}

The foundation to our understanding comes from the simplest structure lying within the canonical system \eqref{eq:oneD_canonical} which is the bifurcation structure. This means finding the general form for the equilibria in \eqref{eq:oneD_canonical} with $A=0$ and $\epsilon=0$, which is our basic model with a static $\mu$ and no forcing. As we have a fixed parameter value, we search for a point or set of points that the solution relaxes to with sufficiently long time. We call these points the equilibrium points and depending on their sensitivity, are either stable or unstable equilibria. But as we are considering all possible $\mu$, we want all of the equilibrium points for each $\mu$. We call these the equilibrium branches.

To find all equilibrium branches, we search for when the solution has come to a rest, which is equivalent to setting the derivative of $x$ is zero. Thus we set \eqref{eq:oneD_canonical} to zero with

\begin{equation}\label{eq:oneD_static_equil}
0=-\mu +2|x|-x|x|.
\end{equation}

Solving \eqref{eq:oneD_static_equil} results in 3 solutions, the stability of each depends on if small perturbations to the equilibrium result in growth or decay. We denote the stable equilibria as $x_l$ and $x_u$ for lower and upper respectfully, and a single unstable middle branch, $x_{m}$,

\begin{equation*}
x_l=1-\sqrt{1+\mu},\quad x_u=1+\sqrt{1-\mu},\quad
x_{m}=1-\sqrt{1-\mu}.
\end{equation*}

Where $x_l$ is valid for $\mu\ge 0$ and both $x_u$, $x_{m}$ for $\mu\le 1$. Thus this system has a stable equilibrium for every choice in the parameter, but has a region of bi-stability for $0\le \mu\le 1$. This indicates that the boundary of this region are bifurcations with $\mu=0$ and $\mu=1$. These are the points ${(\mu,x)=(0,0)}$ and $(\mu,x)=(1,1)$ which are the non-smooth and smooth saddle-node bifurcations respectfully. Both are of saddle-node type due to pairs of equilibria annihilating at these locations. This all is shown in figure~\ref{fig:oneD_static_bifdiagram}.

\begin{figure}[H]
\centering
\includegraphics[width=\textwidth]{oneD/bif_diagram.jpg}
\caption{The one-dimensional bifurcation diagram with the upper and lower equilibrium branches as well as the unstable middle branch. The non-smooth bifurcation occurs at (0,0) with the black circle and the smooth bifurcation occurs at (1,1) with the black cross. }
\label{fig:oneD_static_bifdiagram}
\end{figure}


\section{Slowly Varying Parameter}
\label{sec:oneD_slow}

To develop a method for the slowly varying Stommel model, let us consider \eqref{eq:oneD_canonical} with $\epsilon\ll 1$ and $A=0$. Under these conditions, $\mu(t)$ is a function of time and thus a bifurcation no longer occurs. Instead, it is expected that a tipping point occurs nearby the previous bifurcation points as long as $\epsilon$ is small. The smooth case is well understood, see Zhu \& Kuske \cite{zhu2015tipping}, so let us consider the behavior of the non-smooth bifurcation with $x<0$. Since the parameter $\mu(t)$ is slowly varying in time, it makes sense to rescale using this as our 'slow' time, $\tau=\epsilon t$. Applying both $x<0$ and this 'slow' time approach to the system \eqref{eq:oneD_canonical} then gives

\begin{equation}\label{eq:oneD_slow_scaled}
\begin{aligned}
\epsilon x_\tau=&-\mu(\tau)-2x+x^2,\\
\mu_\tau=&-1.
\end{aligned}
\end{equation}

A standard approach to extracting information out of complicated models is to try to find reduced equations by separating the behavior at each order. This approach is known as using an asymptotic expansion and further details can be found in Murray's \textit{Asymptotic Analysis} \cite{murray2012asymptotic}. With $\epsilon$ being the small quantity that dictates our 'slow' time, we choose to use an asymptotic expansion of $x$ with

\begin{equation}\label{eq:oneD_slow_asympexpan}
x(\tau)\sim x_0(\tau)+\epsilon x_1(\tau)+\epsilon^2 x_2(\tau)+O(\epsilon^3).
\end{equation}

This approach captures the slowly varying behavior of the solution in terms of this small quantity $\epsilon$ and aims to relate the slow variation to the solution. We substitute the expansion \eqref{eq:oneD_slow_asympexpan} into the scaled system \eqref{eq:oneD_slow_scaled} to get

\begin{equation*}
\epsilon {x_0}_\tau +\epsilon^2 {x_1}_\tau+\ldots= -\mu(\tau) -2x_0+x_0^2+\epsilon(-2x_1+2x_1x_0)+\epsilon^2(-2x_1+2x_2x_0+x_1^2)+\ldots
\end{equation*}

Once we separate the equations at each order, we find the following system of equations

\begin{align}
\label{eq:oneD_slow_outerO1}
O(1):& \quad 0=-\mu(t)-2x_0+x_0^2,\\
\label{eq:oneD_slow_outerO2}
O(\epsilon):& \quad {x_0}_\tau=-2x_1+2x_1 x_0,\\
\label{eq:oneD_slow_outerO3}
O(\epsilon^2):& \quad {x_1}_\tau=-2x_2+2x_2x_0+x_1^2.
\end{align}

We then solve each equation \eqref{eq:oneD_slow_outerO1}-\eqref{eq:oneD_slow_outerO3} progressively **** Should the work for this be shown in the appendix? Its only algebra **** to find the terms of our asymptotic expansion \eqref{eq:oneD_slow_asympexpan} as

\begin{equation}\label{eq:oneD_slow_outersoln}
x(t)\sim 1-\sqrt{1+\mu(t)}+ \frac{\epsilon}{4(1+\mu(t))}-\frac{3\epsilon^2}{32(1+\mu(t))^{5/2}}+O(\epsilon^3).
\end{equation}

We call \eqref{eq:oneD_slow_outersoln} the outer solution as it approximates the solution well for large values of $x(t)$. But since the dynamics of the system \eqref{eq:oneD_canonical} change at $x=0$ due to the non-smoothness of the problem, this solution is valid only for $x<0$ and $\mu>0$. This gives rise to a critical point at $(\mu_c,x_c)=(0,0)$ which is the non-smooth bifurcation, and a local analysis about $x=0$ is necessary.

It is a key assumption of an asymptotic expansion that the terms in are clearly separated by order of $\epsilon$. To perform the inner analysis, we search for a scaling of $\mu$ and $x$ for which \eqref{eq:oneD_slow_outersoln} is no longer valid under this assumption of order separation, here occurring when $x_0\sim \epsilon x_1$. We suspect that $\mu\sim O(\epsilon)$ causes the expansion to fail, but we conduct a simple scale analysis to determine the appropriate scaling for the local region about $x=0$. We consider the general scales

\begin{equation*}
x=\epsilon^\alpha y,\quad \mu = \epsilon^\beta m,
\end{equation*}

with $\alpha$ and $\beta$ being some positive number to be an inner scaling. We apply this scaling in \eqref{eq:oneD_canonical} to give the system

\begin{equation}\label{eq:oneD_slow_scalesearch}
\begin{aligned}
\epsilon^\alpha \dot{y}=&-\epsilon^\beta m +\epsilon^\alpha 2|y|-\epsilon^{2\alpha}y|y|,\\
 \epsilon^\beta \dot{m}=&-\epsilon.
\end{aligned}
\end{equation}

We balance the leading order terms $\epsilon^\alpha\dot{y}$ with $\epsilon^\beta m$ to give that $\alpha=\beta$. But the equation for $m$ gives that $\beta=1$, thus we have the scaling for the local analysis

\begin{equation}\label{eq:oneD_slow_scales}
x=\epsilon y,\quad \mu=\epsilon m.
\end{equation}

Now, we have found that the scalings in \eqref{eq:oneD_slow_scales} apply to all $x$ and thus we consider the region of $x>0$. Substituting the scales into \eqref{eq:oneD_canonical} we find the following inner system for the region of $x>0$

\begin{equation}\label{eq:oneD_slow_innereq}
\begin{aligned}
\dot{y}=&-m(t)+2 y-\epsilon y^2,\\
\dot{m}=&-1.
\end{aligned}
\end{equation}

We recall that we are searching for a link between $y$ and $m$, it is then convenient to change the differentiation on $y$ to be with respect to the parameter $m$. This incorporates the behavior of $m(t)$ directly into the equation we solve and gives us a direct method for finding the tipping. Then the leading order is

\begin{equation}\label{eq:oneD_slow_innerm}
y_m = m-2y.
\end{equation}

Where the leading order solution to \eqref{eq:oneD_slow_innerm} is found explicitly as follows

\begin{equation*}
y(m) = C e^{-2m}+\frac{m}{2}-\frac{1}{4}+O(\epsilon).
\end{equation*}

With the inner solution found in terms of the parameter, we write this in terms of the original coordinates with

\begin{equation}\label{eq:oneD_slow_innersoln}
x(t)\sim \epsilon Ce^{-2\mu(t)/\epsilon}+\frac{\mu(t)}{2}+O(\epsilon).
\end{equation}

Since the solution \eqref{eq:oneD_slow_innersoln} behaves exponentially, the tipping point occurs when the exponential term begins to grow rapidly, here we consider this to be $O(1/\epsilon)$. This behavior occurs when

\begin{equation}\label{eq:oneD_slow_tipping}
\mu_{\text{slow}}= \frac{1}{2}\epsilon \log (\epsilon).
\end{equation}

Thus we have the tipping point for the purely smooth one-dimensional model. Notice that for small values of $\epsilon$, the slowly varying parameter causes tipping to occur when $\mu(t)<0$; which is after the non-smooth bifurcation and is consistent with considering the inner equation \eqref{eq:oneD_slow_innereq} for the region $x>0$ as we found in the analysis. Thus we find that a slow varying parameter causes a delay in the non-smooth bifurcating behavior for this problem and we expect the solution to remain in a state of bi-stability for longer than the static problem. In terms of hysteresis, then slow variation allows for a longer period before the states switch from the lower branch to the upper.

In figure~\ref{fig:oneD_slow_numerics} (a,b), an example of the tipping occurring is given for a choice in $\epsilon$ along with the standard bifurcation diagram where (c) demonstrates the tipping approximation across a range of $\epsilon$. The concavities match as well as clear agreement in the estimation and bifurcation as $\epsilon$ goes to 0.

\begin{figure}[H]
\centering
\begin{subfigure}{.5\textwidth}
  \centering
  \includegraphics[width=\linewidth]{oneD/slow_bif_diagram.jpg}
  \caption{}
\end{subfigure}%
\begin{subfigure}{.5\textwidth}
  \centering
  \includegraphics[width=\linewidth]{oneD/slow_bif_diagram_zoom.jpg}
  \caption{}
\end{subfigure}
\begin{subfigure}{.5\textwidth}
\centering
\includegraphics[width=\linewidth]{oneD/slow_epscomp.jpg}
\caption{}
\label{fig:oneD_slow_comp}
\end{subfigure}
\caption{In (a) the numerical solution (black dotted line) to \eqref{eq:oneD_canonical} is given with $A=0$ and $\epsilon=.01$. The bifurcation plot is overlayed for convenience. In (b) a zoom in of what happens near the non-smooth bifurcation. The solid vertical lines dictate the tipping estimate (blue) and the dotted vertical line is the numerical tipping, when $x>.5$. In (c) a range of $\epsilon$ and their corresponding tipping (red stars) are compared to our estimate (solid black line) from \eqref{eq:oneD_slow_tipping}.}
\label{fig:oneD_slow_numerics}
\end{figure}


\subsection{Stability}
From the basic model we know our outer solution \eqref{eq:oneD_slow_outersoln} to be stable, but to verify the inner solution \eqref{eq:oneD_slow_innersoln} we found is stable, we use a simple linear perturbation on the inner system. Typically to do this, an analysis would be performed about an equilibrium to see if perturbations would grow or decay. But as this problem has a parameter that is allowed to vary, we instead must be careful to note the analysis must be done on the pseudo-equilibrium instead. In the first region of interest, $m(t)\ge 0$, the following inner equation and pseudo-equilibrium hold below the axis

\begin{equation}\label{eq:oneD_slow_stability1}
\dot{y}=-m(t)-2y=f(t,y), \quad z^0(t)=-\frac{m(t)}{2}.
\end{equation}

We then consider simple perturbations about the pseudo-equilibrium in \eqref{eq:oneD_slow_stability1} with

\begin{equation*}
y(t)=z^0(t)+u(t), \quad \lVert u(t) \rVert \ll 1.
\end{equation*}

Here we must treat the pseudo-equilibrium with care, normally a Taylor expansion would result in expressing the perturbations with their own equation that we could use to determine stability. Since $z^0(t)$ isn't fixed here, we must consider its contribution to the derivative in this region of the parameter space (i.e $m(t)\ge 0$). Thus we find

\begin{equation}\label{eq:oneD_slow_stability2}
\begin{aligned}
\dot{y} =& \dot{z^0}+\dot{u},\\
\dot{z^0}= & \begin{cases}
\frac{1}{2} & m(t)>0,\\
0 & m(t)=0.
\end{cases}
\end{aligned}
\end{equation}

Now we apply the standard Taylor expansion to see the behavior of perturbations, with \eqref{eq:oneD_slow_stability2}, \eqref{eq:oneD_slow_stability1} becomes

\begin{equation}\label{eq:oneD_slow_perturbeq}
\begin{aligned}
\dot{y}=& f(t,z^0)+f_y(t,z^0)(y-z^0)
=& f_y(t,z^0)u,\\
\dot{u}=&\begin{cases}
-\frac{1}{2}-2u, & m(t)>0,\\
-2u, & m(t)=0.
\end{cases}
\end{aligned}
\end{equation}

If this were the fixed parameter problem, we would always have the second case in \eqref{eq:oneD_slow_perturbeq}, which is always stable due to the sign. But since we allow for a varying parameter, we learn that the solution is attracted to just below the pseudo-equilibrium $z^0(t)$. But as this system always experiences the critical point $m_c=0$ due to the smooth decrease in $m(t)$, the slowly varying parameter eventually acts like the fixed parameter in \autoref{sec:oneD_static}. Hence we have that for $x<0$, the pseudo-equilibrium is hyperbolic and asymptotically stable. But as we had noticed from our analysis, there is a critical point $(\mu_c,x_c)=(0,0)$ which corresponds to a non-hyperbolic equilibrium point. Generally, non-hyperbolic behavior signals equilibrium structure to be changing. Here, this signals a transition in behavior for $x>0$ and helps identify that the tipping occurs here.

For the second region of interest, $m(t)<0$, we found a solution that had the following inner equation which has the pseudo-equilibrium above the axis with

\begin{equation}\label{eq:oneD_slow_innerstability}
\dot{y}=-m(t)+2y, \quad z^0(t) = \frac{m(t)}{2}.
\end{equation}

But we find a contradiction with \eqref{eq:oneD_slow_innerstability}, here $m(t)<0$ yet and the solution of this region is supposed to be above the axis. Thus we may conclude that this inner equation has no equilibrium in this region and further verifies that the critical point $(\mu_c,x_c)$ was non-hyperbolic and tipping occurs for $m(t)<0$.

\section{High Frequency Oscillatory Forcing}
\label{sec:oneD_highfreqosc}

To understand the oscillatory forcing in the Stommel model, consider the canonical system \eqref{eq:oneD_canonical} with $A\sim O(1)$, $\Omega\gg 1$ and $\epsilon=0$, which gives purely high frequency oscillatory forcing in the system. Under these conditions, we are back to seeing fixed parameter values but we have equilibrium points for each parameter value but with the oscillatory forcing there are oscillations about this point. Thus we should expect to find a regular bifurcation influenced by oscillations occurring under these conditions. Thus, we develop a method to find the equilibrium up to the oscillations to determine what the effect of oscillatory forcing has on the bifurcation of \eqref{eq:oneD_canonical}. Where \autoref{sec:oneD_slow} focused only on the slowly varying dynamics, here we have both a 'slow' time scale $t$ and a 'fast' time scale $T=\Omega t$. This naturally suggests a multiple scales approach where we search for a solution that is dependent on both of these scales, $x(t)=x(t,T)$. This method is commonly used in problems that have behavior observable on multiple scalings, and we use it here to find a way to accurately analyze each scale and effectively combine their behavior into a single unifying solution. Further discussion on this method can be found in Sanchez \cite{sanchez1996method}.

Recall that our focus is on the non-smooth behavior and hence we restrict the solution to follow along the lower stable equilibrium branch where $x<0$. Using this multiple scales approach, our canonical system \eqref{eq:oneD_canonical} has the following form

\begin{equation}\label{eq:oneD_osc_multiscale}
x_T+\Omega^{-1}x_t=\Omega^{-1}\left(-\mu-2x+x^2+A\sin(T)\right).
\end{equation}

From \eqref{eq:oneD_osc_multiscale}, we see the small quantity $\Omega^{-1}$ appearing which suggests an asymptotic expansion in powers of this quantity

\begin{equation}\label{eq:oneD_osc_asymptotic}
x(t,T)\sim x_0(t,T)+\Omega^{-1}x_1(t,T)+\Omega^{-2}x_2(t,T)+O(\Omega^{-3}).
\end{equation}

Substituting \eqref{eq:oneD_osc_asymptotic} in our multiple scales system \eqref{eq:oneD_osc_multiscale}, we find

\begin{equation*}
{x_0}_T+\Omega^{-1}{x_0}_t+\Omega^{-1}{x_1}_T+\ldots=\Omega^{-1}(-\mu-2x_0+x_0^2+A\sin(T))+\Omega^{-2}(-2x_1+2x_1x_0)+\ldots
\end{equation*}

Where we separate the reduced equations at order of $\Omega$ to get

\begin{align}
\label{eq:oneD_osc_outerO1}
O(1):& \quad {x_0}_T=0, \\
\label{eq:oneD_osc_outerO2}
O(\Omega^{-1}):& \quad {x_1}_T+{x_0}_t =-\mu-2x_0+x_0^2+A\sin(T),\\
\label{eq:oneD_osc_outerO3}
O(\Omega^{-2}):& \quad {x_2}_T + {x_1}_t= -2x_1+2x_0 x_1.
\end{align}

With an equation at each order, we must able to solve each to proceed to the next but we must also further restrict our solution from having resonant or linearly growing terms to prevent any multiplicity or exponential growth. This assures that the terms in the asymptotic expansion are compatible with one another and we find a robust solution. A common method to guarantee compatible solutions at each order are found with less than linearly growing terms is the Fredholm alternative. This provides a solvability condition for each equation of the form ${x_i}_T=R_i(t,T)$ with

\begin{equation*}
\lim\limits_{T\to\infty}\frac{1}{T}\int_0^T R_i(t,u)\,du=0,
\end{equation*}

which for this system we consider the periodic form of the Fredholm alternative

\begin{equation} \label{eq:Fredholm}
\frac{1}{2\pi}\int_0^{2\pi}R_i(t,T)\,dT=0.
\end{equation}

Both the general and periodic form of the Fredholm alternative have been well studied, a more theoretic approach to the periodic version is discussed in Bensoussan's \textit{Asymptotic analysis for periodic structures} \cite{bensoussan2011asymptotic}. From \eqref{eq:oneD_osc_outerO1}, we learn the leading order term is only dependent on the 'slow' time, $x_0=x_0(t)$. Applying the Fredholm alternative \eqref{eq:Fredholm} to \eqref{eq:oneD_osc_outerO2} gives

\begin{equation}\label{eq:oneD_osc_outerO2soln}
\begin{aligned}
0=&\frac{1}{2\pi}\int_0^{2\pi} -{x_0}_t(t) -\mu -2x_0(t)+x_0(t)^2+A\sin(T)\,dT ,\\
{x_0}_t=& -\mu -2x_0+x_0^2 ,\\
{x_1}_T =& A\sin(T).
\end{aligned}
\end{equation}

Searching for the equilibrium solution of \eqref{eq:oneD_osc_outerO2soln} leads to the leading order solution, $x_0$ but also allows us to partially solve the first correction term $x_1$ with

\begin{equation*}
\begin{aligned}
x_0 =& 1-\sqrt{1+\mu},\\
x_1(t,T) =& v_1(t) - A\cos(T).
\end{aligned}
\end{equation*}

Repeating this procedure with \eqref{eq:oneD_osc_outerO3}, the work for this is found in \autoref{app:oneD}, results in the terms of the expansion \eqref{eq:oneD_osc_asymptotic} with original coordinates

\begin{equation}\label{eq:oneD_osc_outersoln}
x\sim 1-\sqrt{1+\mu}-\Omega^{-1} A \cos(\Omega t)+O(\Omega^{-2}).
\end{equation}

Once again, the explicit outer solution \eqref{eq:oneD_osc_outersoln} performs well for large $x$, we search for when the assumptions of the asymptotic series fail to see where an inner analysis is needed. This occurs when $x_0\sim \epsilon x_1$ which we suspect happens for $\mu\sim O(\Omega^{-1})$.

We consider a general scaling for $x=\Omega^{-\alpha}y$ and $\mu = \Omega^{-\beta}m$ where since we are searching for an inner equation, we anticipate $\alpha>0$ and $\beta>0$. Applying these scalings to \eqref{eq:oneD_canonical} results in

\begin{equation}\label{eq:oneD_osc_generalinner}
\dot{y} = -\Omega^{\alpha-\beta}m+2|y|-\Omega^{-\alpha}y|y|+\Omega^{\alpha}A\sin(\Omega t)
\end{equation}

Since the parameter is fixed for this section, we are still able to use the same scales, $t=t$ and $T=\Omega t$, with the assumption that $y(t)=y(t,T)$, and hence a similar multiple scales argument in \eqref{eq:oneD_osc_generalinner} leads to

\begin{equation}\label{eq:oneD_osc_innergeneralmulti}
y_T+\Omega^{-1}y_t = - \Omega^{\alpha-\beta-1}m+\Omega^{-1}2|y|-\Omega^{\alpha-1}y|y|+\Omega^{-\alpha-1}A\sin(T).
\end{equation}

With a standard balancing argument between the leading order terms in \eqref{eq:oneD_osc_innergeneralmulti},$y_T$ and $\Omega^{\alpha-1} A\sin(T)$, we see that $\alpha=1$. But we also want to see the terms $\Omega^{\alpha-\beta-1}m$ communicating with $\Omega^{-1}2|y|$, which gives us that $\beta=1$ as well. This results in the inner equation

\begin{equation}\label{eq:oneD_osc_naivemultiscales}
y_T+\Omega^{-1}y_t = \Omega^{-1}\left(-m+2|y|\right)-\Omega^{-2}y|y|+A\sin(T).
\end{equation}

Similarly to the outer equation, we have a need for the asymptotic expansion in terms of $\Omega^{-1}$ with

\begin{equation}\label{eq:oneD_osc_innerasymptotic}
y(t,T)\sim y_0(t,T)+\Omega^{-1}y_1(t,T)+O(\Omega^{-2}).
\end{equation}

Substituting the expansion \eqref{eq:oneD_osc_innerasymptotic} into the inner multiple scales system \eqref{eq:oneD_osc_naivemultiscales} we find

\begin{equation*}
{y_0}_T+\Omega^{-1}{y_0}_t+\Omega^{-1}{y_1}_T+\ldots =\begin{aligned}[t]\Omega^{-1}&(-m+2|y_0+\Omega^{-1}y_1+\ldots|)+A\sin(T)\\
&+\Omega^{-2}(y_0+\Omega^{-1}y_1+\ldots)|y_0+\Omega^{-1}y_1+\ldots|
\end{aligned}
\end{equation*}

Where we then find the following reduced equations at each order of $\Omega^{-1}$ with the system

\begin{align}
\label{eq:oneD_osc_innerO1}
O(1):\quad & {y_0}_T = A\sin(T),\\
\label{eq:oneD_osc_innerO2}
O(\Omega^{-1}):\quad & {y_1}_T+{y_0}_t =  -m+2|y_0|.
\end{align}

Solving the leading order equation \label{eq:oneD_osc_innerO1} gives that the leading order term has the form, $y_0(t,T)=v_0(t)-A\cos(T)$. But applying the Fredholm alternative \eqref{eq:Fredholm} on \eqref{eq:oneD_osc_innerO2} leads to

\begin{equation}\label{eq:oneD_osc_integral}
{v_0}_t(t)=-m+\frac{1}{\pi}\int_0^{2\pi} |v_0(t)-A\cos(T)|\,dT.
\end{equation}

Here we must consider two cases of $v_0(t)$ that determine the difficulty of this integrand, Case I: if $v_0(t)$ is large enough to keep the interior from ever changing signs and Case II: if $v_0(t)$ is too small and the interior changes sign. In figure~\ref{fig:oneD_osc_cases} we see the behavior of each case where the solution on the right is following under Case I, the first vertical line defining the parameter range between the cases, the middle region following under Case II and the second vertical line giving the bifurcation.

\begin{figure}[H]
\centering
\includegraphics[width=.7\textwidth]{oneD/osc_cases.jpg}
\caption{The parameter range is shown here with Case I being to the right of the vertical green line and Case II being in between the vertical blue and green lines. For reference, the original bifurcation diagram is overlayed.}
\label{fig:oneD_osc_cases}
\end{figure}

\subsection{Case I: $v_0(t) \le -|A|$}
\label{subsec:oneD_osc_CaseI}

We call this the entirely below axis case, the solution stays far from the axis $x=0$ for most of it's oscillation and thus the behavior is consistent and controllable. We don't expect to see the bifurcation occur under these conditions but instead we find the parameter range for the cases. Here the integral equation \eqref{eq:oneD_osc_integral} simplifies nicely to a simple equation that we can find the equilibrium of

\begin{equation*}
{v_0}_t(t)=-m-2v_0(t),\quad v_0(t)=-\frac{m}{2}.
\end{equation*}

Which gives the leading order equilibrium solution with oscillations to our inner equation for this case which we write in original coordinates

\begin{equation}\label{eq:oneD_osc_caseIsoln}
\begin{aligned}
y(t,T)\sim& -\frac{m}{2}-A\cos(T)+O(\Omega^{-1}),\\ 
x(t)\sim& -\frac{\mu}{2}-\Omega^{-1} A\cos(\Omega t)+O(\Omega^{-2}).
\end{aligned}
\end{equation}

The condition $v_0(t)\le -|A|$ combined with the equilibrium allows us to establish when \eqref{eq:oneD_osc_caseIsoln} holds

\begin{equation}\label{eq:oneD_osc_caseboundary}
\mu\ge \frac{2|A|}{\Omega}.
\end{equation}

Following the equilibrium to \eqref{eq:oneD_osc_caseboundary} leads us to Case II where we see the oscillations crossing the axis and the assumptions of this case no longer hold.

\subsection{Case II: $|v_0(t)|< |A|$}
\label{subsec:oneD_osc_CaseII}

We call this the crossing case; here the equilibrium is small enough that the oscillations can now push the equilibrium above the axis. Under these conditions, the solution spends most of its time near the axis $x=0$ and thus experiences much of the non-smooth influence. As the crossing continues, the solution is gradually becoming more and more uncontrollable and thus we expect to find the bifurcation here. From \eqref{eq:oneD_osc_caseboundary}, we have a range of $\mu$ for when this case applies, $\mu<\frac{2|A|}{\Omega}$. But the integrand in \eqref{eq:oneD_osc_integral} is non-trivial when $|v_0(t)|<|A|$. In order to deal with the irregular sign changing inside the integral, we break the integration into regions based on sign. Recall that we are searching for equilibrium behavior, and so we may make the assumption that we are dealing with a fixed value of $v_0$ such that $|v_0|\le |A|$. In figure~\ref{fig:oneD_osc_t1t2_graphic} we observe the function that we are integrating over.

\begin{figure}[H]
\centering
\includegraphics[width=0.8\textwidth]{oneD/t1t2_graphic.jpg}
\caption{The non-smooth function $|y_0|=|v_0-A\cos(T)|$ that we integrate over is shown as the solid red line. We also show an example choice of $v_0$ as a horizontal blue dotted line. Here the value of $|v_0|\le|A|$, which causes kinks to appear at the roots, $T_1$ and $T_2$ respectively which are vertical black dashed dotted lines. }
\label{fig:oneD_osc_t1t2_graphic}
\end{figure}

We now consider the following roots of the integrand with 

\begin{equation*}
T_1=\arccos (v_0/A),\qquad T_2= 2\pi - \arccos (v_0/A).
\end{equation*}

Where we notice $0<T_1<T_2<2\pi$ and that the sign of the integrand stays the same on each interval. We only assume that the first interval is following the solution while the center is still negative, thus the integrand will also be negative. From this, we expand the integral as

\begin{equation}\label{eq:oneD_osc_expandedintegral}
\begin{aligned}
\int_0^{2\pi}|v_0-A\cos(T)|\,dT=&-\int_0^{T_1}v_0-A\cos(T)\,dT+\\
&\int_{T_1}^{T_2}v_0-A\cos(T)\,dT-\int_{T_2}^{2\pi}v_0-A\cos(T)\,dT.
\end{aligned}
\end{equation}

Evaluating \eqref{eq:oneD_osc_expandedintegral} and using a trig identity, $\sin(\arccos(x))=\sqrt{1-x^2}$, we find the integral to be

\begin{equation*}
\int_0^{2\pi}|v_0-A\cos(T)\,dT=\frac{2}{\pi}\left(\arcsin(v_0/A)v_0+\sqrt{A^2-v_0^2}\right).
\end{equation*}

Now, we notice that our argument above is simple for $v_0$ in equilibrium, but we may even take advantage of our 'fast' time with $t\ll T$ implying that $v_0(t)$ is approximately fixed over $T\in [0,2\pi]$. This holds true due to having a high frequency $\Omega$ and otherwise would not be a valid approximation. Thus we can evaluate \eqref{eq:oneD_osc_integral} to find the inner equation 

\begin{equation}\label{eq:oneD_osc_caseIIexact}
{v_0}_t=-m+\frac{4}{\pi}\left(\arcsin(v_0/A)v_0+\sqrt{A^2-v_0^2}\right).
\end{equation}

But in its current form, \eqref{eq:oneD_osc_caseIIexact} restricts any analytic insight due to its difficulty, so we use a quadratic Taylor approximation to be able to solve this equation explicitly

\begin{equation}\label{eq:oneD_osc_caseIItaylor}
{v_0}_t \approx -m + \frac{4|A|}{\pi} + \frac{2}{\pi |A|}v_0^2,
\end{equation}

which has the following equilibrium with positive constant $C$

\begin{equation}\label{eq:oneD_osc_caseIIequil}
v_0=-C\sqrt{m-\frac{4|A|}{\pi}}.
\end{equation}

Thus we have the leading order inner equilibrium solution \eqref{eq:oneD_osc_caseIIequil}, we translate back into the original coordinates,

\begin{equation}\label{eq:oneD_osc_innersoln}
\begin{aligned}
y\sim& -C\sqrt{m-\frac{4|A|}{\pi}}-A\cos(T)+O(\Omega^{-1}),\\ 
x(t)\sim& -C\sqrt{\Omega^{-1} \left(\mu-\frac{4|A|}{\pi \Omega}\right)}-\Omega^{-1} A\cos(\Omega t)+O(\Omega^{-2}).
\end{aligned}
\end{equation}

It then is clear that the bifurcation happens when \eqref{eq:oneD_osc_innersoln} fails, here being when the square root no longer makes sense

\begin{equation}\label{eq:oneD_osc_bif}
\mu_{\text{osc}}=\frac{4|A|}{\pi \Omega}.
\end{equation}

From the result \eqref{eq:oneD_osc_bif}, we gather that the oscillatory forcing in the system causes the bifurcation to occur sooner and that this is controlled by the size of $A$ and $\Omega$. This should be expected as we are causing the model to experience the non-smooth behavior sooner with the oscillations as opposed to later with the slow variation. This effect is opposite that of the slow variation where the solution experienced a delayed tipping, it now experiences an accelerated bifurcation. This also indicated that the region of bi-stability is shrunk with oscillatory forcing and thus can be used to eliminate the region entirely with $A$ and $\Omega$ chosen properly.  We now compare our estimate to numerical results for varying sizes of $\Omega^{-1}$.

\begin{figure}[H]
\centering
\begin{subfigure}{.5\textwidth}
  \centering
  \includegraphics[width=\linewidth]{oneD/osc_timeseries.jpg}
  \caption{}
\end{subfigure}%
\begin{subfigure}{.5\textwidth}
  \centering
  \includegraphics[width=\linewidth]{oneD/osc_bif_diagram.jpg}
  \caption{}
\end{subfigure}
\begin{subfigure}{.5\textwidth}
  \centering
  \includegraphics[width=\linewidth]{oneD/osc_bif_diagram_zoom.jpg}
  \caption{}
\end{subfigure}%
\begin{subfigure}{.5\textwidth}
\centering
\includegraphics[width=\linewidth]{oneD/osc_Omegacomp.jpg}
\caption{}
\label{fig:oneD_osc_comp}
\end{subfigure}
\caption{In (a) the numerical time series solutions to \eqref{eq:oneD_canonical} are given from bottom to top with $\mu$ in Case I, Case II and after the bifurcation respectively with $A=2$, $\Omega=10$ and $\epsilon=0$. In (b) we show the time series in the phase plane. In (c) a zoom in closer to the non-smooth bifurcation, where the dotted vertical lines dictate the region between Case I and Case II (green) as well as the bifurcation estimate (blue) respectively. The dotted vertical line is the numerical bifurcation, when $x>.2$. In (d) a range of inverse frequencies and their corresponding bifurcations (red stars) are compared to our prediction (black solid line).}
\label{fig:oneD_osc_numerics}
\end{figure}

In figure~\ref{fig:oneD_osc_numerics} an example is given of the effect oscillatory forcing has for a choice of $A$ and $\Omega$ with (a), (b) and (c), but (c) shows the bifurcation approximation across a range of $\Omega^{-1}$. There is an allowed range of $\Omega$ from our assumption of $\Omega \gg 1$, which in this region we see agreement. The concavity is well represented and the behavior as $\Omega^{-1}\to 0$ converges to the standard bifurcation. This shows that our methodology will work well in the Stommel problem.

\subsection{Stability}

Once more, we have that the outer solution \eqref{eq:oneD_osc_outersoln} is stable from the basic model. In this section, we have two regions of interest and care to establish their stabilities agree with our analysis. Each region has a particular version of the same inner equation dictating their solution's behavior

\begin{equation}\label{eq:oneD_osc_stabilityI}
{v_0}_t(t)=-m+\frac{1}{\pi}\int_0^{2\pi}|v_0(t)-A\cos(T)|\,dT=f(v_0(t)).
\end{equation}

\subsubsection{Case I: $v_0(t)< -|A|$}

In this region we didn't find any bifurcating behavior and \eqref{eq:oneD_osc_stabilityI} simplifies and has the inner equation and equilibrium as follows

\begin{equation*}
{v_0}_t(t)=-m-2v_0(t)=f(v_0), \quad z^0=-\frac{m}{2}.
\end{equation*}

Similarly to \autoref{sec:oneD_slow}, we have a fixed parameter equation that we have shown to cause perturbations to decay exponentially and hence we find the equilibrium to be hyperbolic and asymptotically stable in the paramter range found in the analysis \eqref{eq:oneD_osc_caseboundary}

\begin{equation*}
\mu \ge \frac{2|A|}{\Omega}.
\end{equation*}

\subsubsection{Case II: $|v_0(t)|<|A|$}

For this region we did find the bifurcating behavior to occur and we found the Taylor approximation \eqref{eq:oneD_osc_caseIItaylor} for the inner equation and equilibrium to be

\begin{equation}\label{eq:oneD_osc_stabilityII}
{v_0}_t(t)= -m +\frac{4|A|}{\pi}+\frac{2}{\pi |A|}v_0(t)^2,\quad z^0=-C \sqrt{m-\frac{4|A|}{\pi}}.
\end{equation}

We consider a simple linear perturbation of \eqref{eq:oneD_osc_stabilityII}, $v_0(t)=z^0+u(t)$ with $\lVert u(t) \rVert \ll 1$. Applying the standard Taylor expansion to determine the equation for the perturbations, we find

\begin{equation}\label{eq:oneD_osc_innerpertubation}
\begin{aligned}
{v_0}_t(t) =& f(z^0)+f_{v_0}(z^0)(v_0(t)-z^0)+ O(\lVert v_0(t)-z^0\rVert^2),\\ 
u_t(t) =& f(z^0)+f_{v_0}(z^0)u(t)+O(\lVert u(t) \rVert^2),\\
u_t(t) =& -2\sqrt{m-\frac{4|A|}{\pi}} u(t) .
\end{aligned}
\end{equation}

From \eqref{eq:oneD_osc_innerpertubation}, the sign gives that the perturbations decay exponentially and hence the equilibrium is hyperbolic and asymptotically stable as long as $m>\frac{4|A|}{\pi}$ or $\mu>\frac{4|A|}{\pi \Omega}$. We find that once $\mu$ reaches the value in \eqref{eq:oneD_osc_bif}, then the stability of \eqref{eq:oneD_osc_innerpertubation} is non-hyperbolic. When the stability switches like this, we expect a bifurcation, thus we have further evidence to support that \eqref{eq:oneD_osc_bif} is the oscillatory bifurcation we seek.

\section{Slowly Varying and Oscillatory Forcing}
\label{sec:oneD_slowosc}

Since we have established an approach for each feature of the problem individually, we combine them in the full one-dimensional model \eqref{eq:oneD_canonical} with $\epsilon\ll 1$ and ${A\sim O(1)}$. Once again, due to the slow variation in $\mu$, we wont see a bifurcation occurring for under these conditions but rather a tipping point. Hence we must find the behavior of the solution and search for when it becomes uncontrollable. We also seek a relationship between the slow drift and the high frequency, so we consider a generic $\Omega = \epsilon^{-\lambda}$ with strength parameter $\lambda>0$ to determine this relationship. With a general $\lambda$, we  classify regions of behavior with ranges of $\lambda$ and will be able to find the boundaries of high frequency with respects to the slow variation. With both mechanisms in effect, we again choose to use a multiple scales approach that captures both slow behavior and fast oscillations. But now, we truly have 'slow' time, the slowly varying parameter $\mu(t)$, as well as 'fast' time, the rapid oscillations $\sin(\Omega t)$. The choice in scales would then be $\tau=\epsilon t$ and $T=\epsilon^{-\lambda} t$, which leads to the system

\begin{equation*}
\begin{aligned}
x_T+\epsilon^{\lambda+1}x_\tau =& \epsilon^\lambda (-\mu(\tau)+2|x|-x|x|+A\sin(T)),\\
\mu_\tau(\tau)=&-1.
\end{aligned}
\end{equation*}

Once again, we begin in the region $x<0$ before any crossing occurs to find the outer solution of the non-smooth behavior. Thus we have the system

\begin{equation}\label{eq:oneD_slowosc_outereq}
\begin{aligned}
x_T+\epsilon^{\lambda+1}x_\tau =& \epsilon^\lambda (-\mu(\tau)-2x+x^2+A\sin(T)),\\
\mu_\tau(\tau)=&-1.
\end{aligned}
\end{equation}

We consider an asymptotic expansion in terms of the small quantity $\epsilon^\lambda$ which is in terms of the generic frequency relationship 

\begin{equation}\label{eq:oneD_slowosc_asympexpansion}
x(\tau,T)\sim x_0(\tau,T)+\epsilon^\lambda x_1(\tau,T)+O(\epsilon^{1+\lambda},\epsilon^{2\lambda}).
\end{equation}

Depending on the value of $\lambda$, $O(\epsilon^{\lambda+1})$ may be the next order before $O(\epsilon^{2\lambda})$, although both produce the same equation at their respective order and hence our choice in $\lambda$ doesn't change the calculations up to this correction term. Introducing the expansion \eqref{eq:oneD_slowosc_asympexpansion} into the outer multi-scaled equation \eqref{eq:oneD_slowosc_outereq} gives 

\begin{equation}\label{eq:oneD_slowosc_outerexplicit}
{x_0}_T+\epsilon^{\lambda+1}{x_0}_\tau+\epsilon^\lambda {x_1}_T+\ldots=\epsilon^\lambda(-\mu(\tau)-2x_0+x_0^2+A\sin(T))+\epsilon^{2\lambda}(-2x_1+x_1x_0)+\ldots
\end{equation}

Where we separate \eqref{eq:oneD_slowosc_outerexplicit} to find equations at each order of $\epsilon^\lambda$, but we consider the system with $O(\epsilon^{2\lambda})$ here

\begin{align}
\label{eq:oneD_slowosc_outerO1}
O(1):\quad & {x_0}_T=0, \\
\label{eq:oneD_slowosc_outerO2}
O(\epsilon^\lambda):\quad&  {x_1}_T=-\mu(\tau)-2x_0+x_0^2+A\sin(T),\\
\label{eq:oneD_slowosc_outerO3}
O(\epsilon^{2\lambda}):\quad& {x_2}_T+\epsilon^{1-\lambda}{x_0}_\tau= -2x_1+2x_0x_1.
\end{align}

Each equation reveals more about the behavior of each order of the solution; \eqref{eq:oneD_slowosc_outerO1} indicates that the leading order term is purely slow dependent, $x_0=x_0(\tau)$. In \autoref{app:oneD} we apply the Fredholm alternative \eqref{eq:Fredholm} to \eqref{eq:oneD_slowosc_outerO2} and \eqref{eq:oneD_slowosc_outerO3} to find the first few terms of our expansion \eqref{eq:oneD_slowosc_asympexpansion} explicitly, the resulting solution is

\begin{equation}\label{eq:oneD_slowosc_outersoln}
x\sim 1-\sqrt{1+\mu(t)}-\frac{\epsilon}{4(1+\mu(t))}-\epsilon^\lambda A \cos(\Omega t)+O(\epsilon^{1+\lambda},\epsilon^{2\lambda}).
\end{equation}

With the outer solution \eqref{eq:oneD_slowosc_outersoln}, we search for when the terms violate the assumptions of the expansion to find where we need to form an inner equation. This happens either when $x_0\sim O(\epsilon)$ or when $x_0\sim O(\epsilon^\lambda)$ where we suspect is once $\mu\sim O(\epsilon)$ or $\mu\sim O(\epsilon^\lambda)$ respectively, the occurrence of which depends on $\lambda$. 

To find an inner equation we chose a general scaling for both $x$ and $\mu$ given the ambiguity of choice in $\mu$ with

\begin{equation}\label{eq:oneD_slowosc_general_scaling}
x=\epsilon^\alpha y ,\quad \mu(t)=\epsilon^\beta m(t),
\end{equation}

where we anticipate $\alpha>0$ and $\beta>0$. Applying the scaling \eqref{eq:oneD_slowosc_general_scaling} to the canonical equation \eqref{eq:oneD_canonical} gives

\begin{equation}\label{eq:oneD_slowosc_innerscaled}
\begin{aligned}
\epsilon^\alpha \dot{y}=& -\epsilon^\beta m(t)+\epsilon^\alpha 2|y| - \epsilon^{2\alpha}y|y| +A\sin(\epsilon^{-\lambda}t),\\
\dot{m}(t)=&-\epsilon^{1-\beta}.
\end{aligned}
\end{equation}

From \eqref{eq:oneD_slowosc_innerscaled} we find the 'fast' time still appears but the 'slow' time has multiple choices depending on $\lambda$. For convenience we choose to take a multiple scales approach with scales $t$ and $T=\epsilon^{-\lambda}t$ in \eqref{eq:oneD_slowosc_innerscaled} to find

\begin{equation}\label{eq:oneD_slowosc_innergeneral}
\begin{aligned}
\epsilon^{\alpha-\lambda} y_T+\epsilon^{\alpha}y_t=& -\epsilon^{\beta}m(t)+\epsilon^{\alpha}2|y|-\epsilon^{2\alpha}y|y|+A\sin(T),\\
m_t(t)=&-\epsilon^{1-\beta}.
\end{aligned}
\end{equation}

To determine the correct scalings in \eqref{eq:oneD_slowosc_general_scaling}, we balance the leading order terms on both sides of \eqref{eq:oneD_slowosc_innergeneral} $\epsilon^{\alpha-\lambda}y_T$ and $A\sin(T)$, which gives us that $\alpha=\lambda$. This suggests the oscillatory term will persist in the inner asymptotic expansion of \eqref{eq:oneD_canonical} regardless of choice in $\lambda$.

We now consider the same scales $t$ and $T=\epsilon^{-\lambda}t$ on the canonical system \eqref{eq:oneD_canonical} 

\begin{equation}\label{eq:oneD_slowosc_general_outermulti}
\begin{aligned}
x_T+\epsilon^{\lambda}x_t =& -\epsilon^{\lambda+\beta}m(t)+\epsilon^{\lambda}2|x|-\epsilon^{\lambda}x|x|+\epsilon^{\lambda}A\sin(T),\\
m_t(t) =&-\epsilon^{1-\beta}.
\end{aligned}
\end{equation}

Where we use the expansion 

\begin{equation*}
x(t,T) = \epsilon^{\lambda}y_0(t,T) +\ldots 
\end{equation*}

that has the next terms of this expansion depending on whether $\lambda\le1$ or if $\lambda> 1$. We consider these ranges as Case I and Case II respectively.

\subsection{Case I: $\lambda \le 1$}
\label{subsec:oneD_slowosc_caseI}

We call this the mixed effects case due to both slow variation and oscillatory forcing causing noticeable effects on the solution for this range of $\lambda$. We consider the expansion

\begin{equation}\label{eq:oneD_slowosc_caseI_expansion}
x(t,T)\sim \epsilon^{\lambda} y_0(t,T)+\epsilon^q y_1(t,T)+\ldots
\end{equation}

with $q>\lambda$ to be consistent with our findings thus far. Substituting \eqref{eq:oneD_slowosc_caseI_expansion} into \eqref{eq:oneD_slowosc_general_outermulti} gives

\begin{equation*}
\begin{aligned}
{y_0}_T+\epsilon^{\lambda}{y_0}_t+\epsilon^{q-\lambda} {y_1}_T+\epsilon^{q} {y_1}_t+\ldots={} & -\epsilon^{\beta}m(t)+\epsilon^\lambda 2|y_0 +\epsilon^{q-\lambda} y_1+\ldots|+  A\sin(T)  \\
& + \epsilon^{2\lambda}( y_0 +\epsilon^{q-\lambda} y_1+\ldots)|y_0 +\epsilon^{q-\lambda} y_1+\ldots |
\end{aligned}
\end{equation*}

Separation by distinct orders of $\epsilon$ then gives the following equations at each order

\begin{align} \label{eq:oneD_slowosc_caseI_O1}
O(1):\, & {y_0}_T = A\sin(T),\\ \label{eq:oneD_slowosc_caseI_O2}
O(\epsilon^\lambda): \, & \epsilon^{q-2\lambda}{y_1}_T+{y_0}_t=-\epsilon^{\beta-\lambda}m(s)+2|y_0|.
\end{align}

We learn from \eqref{eq:oneD_slowosc_caseI_O2} that the appropriate next term in the expansion \eqref{eq:oneD_slowosc_caseI_expansion} is with $q= 2\lambda$, which implies that $\lambda> \frac{1}{2}$ for an expansion to be found without including the quadratic terms, as with the quadratic terms the equations become inseparable and unsolvable. We also have the choice between $\beta=\lambda$ or $\beta=1$ and each has a particular appeal. With $\beta=\lambda$, the form of \eqref{eq:oneD_slowosc_caseI_O2} is simple, but small coefficients appear further in the analysis. We choose to allow $\beta=1$ for convenience and track the small coefficient on $m(t)$ now in exchange for having a simple equation for $m(t)$. Using \eqref{eq:oneD_slowosc_caseI_O1} gives the appropriate separation in slow and fast scales, $y_0(t,T)=v_0(t)-A\cos(T)$. We then apply the Fredholm alternative \eqref{eq:Fredholm} to \eqref{eq:oneD_slowosc_caseI_O2} we find a similar equation to the integral \eqref{eq:oneD_osc_integral} in \autoref{sec:oneD_highfreqosc} with

\begin{equation}\label{eq:oneD_slowosc_caseIintegral}
{v_0}_t = -\epsilon^{1-\lambda}m(t)+\frac{1}{\pi}\int_0^{2\pi} |v_0(t)-A\cos(T)|\,dT.
\end{equation}

The approach developed in \autoref{sec:oneD_highfreqosc} is applied here to \eqref{eq:oneD_slowosc_caseIintegral}, where we separate the difficulty of the integral based on the relative size of $v_0(t)$. We have the following situations, Sub-Case I: $v_0(t)\le -|A|$ and Sub-Case II: $|v_0(t)|<|A|$.

\subsubsection{Sub-Case I: $v_0(t) \le -|A|$} 
\label{subsubsec:oneD_slowosc_subcaseI}

Once more, we call this the entirely below axis sub-case; We don't expect that tipping to occur under these conditions since the solution is entirely negative and far from the axis most of the time. Under these conditions, \eqref{eq:oneD_slowosc_caseIintegral} gives the simple inner equation

\begin{equation}\label{eq:oneD_slowosc_innersubcaseI}
{v_0}_t= -\epsilon^{1-\lambda}m(t)-2v_0.
\end{equation}

Solving \eqref{eq:oneD_slowosc_innersubcaseI} can be done under our assumptions much like in \autoref{subsec:oneD_osc_CaseI} but instead we focus on how the solution evolves via the equilibrium. This choice results in finding the effective parameter range for $\mu$ which distinguishes these sub-cases, which helps to determine when the solution will enter the region we do expect tipping to occur. Since $m(t)$ is allowed to vary, this must be thought of more as a pseudo-equilibrium and we are only interested in when the pseudo-equilibrium violates the assumptions of this case. Finding the pseudo-equilibrium of \eqref{eq:oneD_slowosc_innersubcaseI} gives

\begin{equation*}
v_0=-\epsilon^{1-\lambda}\frac{m(t)}{2}.
\end{equation*}

Using the condition $v_0(t)\le -|A|$ gives that $m(t)\ge \epsilon^{\lambda-1}2|A|$ for Sub-Case I to be effective. Writing this result in original coordinates gives us the parameter range for Sub-Case I

\begin{equation}\label{eq:oneD_slowosc_subcaseboundary}
\mu(t)\ge \frac{2 |A|}{\Omega},
\end{equation}

which agrees with the range from \eqref{eq:oneD_osc_caseboundary} in \autoref{sec:oneD_highfreqosc}. Following the pseudo-equilibrium to the boundary \eqref{eq:oneD_slowosc_subcaseboundary}, we eventually reach Sub-Case II where we see the oscillations crossing the axis.

\subsubsection{Sub-Case II: $|v_0(t)|< |A|$}
\label{subsubsec:oneD_slowosc_subcaseII}

Again, we call this the crossing sub-case; here the behavior of the solution depends strongly on the sign of the solution similarly to \autoref{sec:oneD_highfreqosc}, but we seek the relationship between slow variation and oscillatory forcing. As the pseudo-equilibrium get closer to the axis, the solution spends more time above the axis and more complicated contributions from the flipping signs appears. With this increasingly erratic behavior, we expect tipping to happen under these conditions.

The methodology of solving this integral \eqref{eq:oneD_slowosc_caseII_integral} holds identically to that of \autoref{subsec:oneD_osc_CaseII} and thus we evaluate by separating the sign of the integrand and then approximate with a quadratic Taylor expansion to find

\begin{equation}\label{eq:oneD_slowosc_subcaseIItaylor}
{v_0}_t = -\epsilon^{1-\lambda}m(t) + \frac{4|A|}{\pi} + \frac{2}{\pi |A|}v_0^2.
\end{equation}

But as \eqref{eq:oneD_slowosc_subcaseIItaylor} is in terms of 'slow' time, it restricts any analytical approaches to the effects of the varying parameter. Instead we switch the differentiation onto the parameter $m$ with

\begin{equation}\label{eq:oneD_slowosc_subcaseIItaylorm}
{v_0}_m = \epsilon^{1-\lambda}m - \frac{4|A|}{\pi} - \frac{2}{\pi |A|}v_0^2.
\end{equation}

It is here we take advantage of the form of \eqref{eq:oneD_slowosc_subcaseIItaylorm} with the result from Zhu \& Kuske \eqref{eq:intro_Zhuresult} to solve, this results in

\begin{equation*}
v_0(m)\sim \epsilon^{(1-\lambda)/3}\left( \frac{\pi |A|}{2} \right)^{2/3}\frac{Ai'\left(\epsilon^{2(\lambda-1)/3}\left(\frac{2}{\pi |A|}\right)^{1/3}(\epsilon^{1-\lambda}m-\frac{4|A|}{\pi})\right)}{Ai\left(\epsilon^{2(\lambda-1)/3}\left(\frac{2}{\pi |A|}\right)^{1/3}(\epsilon^{1-\lambda}m-\frac{4|A|}{\pi})\right)}.
\end{equation*}

With the solution to \eqref{eq:oneD_slowosc_caseI_expansion} we rewrite back into the original coordinates

\begin{equation}\label{eq:oneD_slowosc_caseIsoln}
\begin{aligned}
y_0(t,T)\sim& \epsilon^{(1-\lambda)/3}\left( \frac{\pi |A|}{2} \right)^{2/3}\frac{Ai'\left(\epsilon^{2(\lambda-1)/3}\left(\frac{2}{\pi |A|}\right)^{1/3}(\epsilon^{1-\lambda}m(t)-\frac{4|A|}{\pi})\right)}{Ai\left(\epsilon^{2(\lambda-1)/3}\left(\frac{2}{\pi |A|}\right)^{1/3}(m(t)-\frac{4|A|}{\pi})\right)}-\epsilon^\lambda A\cos(T)+\ldots,\\
x(t) \sim& \epsilon^{(2\lambda-1)/3}\left(\frac{\pi |A|}{2}\right)^{2/3}\frac{Ai'\left(\left(\frac{\Omega}{\epsilon^2}\right)^{1/3}\left(\frac{2}{\pi |A|}\right)^{1/3}(\mu(t)-\frac{4|A|}{\pi \Omega})\right)}{Ai\left(\left(\frac{\Omega}{\epsilon^2}\right)^{1/3}\left(\frac{2}{\pi |A|}\right)^{1/3}(\mu(t)-\frac{4|A|}{\pi \Omega})\right)}-\epsilon^\lambda A\cos(\Omega t) +\ldots.
\end{aligned}
\end{equation}

With the inner solution \eqref{eq:oneD_slowosc_caseIsoln} we search for the singularity of this solution in order to identify tipping. Recall from \eqref{eq:intro_Zhuresult} that the singularity relates to the first root of the Airy equation, which is when the argument is $-2.33811\ldots$. We find the tipping to be

\begin{equation}\label{eq:oneD_slowosc_uglymu}
\mu_{\text{mixed}}=\left(\frac{\epsilon^2}{\Omega}\right)^{1/3}\left(\frac{\pi |A|}{2}\right)^{1/3}(-2.33811\ldots)+\frac{4|A|}{\pi \Omega}.
\end{equation}

This value of $\mu$ that causes this singularity in turn is our tipping point, where we rewrite \eqref{eq:oneD_slowosc_uglymu} to show the interaction between slow variation on the parameter and the oscillatory forcing

\begin{equation}\label{eq:oneD_slowosc_caseItipping}
\mu_{\text{mixed}} = \left(\frac{1}{\epsilon\Omega}\right)^{1/3}\left(\frac{\pi |A|}{2}\right)^{1/3} \mu_{\text{smooth}}+\mu_{\text{osc}},
\end{equation}

with $\mu_{\text{smooth}}=\epsilon\left(-2.33811\ldots\right)$, similarly to the smooth problem from Zhu \& Kuske \cite{zhu2015tipping}, and $\mu_{\text{osc}}$ from \eqref{eq:oneD_osc_bif} respectively.

The resulting tipping approximation \eqref{eq:oneD_slowosc_caseItipping} indicates that the size of the amplitude $A$ will determine whether the tipping occurs early or late in reference to the bifurcation, naturally we see a larger cause more contribution from the oscillations and hence an earlier tipping. On the other hand, larger values in $\epsilon$ will cause this tipping to occur later. So these effects have opposite pulls on the tipping and can effectively cancel one another out under the proper conditions. It would even be possible to break the hysteresis cycle by eliminating the region of bi-stability of this problem with sufficiently large amplitude and small $\epsilon$. This tipping holds for any $\lambda\in (\frac{1}{2},1]$ and we see different behavior for larger $\lambda$.

\subsection{Case II: $\lambda>1$}
\label{subsec:oneD_slowosc_caseII}

We call this the slowly dominate case as this is when we begin to see the oscillations contribute less than the slow variation. For this range of $\lambda$ the scaling for $\mu$ is simple, $\mu=\epsilon m$ and thus we expect to see integer powers along with $\lambda$ powers so we choose the expansion

\begin{equation}\label{eq:oneD_slowosc_caseII_expansion}
x(t,T) \sim \epsilon^\lambda y_0(t,T)+\epsilon y_1(t,T)+\epsilon^q y_2(t,T)+\ldots
\end{equation}


Substituting \eqref{eq:oneD_slowosc_caseII_expansion} into \eqref{eq:oneD_slowosc_general_outermulti} gives

\begin{equation*}
\begin{aligned}
\epsilon {y_0}_T+\epsilon^{\lambda+1} {y_0}_t+\epsilon^\lambda {y_1}_T+\epsilon^{2\lambda}{y_1}_t+\epsilon^q {y_2}_T+\ldots=&-\epsilon^{\lambda+1}m(t)+\epsilon^{\lambda+1} 2|y_0+\epsilon^{\lambda-1} y_1 +\ldots|\\
&+ \epsilon^{\lambda+2}( y_0+\epsilon^{\lambda-1} y_1 +\ldots)| y_0+\epsilon^{\lambda-1} y_1 +\ldots|\\
&+\epsilon^\lambda A\sin(T) 
\end{aligned}
\end{equation*}

Where we separate out each order of $\epsilon$ to find the equations at each order

\begin{align} \label{eq:oneD_slowosc_caseII_O1}
O(\epsilon):\, &{y_0}_T=0,\\ \label{eq:oneD_slowosc_caseII_O2}
O(\epsilon^\lambda): \, & {y_1}_T = A\sin(T),\\ \label{eq:oneD_slowosc_caseII_O3}
O(\epsilon^{\lambda+1}):\, & \epsilon^{q-\lambda-1}{y_2}_T+ {y_0}_t = -m(t) +2|y_0+\epsilon^{\lambda-1}y_1|.
\end{align}

We learn in \eqref{eq:oneD_slowosc_caseII_O3} that $q=\lambda+1$ to keep terms from becoming trivial or unbalanced. Where \eqref{eq:oneD_slowosc_caseII_O1} tells us that the dominate behavior for this case is purely slow, $y_0=y_0(t)$. We find the oscillatory behavior in $y_1$ with \eqref{eq:oneD_slowosc_caseII_O2} which gives $y_1(t,T)=v_1(t)-A\cos(T)$. But as we have $y_1$ as a correction to $y_0$, we may absorb it's slow behavior into $y_0$. Thus we treat $y_0(t)=y_0(t)+\epsilon^{\lambda+1} v_1(t)\approx y_0(t)$. Applying Fredholm to \eqref{eq:oneD_slowosc_caseII_O3} gives 

\begin{equation}\label{eq:oneD_slowosc_caseII_integral}
\begin{aligned}
{y_0}_t=& -m(t)+\frac{1}{\pi}\int_0^{2\pi}|y_0(t)-\epsilon^{\lambda-1}A\cos(T)|\,dT.
\end{aligned}
\end{equation}

With $\lambda$ being slightly larger than 1, we still see nearly identical behavior in \eqref{eq:oneD_slowosc_caseII_integral} as that of what we explored in \autoref{subsec:oneD_slowosc_caseI} as long as the amplitude of oscillations here behave similarly to those of Case I, $\epsilon^{\lambda-1}A\sim O(1)$. To see this, we follow the same approach as to integrate \eqref{eq:oneD_slowosc_caseII_integral} with $T_1=\arccos(y_0/\epsilon^{\lambda-1}A)$ and $T_2=2\pi- \arccos(y_0/\epsilon^{\lambda-1}A)$ then apply the same quadratic Taylor approximation to find

\begin{equation}\label{eq:oneD_slowosc_caseII_taylor}
{y_0}_t=-m(t)+\epsilon^{\lambda-1}\frac{2|A|}{\pi}+\epsilon^{1-\lambda}\frac{2}{\pi |A|}y_0^2.
\end{equation}

We use the result from Zhu \& Kuske \eqref{eq:intro_Zhuresult} to find the tipping, which we then write into original coordinates

\begin{equation}
\begin{aligned}
m_{\text{mixed}}=&\epsilon^{(\lambda-1)/3}\left(\frac{\pi |A|}{2}\right)^{1/3}(-2.33811\ldots)+\epsilon^{\lambda-1}\frac{4|A|}{\pi},\\
\mu_{\text{mixed}}=&\left(\frac{1}{\epsilon\Omega}\right)^{1/3}\left(\frac{\pi |A|}{2}\right)^{1/3} \mu_{\text{smooth}}+\mu_{\text{osc}}.
\end{aligned}
\end{equation}

We conclude that there is a natural transition into Case II from Case I with almost the same behavior and identical tipping as in \eqref{eq:oneD_slowosc_caseItipping}. As $\lambda$ continues to grow, the oscillations in \eqref{eq:oneD_slowosc_caseII_integral} begin to die off. This allows us to say that the integral is approaching

\begin{equation}\label{eq:oneD_slowosc_caseII_inner}
{y_0}_t = -m(t) +2|y_0|.
\end{equation}

But \eqref{eq:oneD_slowosc_caseII_inner} has the same form as in \autoref{sec:oneD_slow} allowing us to use the results there to find the solution, where we put this back in terms of the original coordinates

\begin{equation}\label{eq:oneD_slowosc_caseIIsoln}
\begin{aligned}
y_0(t,T)\sim& C e^{-2m(t)}+\frac{m(t)}{2}-1/4 +\epsilon^\lambda A\cos(T),\\
x(t)\sim& \epsilon C e^{-2\mu(t)/\epsilon}+\frac{\mu(t)}{2} -\epsilon^\lambda A\cos(\Omega t)+O(\epsilon^{2\lambda}).
\end{aligned}
\end{equation}

Which also leads to the same tipping from the slow case with 

\begin{equation*}
\mu_{\text{slow}}=\frac{1}{2}\epsilon\log\epsilon.
\end{equation*}

Thus we find that in this case that even as $\lambda$ grows past 1, the same $\mu_{\text{mixed}}$ tipping as in \eqref{eq:oneD_slowosc_caseItipping} occurs from \autoref{subsec:oneD_slowosc_caseI}. But as $\lambda$ continues to grow, the oscillations continue to have less of an impact until the solution tips entirely like $\mu_{\text{slow}}$ as in \eqref{eq:oneD_slow_tipping} from \autoref{sec:oneD_slow}. For convenience, this is summarized in the following table.

\begin{table}[H]\label{table:oneD_tipping}
\centering
\begin{tabular}{|c|c|}
\hline 
 \multicolumn{2}{|c|}{One-Dimensional Tipping} \\ 
\hline
Slow: & $\mu_{\text{slow}}=\epsilon\ln(\epsilon)/2$ \\ 
\hline 
High Freq. Osc: & $\mu_{\text{osc}}=\frac{4|A|}{\pi \Omega}$\\ 
\hline 
Slowly Oscillatory $\lambda\le 1$: & $\mu_{\text{mixed}}=\epsilon^{(\lambda-1)/3}\left(\frac{\pi |A|}{2}\right)^{1/3} \mu_{smooth}+\mu_{osc}$ \\ 
\hline 
Slowly Oscillatory $\lambda\le 1$ and $\lambda\approx 1$: & $\mu_{\text{mixed}}=\epsilon^{(\lambda-1)/3}\left(\frac{\pi |A|}{2}\right)^{1/3} \mu_{smooth}+\mu_{osc}$ \\
\hline
Slowly Oscillatory $\lambda>1$: & $ \mu_{\text{slow}}=\epsilon\ln(\epsilon)/2$\\
\hline
\end{tabular} 
\caption{The tipping of the one-dimensional model for each mechanism and case.}
\end{table}

In figure~\ref{fig:oneD_slowosc_numerical_small}, we see an example of the numerical solution to the canonical system \eqref{eq:oneD_canonical} with slow variation and oscillatory forcing. This example has tipping occurring in Case I due to $\lambda\in (\frac{1}{2},1]$ allowing the slow variation and oscillatory forcing to produce a mixed effect on the tipping. Although we see noticeable contributions from the slow varying parameter the tipping still is occurring in the region near the oscillatory bifurcation. This tells us that for these choices in the values the strongest effect is the oscillatory forcing. It is now possible here find values of $\epsilon$, $A$ and $\lambda$ that will cause the non-smooth tipping to occur at the same place as the smooth bifurcation. This in theory, would eliminate the region of bi-stability and destroy the hysteresis curve entirely for this model.

\begin{figure}[H]
\centering
\begin{subfigure}{.5\textwidth}
  \centering
  \includegraphics[width=\linewidth]{oneD/slowosc_bif_diagram_small.jpg}
  \caption{}
\end{subfigure}%
\begin{subfigure}{.5\textwidth}
  \centering
  \includegraphics[width=\linewidth]{oneD/slowosc_bif_diagram_small_zoom.jpg}
  \caption{}
\end{subfigure}
\caption{Parameter values are $\epsilon=.05$, $\lambda=.8$ and $A=1$. On the left, this is the bifurcation diagram for the 1D system with the numerical solution to \eqref{eq:oneD_canonical} (black dotted line). The solid vertical lines dictate the region between Sub-Case I and Sub-Case II (green) as well as the tipping estimate (blue) respectively. The dotted vertical line is the numerical tipping, when $x>.2$. On the right, this is a zoom in.}
\label{fig:oneD_slowosc_numerical_small}
\end{figure}

In figure~\ref{fig:oneD_slowosc_numerical_medium}, we see an example of $\lambda$ falling into Case II but is close enough to 1 that we see mixed behavior in the tipping still. Here the slow variation is now dominant and the oscillations are only noticeable in the zoom in. 

\begin{figure}[H]
\centering
\begin{subfigure}{.5\textwidth}
  \centering
  \includegraphics[width=\linewidth]{oneD/slowosc_bif_diagram_medium.jpg}
  \caption{}
\end{subfigure}%
\begin{subfigure}{.5\textwidth}
  \centering
  \includegraphics[width=\linewidth]{oneD/slowosc_bif_diagram_medium_zoom.jpg}
  \caption{}
\end{subfigure}
\caption{Parameter values are $\epsilon=.05$, $\lambda=1.1$ and $A=1$. On the left, this is the bifurcation diagram for the 1D system with the numerical solution to \eqref{eq:oneD_canonical} (black dotted line). The solid vertical lines dictate the region between Sub-Case I and Sub-Case II (green) as well as the tipping estimate (blue) respectively. The dotted vertical line is the numerical tipping, when $x>.2$. On the right, this is a zoom in.}
\label{fig:oneD_slowosc_numerical_medium}
\end{figure}

In figure~\ref{fig:oneD_slowosc_numerical_large}, we see an example of $\lambda$ falling into Case II but is far enough from 1 that we see almost entirely slow behavior in the tipping. Even upon closer inspection there its hardly noticeable that oscillations are present in the model. 

\begin{figure}[H]
\centering
\begin{subfigure}{.5\textwidth}
  \centering
  \includegraphics[width=\linewidth]{oneD/slowosc_bif_diagram_large.jpg}
  \caption{}
\end{subfigure}%
\begin{subfigure}{.5\textwidth}
  \centering
  \includegraphics[width=\linewidth]{oneD/slowosc_bif_diagram_large_zoom.jpg}
  \caption{}
\end{subfigure}
\caption{Parameter values are $\epsilon=.05$, $\lambda=1.6$ and $A=1$. On the left, this is the bifurcation diagram for the 1D system with the numerical solution to \eqref{eq:oneD_canonical} (black dotted line). The solid vertical lines dictate the region between Sub-Case I and Sub-Case II (green) as well as the tipping estimate (blue) respectively. The dotted vertical line is the numerical tipping, when $x>.2$. On the right, this is a zoom in.}
\label{fig:oneD_slowosc_numerical_large}
\end{figure}

In figure~\ref{fig:oneD_slowosc_lambdacomp} we compare the tipping between Case I and Case II with the numerical tipping. For smaller $\lambda$, the frequency $\Omega$ gets smaller and the Case I tipping becomes more predominant. But for the analysis performed in this section, $\Omega\gg 1$ and for $\lambda\le\frac{1}{2}$ we have $\Omega\sim O(1)$. We will not consider low frequency corresponding to $\lambda\le\frac{1}{2}$ in this section. The larger $\lambda$ becomes, the less effect we see from the oscillatory forcing until it is negligible for some $\lambda>1$. Further intuition behind this is in the asymptotic solution for each case, \eqref{eq:oneD_slowosc_outersoln}, \eqref{eq:oneD_slowosc_caseIsoln}, and \eqref{eq:oneD_slowosc_caseIIsoln}. The oscillatory component of the term has a $\epsilon^\lambda$ coefficient and will shrink the effects as $\lambda$ grows.

\begin{figure}[H]
\centering
\includegraphics[width=0.7\textwidth]{oneD/slowosc_lambdacomp.jpg}
\caption{An example of numerical tipping (red stars) as the numerical solution to \eqref{eq:oneD_canonical} passes $x=.2$ for the last time. Parameter values are $\epsilon=.01$ and $A=1$. The lines are the Case I tipping estimate (black solid line) and the Case II tipping estimate (blue dotted line).}
\label{fig:oneD_slowosc_lambdacomp}
\end{figure} 

The performance of our estimates are seen in figure~\ref{fig:oneD_slowosc_epscomp}. For Case I tipping, the range of appropriate $\epsilon$ is highly dependent on the choice in $\lambda$. Often, the range is very small to get accurate estimates. Once this range is left, there are interesting phase effects for the tipping which causes oscillations in the numeric tipping points. For Case II tipping, we see very accurate estimates over a larger window, similarly to \autoref{sec:oneD_slow}


\begin{figure}[H]
\centering
\begin{subfigure}{.5\textwidth}
  \centering
  \includegraphics[width=\linewidth]{oneD/slowosc_epscomp_case2.jpg}
  \caption{$\lambda=.8$}
\end{subfigure}%
\begin{subfigure}{.5\textwidth}
  \centering
  \includegraphics[width=\linewidth]{oneD/slowosc_epscomp_case3.jpg}
  \caption{$\lambda=1.3$}
\end{subfigure}
\caption{The numerical tipping (red stars) follows the appropriate case depending on $\lambda$ for $\epsilon=0.005$. The Case I tipping estimate (black solid line) and the Case II tipping estimate (blue dotted line) are shown.}
\label{fig:oneD_slowosc_epscomp}
\end{figure}

\subsection{Stability}

Similarly to \autoref{sec:oneD_highfreqosc}, there are two ranges for $\lambda$ of interest that will govern the stability of solutions found in our analysis, $\lambda\le 1$ and $\lambda>1$.

\subsubsection{Case I: $\lambda\le 1$}

Recall for this case, we must have $\lambda\in (\frac{1}{2},1]$ and for this range, we found the inner equation

\begin{equation}\label{eq:oneD_slowosc_innerintegral}
{v_0}_t= -\epsilon^{1-\lambda}m(t)+\frac{1}{\pi}\int_0^{2\pi}| v_0(t)- A\cos(T)|\,dT=f(t,v_0).
\end{equation}

Like in our analysis, we consider two regions in the solution $v_0(t)$ for this integral, Sub-Case I: $v_0(t)\le - |A|$ and Sub-Case II: $|v_0(t)|\le |A|$ where these sub-cases deal with the respective size of $v_0(t)$ to the oscillations and their resulting stability.

\subsubsection*{Sub-Case I: $v_0(t)\le - |A|$}

Recall from the analysis that the equation \eqref{eq:oneD_slowosc_innerintegral} simplifies in this region of $v_0(t)$ and has the following inner equation and pseudo-equilibrium

\begin{equation}\label{eq:oneD_slowosc_stabilitysubcaseI}
{v_0}_t= -\epsilon^{1-\lambda}m(t) -2v_0=f(t,v_0), \quad z^0(t)=-\epsilon^{1-\lambda}\frac{m(t)}{2}.
\end{equation}

As we saw in \autoref{sec:oneD_slow}, special treatment of the pseudo-equilibrium stability analysis is needed with linear perturbations $v_0(t)=z^0(t)+u(t)$ where $\lVert u(t)\rVert \ll 1$ and $z^0_t = -\epsilon^{1-\lambda}\frac{m_t}{2}=\frac{\epsilon^{1-\lambda}}{2}$. The resulting Taylor expansion is thus

\begin{equation*}
\begin{aligned}
{v_0}_t =& f(t,z^0)+f_{v_0}(t,v_0)(v_0(t)-z^0(t))+O(\lVert v_0(t)-z^0(t) \rVert^2),\\
u_t+z^0_t=&-2u(t)+O(\lVert u(t)\rVert^2),\\
u_t =&-\frac{\epsilon^{1-\lambda}}{2}-2u(t).
\end{aligned}
\end{equation*}

Which leads to the conclusion that equation \eqref{eq:oneD_slowosc_stabilitysubcaseI} causes perturbations to decay exponentially to a nearby equilibrium. Hence we find the pseudo-equilibrium to be hyperbolic and asymptotically stable.

\subsubsection*{Sub-Case II: $v_0(t)\le  |A|$}

With the Taylor approximation from the analysis \eqref{eq:oneD_slowosc_subcaseIItaylor}, we have the following inner equation and pseudo-equilibrium

\begin{equation}\label{eq:oneD_slowosc_stabilitysubcaseII}
{v_0}_t= -\epsilon^{1-\lambda}m(t) +\frac{4|A|}{\pi}+\frac{2}{\pi |A|}v_0^2=f(t,v_0),\quad z^0(t)=-C \sqrt{\epsilon^{1-\lambda}m(t)-\frac{4|A|}{\pi}}
\end{equation}

We consider simple linear perturbations to this pseudo-equilibrium \eqref{eq:oneD_slowosc_stabilitysubcaseII}, $v_0(t)=z^0(t)+u(t)$ with $\lVert u(t) \rVert \ll 1$. Treating the pseudo-equilibrium carefully, we find that the slowly varying component of the equilibrium contributes to the derivative. Thus we have

\begin{equation}
\begin{aligned}
{v_0}_t =& z^0_t(t) +u_t,\\
z^0_t(t) = & \begin{cases}
\frac{\epsilon^{1-\lambda}}{2C\sqrt{\epsilon^{1-\lambda}m(t)-\frac{4|A|}{\pi}}} & \epsilon^{1-\lambda}m(t)> \frac{4|A|}{\pi},\\
0 & \epsilon^{1-\lambda}m(t) =\frac{4|A|}{\pi}.
\end{cases}
\end{aligned}
\end{equation}

Now applying a Taylor expansion, we find the following behavior of perturbations

\begin{equation}\label{eq:oneD_slowosc_stabilitysubcaseIIeq}
\begin{aligned}
{v_0}_t =& f(t,z^0)+f_{v_0}(t,z^0)(v_0-z^0(t))+O(\lVert v_0(t)-z^0(t) \rVert),\\
u_t =&\begin{cases}
-\frac{\epsilon^{1-\lambda}}{2C\sqrt{\epsilon^{1-\lambda}m(t)-\frac{4|A|}{\pi}}}-2\sqrt{\epsilon^{1-\lambda}m(t)-\frac{4|A|}{\pi}} u & \epsilon^{1-\lambda}m(t)>\frac{4|A|}{\pi},\\
0 & \epsilon^{1-\lambda}m(t)=\frac{4|A|}{\pi}.
\end{cases}
\end{aligned}
\end{equation}

From \eqref{eq:oneD_slowosc_stabilitysubcaseIIeq}, we find that the perturbations decay to a fixed negative quantity. This indicates, much like \autoref{sec:oneD_slow}, that there is an attracting equilibrium below the pseudo-equilibrium. The negative sign describes exponential decay and hence this equilibrium is hyperbolic and asymptotically stable for $\epsilon^{1-\lambda}m(t)>\frac{4|A|}{\pi}$ or $\mu(t)>\frac{4|A|}{\pi \Omega}$. But for $\epsilon^{1-\lambda}m(t)=\frac{4|A|}{\pi}$ or $\mu(t) =\frac{4|A|}{\pi \Omega}$, the stability of \eqref{eq:oneD_slowosc_stabilitysubcaseIIeq} suddenly becomes hyperbolic. This tells us that we lose stability at the oscillatory bifurcation but tipping will occur afterwards, which agrees with the conclusion in the tipping approximation from \eqref{eq:oneD_slowosc_caseItipping}.

\subsubsection{Case II: $\lambda>1$}

From the analysis, we discovered that for as long as $\epsilon^{\lambda-1}A\sim O(1)$, then we have the no different behavior in the tipping. With the Taylor approximation from the analysis \eqref{eq:oneD_slowosc_caseII_integral}, the inner equation and pseudo-equilibrium is

\begin{equation}\label{eq:oneD_slowosc_stabilitycaseII}
\begin{aligned}
y_0=&-m(t) +\epsilon^{\lambda-1}\frac{2|A|}{\pi}+\epsilon^{1-\lambda}\frac{2}{\pi |A|}y_0^2=f(t,y),\\
z^0(t)=&-\epsilon^{\lambda-1}C\sqrt{m(t)-\epsilon^{\lambda-1}\frac{4|A|}{\pi}}.
\end{aligned}
\end{equation}

Similarly to Case I, we consider simple linear perturbations to this pseudo-equilibrium \eqref{eq:oneD_slowosc_stabilitycaseII}, $y_0(t)=z^0(t)+u(t)$ with $\lVert u(t) \rVert \ll 1$. Treating the pseudo-equilibrium carefully, we find that the slowly varying component of the equilibrium contributes to the derivative. Thus we have

\begin{equation}
\begin{aligned}
{y_0}_t =& z^0_t(t) +u_t,\\
z^0_t(t) = & \begin{cases}
\frac{\epsilon^{\lambda-1}}{2C\sqrt{m(t)-\epsilon^{\lambda-1}\frac{4|A|}{\pi}}} & m(t)> \epsilon^{\lambda-1}\frac{4|A|}{\pi},\\
0 & m(t) =\epsilon^{\lambda-1}\frac{4|A|}{\pi}.
\end{cases}
\end{aligned}
\end{equation}

Now applying a Taylor expansion, we find the following behavior of perturbations

\begin{equation}\label{eq:oneD_slowosc_stabilitycaseIIeq}
\begin{aligned}
{y_0}_t =& f(t,z^0)+f_{y_0}(t,z^0)(y_0-z^0(t))+O(\lVert y_0-z^0(t) \rVert^2),\\
u_t =&\begin{cases}
-\frac{\epsilon^{\lambda-1}}{2C\sqrt{m(t)-\epsilon^{\lambda-1}\frac{4|A|}{\pi}}}-2\sqrt{m(t)-\epsilon^{\lambda-1}\frac{4|A|}{\pi}} u & m(t)>\epsilon^{\lambda-1}\frac{4|A|}{\pi},\\
0 & m(t)=\epsilon^{\lambda-1}\frac{4|A|}{\pi}.
\end{cases}
\end{aligned}
\end{equation}

But the conclusions from Case I still apply to \eqref{eq:oneD_slowosc_stabilitycaseIIeq} and thus we still have stability up until $\mu(t)=\frac{4|A|}{\pi\Omega}$ and expect to see tipping occurring after the oscillatory bifurcation which is consistent with our tipping approximation for this case.

On the other hand, for large $\lambda$, the  integral \eqref{eq:oneD_slowosc_caseII_integral} approaches

\begin{equation}\label{eq:oneD_slowosc_stabilitycaseII}
{y_0}_t=-m(t)+2|y_0|.
\end{equation}
But this is the same type of behavior from \autoref{sec:oneD_slow}, where we found that for $m(t)\ge 0$ our pseudo-equilibrium was stable and for $m(t)<0$ that searching for the pseudo-equilibrium causes a contradiction. Thus, we conclude that the tipping will occur in the region of $m(t)<0$ which agrees with \eqref{eq:oneD_slow_tipping}.
