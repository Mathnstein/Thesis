\section*{High Frequency Oscillatory Forcing}

Here we continue the analysis to explicitly find the solution to the outer equation for the purely oscillatory model. Recall we found $x_1 = v_1(t)-A\cos(T)$, we then apply the Fredholm alternative \eqref{eq:Fredholm} to the $O(\Omega^{-2})$ equation in \eqref{eq:oneD_osc_outerO3} to get

\begin{equation}\label{eq:oneD_app_osc}
\begin{aligned}
0=&\frac{1}{2\pi}\int_0^{2\pi}-{x_1}_t-2x_1+2x_0x_1\, dT,\\
{v_1}_t=& -2v_1+2(1-\sqrt{1+\mu})v_1,\\
{v_1}_t=& -2\sqrt{1+\mu}v_1
\end{aligned}
\end{equation}

We search for the equilibrium to find stable behavior on this order but since \eqref{eq:oneD_app_osc} has a very simple form, the equilibrium is $v_1(t)\equiv 0$ and thus we find the correction term to only have oscillatory behavior, $x_1=-A\cos(T)$.

\section*{Slow Variation and Oscillatory Forcing}

Here we continue to find the terms of the outer solution for the slow varying and oscillatory forcing model. We have thus far found $x_0=x_0(\tau)$ and we have equations at $O(\epsilon^\lambda)$ and $O(\epsilon^{2\lambda})$ that give information about $x_0$ and $x_1$ respectively. From the $O(\epsilon^\lambda)$ equation \eqref{eq:oneD_slowosc_outerO2}, we apply the Fredholm alternative \eqref{eq:Fredholm} to find

\begin{equation}\label{eq:oneD_app_slowosc1}
\begin{aligned}
0 = & \frac{1}{2\pi}\int_0^{2\pi} -\mu(\tau) -2x_0(\tau)+x_0(\tau)^2+A\sin(T)\,dT,\\
0=&-\mu(\tau)-2x_0(\tau)+x_0(\tau)^2,\\
x_0(\tau) =& 1-\sqrt{1+\mu(\tau)},\\
{x_1}_T = & A\sin(T)
\end{aligned}
\end{equation}

From \eqref{eq:oneD_app_slowosc1} we find that $x_1=v_1(\tau)-A\cos(T)$, which gives us access to solving the next order equation. Thus we now do the same for the $O(\epsilon^{2\lambda})$ equation \eqref{eq:oneD_slowosc_outerO3} to find

\begin{equation}
\begin{aligned}
0=&\frac{1}{2\pi}\int_0^{2\pi}-\epsilon^{1-\lambda}{x_0}_\tau -2x_1+2x_0x_1\,dT,\\
\epsilon^{1-\lambda}{x_0}_\tau=& -2v_1+2(1-\sqrt{1+\mu(\tau)})v_1,\\
v_1(\tau) =& -\epsilon^{1-\lambda}\frac{{x_0}_\tau}{2\sqrt{1+\mu(\tau)}}.
\end{aligned}
\end{equation}

Where we recall that $\mu_\tau=-1$ and that ${x_0}_\tau = -\frac{{\mu}_\tau}{2\sqrt{1+\mu(\tau)}}=\frac{1}{2\sqrt{1+\mu(\tau)}}$ thus we have the form of the next order term in the expansion as

\begin{equation}
x_1(\tau,T) = -\epsilon^{1-\lambda}\frac{1}{4(1+\mu(\tau))}-A\cos(T).
\end{equation}