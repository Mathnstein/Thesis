%%
%% This is file `ubcsample.tex',
%% generated with the docstrip utility.
%
% The original source files were:
%
% ubcthesis.dtx  (with options: `ubcsampletex')
%% 
%% This file was generated from the ubcthesis package.
%% --------------------------------------------------------------
%% 
%% Copyright (C) 2001
%% Michael McNeil Forbes
%% mforbes@alum.mit.edu
%% 
%% This file may be distributed and/or modified under the
%% conditions of the LaTeX Project Public License, either version 1.2
%% of this license or (at your option) any later version.
%% The latest version of this license is in
%%    http://www.latex-project.org/lppl.txt
%% and version 1.2 or later is part of all distributions of LaTeX
%% version 1999/12/01 or later.
%% 
%% This program is distributed in the hope that it will be useful,
%% but WITHOUT ANY WARRANTY; without even the implied warranty of
%% MERCHANTABILITY or FITNESS FOR A PARTICULAR PURPOSE.  See the
%% LaTeX Project Public License for more details.
%% 
%% This program consists of the files ubcthesis.dtx, ubcthesis.ins, and
%% the sample figures fig.eps and fig.fig.
%% 
%% This file may be modified and used as a base for your thesis without
%% including the licence agreement as long as the content (i.e. textual
%% body) of the file is completely rewritten. You must, however, change
%% the name of the file.
%% 
%% This file may only be distributed together with a copy of this
%% program. You may, however, distribute this program without generated
%% files such as this one.
%% 

% This Sample thesis requires \LaTeX2e
\NeedsTeXFormat{LaTeX2e}[1995/12/01]
\ProvidesFile{ubcsample.tex}[2015/05/31 v1.72 ^^J
 University of British Columbia Sample Thesis]
% This is the \documentclass[]{} command.  The manditory argument
% specifies the "flavour" of thesis (ubcthesis for UBC).  The
% optional arguments (in []) specify options that affect how the
% thesis is displayed.  Please see the ubcthesis documentation for
% details about the options.
\documentclass[msc,oneside]{ubcthesis}
%
% To compile this sample thesis, issue the following commands:
% latex ubcsample
% bibtex ubcsample
% latex ubcsample
% latex ubcsample
% latex ubcsample
%
% To view use xdvi (on unix systems):
% xdvi ubcsample.dvi
%
% To make a postscript file, use dvips:
% dvips -o ubcsample.ps ubcsample.dvi
%
% To view the postscript file, use ghostview or gv (on unix systems):
% gv ubcsample.ps
%
%************************************************
% Optional packages.
%
% The use of these packages is optional, but they provide various
% tools for more flexible formating.  The sample thesis uses these,
% but if you remove the example code, you should be able to exclude
% these packages.  Only standard packages have been described here;
% they should be installed with any complete LaTeX installation, but
% if not, you can find them at the Comprehensive TeX Archive Network
% (CTAN): http://www.ctan.org/
%

%******** afterpage ***************************
% This package allows you to issue commands at the end of the current
% page.  A good use for this is to use the command
% \afterpage{\clearpage} right after a figure.  This will cause the
% figure to be inserted on the page following the current one (or on
% the current page if it will fit) but will not break the page in the
% middle.
\usepackage{afterpage}

%******** float *********************************
% This package allows you to customize the style of
% "floats"---floating objects such as figures and tables.  In
% addition, it allows you to define additional floating objects which
% may be included in a list similar to that produces by \listoftables
% and \listoffigures.  Common uses include introducing floats for
% programs and other code bits in Compute Science and Chemical Schema.
\usepackage{float}

%******** tocloft *******************************
% This package allows you to customize and define custom lists such
% as a list of programs or Chemical Scheme.  Note: if you use the
% subfigure package, you must specify that you do as an option here.
% The title option uses the default formatting.  We do not use this
% here as the default formatting is acceptable.  Use the float
% package instead unless you need the extra formatting control
% provided by tocloft.
%\usepackage[subfigure, titles]{tocloft}

%******** alltt *********************************
% The alltt package allows you to include files and have them
% formatted in a verbatim fashion.  This is useful for including
% source code from an additional file.
%\usepackage{alltt}

%******** listings ******************************
% The listings package may be used to include chunks of source code
% and has facilities for pretty-printing many languages.
%\usepackage{listings}

%******** longtable *****************************
% The longtable package allows you to define tables that span
% multiple pages.
\usepackage{longtable}

%******** graphics and graphicx *****************
% This allows you to include encapsulated postscript files.  If you
% don't have this, comment the \includegraphics{} line following the
% comment "%includegraphics" later in this file.
\usepackage{graphicx}

%******** subfigure *****************************
% The subfigure package allows you to include multiple figures and
% captions within a single figure environment.
%\usepackage{subfigure}

%******** here **********************************
% The here package gives you more control over the placement of
% figures and tables.  In particular, you can specify the placement
% "H" which means "Put the figure here" rather than [h] which means
% "I would suggest that you put the figure here if you think it looks
% good."
\usepackage{here}

%******** pdflscape ********************************
% This allows you to include landscape layout pages by using the
% |landscape| environment.  The use of |pdflscape| is preferred over
% the standard |lscape| package because it automatically rotates the
% page in the pdf file for easier reading.  (Thanks to Joseph Shea
% for pointing this out.)
\usepackage{pdflscape}

%******** natbib ********************************
% This is a very nice package for bibliographies.  It includes options
% for sorting and compressing bibliographic entries.
%\usepackage[numbers,sort&compress]{natbib}

%******** psfrag ******************************
% This allows you to replace text in postscript pictures with formated
% latex text.  This allows you to use math in graph labels
% etc. Uncomment the psfrag lines following the "%psfrag" comment
% later in this file if you don't have this package.  The replacements
% will only be visible in the final postscript file: they will be
% listed in the .dvi file but not performed.
\usepackage{psfrag}

%******** hyperref *****************************
% Please read the manual:
% http://www.tug.org/applications/hyperref/manual.html
%
% This adds hyperlinks to your document: with the right viewers (later
% versions of xdvi, acrobat with pdftex, latex2html etc.) this will
% make your equation, figure, citation references etc. hyperlinks so
% that you can click on them.  Also, your table of contents will be
% able to take you to the appropriate sections.  In the viewers that
% support this, the links often appear with an underscore.  This
% underscore will not appear in printed versions.
%
% Note: if you do not use the hypertex option, then the dvips driver
% may be loaded by default.  This will cause the entries in the list
% of figures and list of tables to be on a single line because dvips
% does not deal with hyperlinks on broken lines properly.
%
% NOTE: HYPERREF is sensitive to the ORDER in which it is LOADED.
% For example, it must be loaded AFTER natbib but BEFORE newly
% defined float environments.  See the README file with the hyperref
% for some help with this.  If you have some very obscure errors, try
% first disabling hyperref.  If that fixes the problem, try various
% orderings.
%
% Note also that there is a bug with versions before 2003/11/30
% v6.74m that cause the float package to not function correctly.
% Please ensure you have a current version of this package.  A
% warning will be issued if you leave the date below but do not have
% a current version installed.
%
% Some notes on options: depending on how you build your files, you
% may need to choose the appropriate option (such as [pdftex]) for the
% backend driver (see the hyperref manual for a complete list).  Also,
% the default here is to make links from the page numbers in the table
% of contents and lists of figures etc.  There are other options:
% excluding the [linktocpage] option will make the entire text a
% hyperref, but for some backends will prevent the text from wrapping
% which can look terrible.  There is a [breaklinks=true] option that
% will be set if the backend supports (dvipdfm for example supports
% it but does not work with psfrag.)
%
% Finally, there are many options for choosing the colours of the
% links.  These will be included by default in future versions but
% you should probably consider changing some now for the electronic
% version of your thesis.
\usepackage[unicode=true,
  linktocpage,
  linkbordercolor={0.5 0.5 1},
  citebordercolor={0.5 1 0.5},
  linkcolor=blue]{hyperref}
  
 % Here is what I have added
 %% Useful packages
\usepackage{amsmath}
\usepackage{graphicx}
\usepackage[colorinlistoftodos]{todonotes}
\usepackage{breqn}
\usepackage{multirow}


%%New packages
\usepackage{physics}
\usepackage{caption}
\usepackage{subcaption}


% If you would like to compile this sample thesis without the
% hyperref package, then you will need to comment out the previous
% \usepackage command and uncomment the following command which will
% put the URL's in a typewriter font but not link them.
%\newcommand\url[1]{\texttt{#1}}

%******** setspace *******************************
% The setspace package allows you to manually set the spacing of the
% file.  UBC may require 1.5 spacing for microfilming of theses.  In
% this case you may obtain this by including this package and issuing
% one of the following commands:
%\usepackage{setspace}
%\singlespacing
%\onehalfspacing
%\doublespacing
\setlength\parindent{0pt}

% These commands are optional.  The defaults are shown.  You only
% need to include them if you need a different value
\institution{The University Of British Columbia}

% If you are at the Okanagan campus, then you should specify these
% instead.
%\faculty{The College of Graduate Studies}
%\institutionaddress{Okanagan}
\faculty{The Faculty of Graduate and Postdoctoral Studies}
\institutionaddress{Vancouver}

% You can issue as many of these as you have...
\previousdegree{B.Sc., Metropolitan State University of Denver, 2016}

% You can override the option setting here.
% \degreetitle{Jack of All Trades}

% These commands are required.
\title{Non-Smooth Dynamics in the Stommel Model of the Thermohaline Circulation}
\author{Cody Griffith}
\copyrightyear{2018}
\submitdate{\monthname\ \number\year} % The "\ " is required after
                                      % \monthname to prevent the
                                      % command from eating the space.
\program{Mathematics}

% These commands are presently not required for UBC theses as the
% advisor's name and title are not presently required anywhere.
\advisor{Rachel Kuske}
\advisortitle{Professor of Mathematics}

% One might want to override the format of the section and chapter
% numbers.  This shows you how to do it.  Note that the current
% format is acceptable for submission to the FoGS: If you wish to modify
% these, you should check with the FoGS explicity. prior to making
% the modifications.
\renewcommand\thepart         {\Roman{part}}
\renewcommand\thechapter      {\arabic{chapter}}
\renewcommand\thesection      {\thechapter.\arabic{section}}
\renewcommand\thesubsection   {\thesection.\arabic{subsection}}
\renewcommand\thesubsubsection{\thesubsection.\arabic{subsubsection}}
\renewcommand\theparagraph    {\thesubsubsection.\arabic{paragraph}}
\renewcommand\thesubparagraph {\theparagraph.\arabic{subparagraph}}

\setcounter{tocdepth}{2}
\setcounter{secnumdepth}{2}
\setcounter{figure}{0}

% Here is an example of a "Program" environment defined with the
% "float" package.  The list of programs will be stored in the file
% ubcsample.lop and the numbering will start with the chapter
% number.  The style will be "ruled".
\floatstyle{ruled}
\newfloat{Program}{htbp}{lop}[chapter]



% Here is the start of the document.
\begin{document}

%% This starts numbering in Roman numerals as required for the thesis
%% style and is mandatory.
\frontmatter

%%% The order of the following components should be preserved.  The order
%%% listed here is the order currently required by FoGS:        \\
%%% Title (Mandatory)                                           \\
%%% Preface (Mandatory if any collaborator contributions)       \\
%%% Abstract (Mandatory)                                        \\
%%% List of Contents, Tables, Figures, etc. (As appropriate)    \\
%%% Acknowledgements (Optional)                                 \\
%%% Dedication (Optional)                                       \\

\maketitle 	%% Mandatory

%% Committee Page
The following individuals certify that they have read, and recommend to the Faculty of Graduate and Postdoctoral Studies for acceptance, a thesis entitled:
$$\textbf{Non-Smooth Dynamics in the Stommel Model of the Thermohaline Circulation}$$

submitted by \textbf{Cody Griffith} in partial fulfillment of the requirements of the degree of \textbf{Master of Science} in \textbf{Applied Mathematics}.\\
\\

\textbf{Examining Committee:}\\
\\
\textbf{Rachel Kuske, Professor of Mathematics}\\
Supervisor\\
\\
\textbf{Brian Wetton, Professor of Mathematics}\\
Supervisory Committee Member

\begin{abstract}                %% Mandatory -  maximum 350 words
We analyzed the non-smooth dynamics for the Stommel model for thermohaline circulation with additional mechanisms like slowly varying bifurcation parameters and high frequency oscillatory forcing. Our goal was to find an analytic approximation to the tipping or bifurcation induced by the new features in the model. We first analyze a simpler one-dimensional model that has similar structure to the Stommel model and gradually build in more complexity into the one-dimensional problem and study the effects on the tipping point. With the one-dimensional model understood, we then apply similar methods to the Stommel model and study the effects on the non-smooth tipping and bifurcations. With these results we have the ability to fully describe the hysteresis found in the Stommel model.
\end{abstract}

\chapter{Lay Summary}
We study the behavior of the thermohaline circulation. This circulation is responsible for moving water from around the globe and thus a model was created to understand how it functions. Work has previously been done on certain components of this model, but the analysis we provide is done around the less studied pieces known as the non-smooth dynamics. We ultimately provide a solution to the non-smooth dynamics and even incorporate more mechanisms in the original model to account for a larger class of observable behavior. This allows for a better understanding of the thermohaline circulation and could help predict and prepare for sudden abrupt changes to the ocean currents that drive our climate.

\chapter{Preface} % Manditory if any of the conditions are met

This thesis is original, unpublished, independent work by the author, C. Griffith.

% Table of contents
\tableofcontents

\listoffigures

\newpage

\chapter*{Acknowledgments}

I would like to thank my supervisor Rachel Kuske for the project idea and guidance as well as my committee member Brian Wetton for helping me put together a coherent thesis.
\newline

Special thanks given to my peers Tim Jaschek and Matthias Kl{\"o}ckner for their aide in writing and support throughout the entire degree. Couldn't have done it without you both!

\newpage

%%%%% My Work
\mainmatter
\chapter{Introduction}
\label{chap:introduction}
Dynamical systems is the study of the possible states an observable solution may experience and is important in most engineering, biological or even chemical systems to name a few. This approach allows conditions to be given for when a solution can be found or when there is stable behavior. Often we find that parameters inherent in the model play huge roles in the dynamical behavior and they can be the difference between a system having an equilibrium or not. When we find a parameter that has this effect, we call it a bifurcation parameter since there is some value that changes the qualitative behavior of the system.

\indent For example, the Hodgkin-Huxley model for neurons contains a parameter for injected current $I$ which turns out to be a bifurcation parameter with a Hopf bifurcation. This bifurcation is responsible for the actual firing of a neuron in the brain. In epidemiological modeling, the basic SIR model with an additional transition function between the infected and recovered population causes the reproduction number $R_0$ to become a bifurcation parameter with a backward bifurcation. This causes a temporary equilibrium to form in the infected population that usually would never see an equilibrium. Even in activation potentials of neural networks, using a hyperbolic tangent function causes a bifurcation to occur in the synaptic feedback parameter $w$ which results in a pitchfork bifurcation. This causes wildly different equilibria for learned parameters in a machine learning setting. The canonical example is the \textit{saddle-node} bifurcation and was the first to be found within a complex system studied from a dynamical perspective. The saddle-node bifurcation has the locally topological equivalent form

\begin{equation}\label{eq:intro_saddlenode}
\dot{x}=a-x^2,
\end{equation}

where by locally topological equivalence we mean the behavior near the bifurcation may be represented in this form. This property is critical to reducing most complex problems and models to simpler local problems that can be studied individually.

\begin{figure}[H]
\centering
\includegraphics[width = .7\linewidth]{intro/saddlenode.jpg}
\caption{Vector field of a saddle-node bifurcation $a=0$.}
\label{fig:intro_saddlenode}
\end{figure}

\begin{figure}[H]
\centering
\includegraphics[width=.7\linewidth]{intro/saddlenode_bif_diagram.jpg}
\caption{Bifurcation diagram of saddle-node bifurcation $a=0$.}
\label{fig:intro_saddlenode_bif_diagram}
\end{figure}

\indent In figure~\ref{fig:intro_saddlenode} we show the vector field of the system that contains a saddle-node bifurcation. The equilibria of this system is $x=\pm \sqrt{a}$ for any $a\ge0$ where stable equilibrium points are marked with red filled points and unstable with unfilled points. Notice that at $a=0$ we are no longer able to find a stable equilibrium and when $a<0$ there are no equilibria at all. Thus $a=0$ is a simple example of a bifurcation where two equilibria annihilate. This behavior is why the bifurcation is often referred to as a \textit{fold} bifurcation, although we refer to this as a saddle-node bifurcation in this thesis. In figure~\ref{fig:intro_saddlenode_bif_diagram} we plot the same system against the parameter $a$, which we call the bifurcation diagram. Here we see the region with two equilibrium $a>0$, the bifurcation $a=0$ and the region of no stability $a<0$. For more on the saddle-node bifurcation see \cite{kuznetsov2006saddle}.

\indent There are many types of bifurcations that appear in different systems that each have their own key properties. Studying these properties leads to a deeper understanding of the system on both a global and a local scale. Work has been done on systems that have smooth bifurcations due to how commonly these appear, but non-smooth dynamics still are present in the physical world.

\indent Non-smooth bifurcations are a topic that arise in special systems and for how frequent they appear, they have not been studied nearly as much as their smooth counter parts. This thesis discusses the role of the non-smooth saddle-node bifurcation in a simplified one component system in \autoref{chap:oneD} as well as in the classic Stommel model for thermohaline circulation dynamics in \autoref{chap:twoD}. Many interesting ocean and weather mechanisms may be incorporated into the Stommel model to provide more realistic predictions for weather patterns. We choose to study slowly varying bifurcation parameters and their effect on the stability of a system while contrasting this with non-autonomous oscillatory forcing. The interaction of these features causes complex dynamics around the standard bifurcations and can lead to an advanced bifurcation or delayed tipping. For the one component system, a detailed analysis of these features is done on the smooth bifurcation in \cite{zhu2015tipping}.

\section*{Tipping in a Slowly Varying System}
A system with a parameter known to cause a bifurcation will no longer admit a bifurcation in the standard sense when the parameter slowly varies. Instead, these conditions give rise to a smooth but rapid change in the system's equilibria. The point in which this behavior occurs is then called a \textit{tipping point}.

\indent More formally, a tipping point is a point that causes an abrupt smooth transition in dynamical behavior as the system moves into a qualitatively different state. This is usually caused by some exterior control system that pushes change towards a different state once a critical point has been passed, for example with biological systems seen in \cite{angeli2004detection}. These are known to be caused by changes in one or more parameters in the system. An analysis that lays the theoretical backing of slowly varying parameters with algebraic bifurcations is found in \cite{haberman1979slowly}.

\indent Tipping points have been discovered to occur in a wide variety of systems and have become a big staple in the study of areas like catastrophe theory and dynamical systems. They aid in predicting the future of a system and even could be a warning for irreversible change like in the case of the Stommel model. A tipping point thus shares similar characteristics to a bifurcation and typically occurs close to the static bifurcation location.

\indent In this thesis we use the results from \cite{zhu2015tipping} where the system

\begin{equation}\label{eq:intro_Zhueq}
\begin{aligned}
\dot{x} =& Da + k_0 +k_1 x + k_2 x^2,\\
\dot{a} =& -\epsilon,
\end{aligned}
\end{equation}

where $\epsilon\ll 1$ was studied. This model is a slowly varying quadratic differential equation containing a smooth saddle-node bifurcation and appears in many physical models, for example \cite{erneux1989jump}. A key result from \cite{zhu2015tipping} is that the solution and tipping point for \eqref{eq:intro_Zhueq} have the form

\begin{equation}\label{eq:intro_Zhuairy}
x\sim \frac{1}{|k_2|}\left(\frac{k_1}{2}+\left(\frac{D|k_2|}{\epsilon}\right)^{1/3}\right)\frac{Ai'\left([D|k_2|\epsilon]^{-2/3}\left(\frac{k_1^2}{4}+k_0|k_2|+D|k_2|a\right)\right)}{Ai\left([D|k_2|\epsilon]^{-2/3}\left(\frac{k_1^2}{4}+k_0|k_2|+D|k_2|a\right)\right)}
\end{equation}

\begin{equation}\label{eq:intro_Zhuresult}
a_{\text{tip}}=(D|k_2|)^{-1/3}a_{\text{Airy}}-\frac{a_s}{D}\quad \text{for} \quad a_s = k_0+\frac{k_1^2}{4|k_2|},
\end{equation}

with $Ai(\cdot)$ being the Airy function and $a_{\text{Airy}}=\epsilon^{2/3}\cdot(-2.33810\ldots)$ corresponding to the first zero of the Airy function. The singularity found in \eqref{eq:intro_Zhuresult} is a recurring tool for the work presented in this thesis, even though we deal with a version of \eqref{eq:intro_Zhueq} that has a non-smooth bifurcation.

\begin{figure}[H]
\centering
\begin{subfigure}{.5\textwidth}
 \centering
 \includegraphics[width=\linewidth]{intro/saddlenode_tipping.jpg}
 \caption{}
\end{subfigure}%
\begin{subfigure}{.5\textwidth}
 \centering
 \includegraphics[width=\linewidth]{intro/saddlenode_tipping_zoom.jpg}
 \caption{}
\end{subfigure}
\caption{The saddle-node bifurcation. In (a) an example of tipping occurring around the bifurcation for two sizes of slow variation, $\epsilon=\{.01,.1\}$. In (b) a zoom in closer to the bifurcation. The dashed ($\epsilon = .01$) and dash-dotted ($\epsilon=.1$) black lines are the numerical solutions, we overlay the bifurcation diagram for reference.}
\label{fig:intro_tipping}
\end{figure}


\indent In figure~\ref{fig:intro_tipping} we show a numerical solution to the simple saddle-node system with tipping \eqref{eq:intro_Zhueq}.  Here we have $D=1$, $k_0=k_1=0$ and $k_2=-1$ which is the model from \eqref{eq:intro_saddlenode}. The solution follows closely to the stable branch even after the bifurcation for the static model, which is an example of this delayed behavior. From here on we refer to the numerical tipping point to be when the numerical solution has passed a threshold away from the equilibrium such that it is reasonable to say the solution is transitioning to a new state. We call this threshold the tipping criterion and it is specified whenever we compare our estimates  to the numerical solution.

The task of finding where tipping points occur depends on the situation, but in general the approach is to search for when a solution to a model fails or becomes large. Examples could be when the solution fails to be real or when an exponential term grows large, both of which are seen throughout this thesis.


\section*{The Stommel Model}

Global circulation models have primarily focused on three different categories: 

\begin{itemize}
\setlength\itemsep{1em}
\item Atmospheric components - the effect greenhouse gases have on the atmosphere,
\item Oceanic components - the effect of tides and interaction of temperature with salinity in the oceans,
\item Sea ice and land surface components.
\end{itemize}

These categories all contribute significantly to the overall prediction of weather and climate for the planet, which has importance to just about every industry and economy. Failure to adhere to and prepare for sudden changes in the climate has led to drastic situations like severe droughts or ocean acidification. Atmospheric models have been vastly studied but far less work has been done on the contribution from the ocean and the dynamics that drive the tides and currents.

\indent A key feature of oceanic models is when patterns form around regions of bi-stability of temperature and salinity. An example of this is the thermohaline circulation (THC) which has abrupt qualitative changes at certain points, see \cite{alley2003abrupt,marotzke2000abrupt,rahmstorf2000thermohaline,rahmstorf2002ocean}. Just earlier this year evidence was found of weakening occurring around these abrupt changes in a system of ocean patterns known as the Atlantic meridional overturning circulation (AMOC) \cite{caesar2018AMOC}. This is the first evidence of ocean dynamics responding to temperature change on the surface and can help further predict the future of the system. It is imperative that appropriate action is taken to prepare for the future of these type of systems as they are outside our realm of control. 

\indent To study these phenomena we create parametric models to replicate the dynamics we observe. Initially, Henry Stommel proposed the two box model in 1961 to understand the physics of the THC, shown in figure~\ref{fig:stommel_boxes}. In \cite{stommel1961thermohaline}, it is suggested that there are actually two different stability regimes which even overlap in the system that is proposed and concluded that oceanic dynamics behave very similarly about these equilibria. These type of systems have since been a heavily studied area for both climatology due to the wide ranging applications and dynamical systems for its generalization into bi-stability.

\begin{figure}[H]
\centering
\includegraphics[width=0.7\textwidth]{intro/Box.jpg}
\caption{The Stommel Two Box Model: Differing volume boxes with a temperature and salinity, $T_i$ and $S_i$. The boxes are connected by an overflow and capillary tube that has a circulation rate $V$. There is also a surface temperature and salinity for each box, ${T_i}^s$ and ${S_i}^s$. We assume that there is some stirring to give a well mixed structure.}
\label{fig:stommel_boxes}
\end{figure}

\indent With emphasis on mathematics, the focus of this thesis is on developing an effective approach to models with bi-stability and additional mechanisms. Thus the physical quantities are brushed aside in favor of their non-dimensional alternatives, see \autoref{app:stommel} for the derivation. The non-dimensionalized Stommel model is represented with the system

\begin{equation}
 \begin{aligned}
  \dot{T} & = \eta_1-T(1+|T-S|), \\
  \dot{S}   & = \eta_2-S(\eta_3+|T-S|). 
 \end{aligned}
\end{equation}

The variables $T$ and $S$ are the temperature and salinity respectively where the non-smoothness is seen directly from the $|T-S|$ term. The parameters $\eta_1$, $\eta_2$, and $\eta_3$ are all dimensionless quantities that each have physical interpretation to the relaxation times and volumes of the box. Here $\eta_1$ is thought of as the thermal variation, $\eta_2$ as the saline variation otherwise known as the freshwater flux, and $\eta_3$ as the ratio of relaxation times of temperature and salinity. It also is a physical restriction for both $\eta_1$ and $\eta_3$ to be positive quantities that take any value. The parameter $\eta_3$ has the additional property to determine the orientation of the equilibria. We denote a standard orientation to be when $\eta_3<1$, reverse orientation for $\eta_3>1$, and $\eta_3=1$ a special case. The different orientations are shown in figure~\ref{fig:Stommel_bif_plots}. Recall that $\eta_3$ is the ratio of relaxation rates and when $\eta_3=1$ the relaxation rates for both the thermal and salinity variables are the same. Under these conditions we lose bi-stability and instead see a single stable equilibrium.

\indent The parameter $\eta_2$ is the most interesting as different values cause major qualitative and quantitative changes in the dynamics of the system. Bifurcations have been discovered at two different points in the system, each being called either a smooth or a non-smooth saddle-node bifurcation. In the Stommel model, it is convenient to view the system in terms of the circulation rate $V=T-S$, see \autoref{app:stommel} for the derivation. This leads to the system

\begin{equation}\label{eq:basic_stommel}
 \begin{aligned}
  \dot{T} & = \eta_1-T(1+|V|), \\
  \dot{V}   & = (\eta_1-\eta_2)-V|V|-T+\eta_3(T-V).
 \end{aligned}
\end{equation}

\begin{figure}[H]
\centering
\begin{subfigure}{.5\textwidth}
 \centering
 \includegraphics[width=\linewidth]{intro/T_equil.jpg}
 \caption{$V$ vs. $T$}
 \label{fig:Tequil}
\end{subfigure}%
\begin{subfigure}{.5\textwidth}
 \centering
 \includegraphics[width=\linewidth]{intro/V_bif.jpg}
 \caption{$\eta_2$ vs. $V$}
 \label{fig:Vbif}
\end{subfigure}
\caption{The equilibria of the non-dimensionalized system \eqref{eq:basic_stommel}. Parameters values are $\eta_1=4$ and $\eta_3=.375$. The above plots are two-dimensional projections of the full 3-dimensional system in ($\eta_2$,$V$,$T$). We see non-smooth behavior happening in both plots when $V=0$. The red line indicates a stable branch where the dashed dotted line is for an unstable branch.}
\label{fig:systemequil}
\end{figure}

\indent As shown in figure~\ref{fig:systemequil}, the equilibrium curves reveal much about the dynamics. In (a) the graph of the equilibria for $V$ vs. $T$ shows non-smooth behavior occurring at $V=0$ and in (b) the two types of bifurcation appear clearly in the graph of equilibria for $\eta_2$ vs. $V$. In this plot, both the upper and lower branches of the equilibrium are stable with the middle branch being unstable. The stable branches relate to which variable is dominant. For the lower branch, we call this the saline branch, and the upper branch the thermal branch. The location of the non-smooth bifurcation is found analytically, $({\eta_2}_{\text{ns}},V_{\text{ns}},T_{\text{ns}})=(\eta_1\eta_3,0,\eta_1)$, and the smooth bifurcation, $({\eta_2}_{\text{smooth}},V_{\text{smooth}},T_{\text{smooth}})$, is the only real solution to a cubic polynomial. The smoothness of each bifurcation is apparent and arises from the absolute value term in the defining dynamics of \eqref{eq:basic_stommel}, which is non-smooth only at $V=0$.

\begin{figure}[H]
\centering
\begin{subfigure}{.5\textwidth}
 \centering
 \includegraphics[width=\linewidth]{intro/V_bif.jpg}
 \caption{$\eta_3=.375$}
\end{subfigure}%
\begin{subfigure}{.5\textwidth}
 \centering
 \includegraphics[width=\linewidth]{intro/V_bif_collapse.jpg}
 \caption{$\eta_3=1$}
\end{subfigure}
\begin{subfigure}{.5\textwidth}
 \centering
 \includegraphics[width=\linewidth]{intro/V_bif_reverse.jpg}
 \caption{$\eta_3=1.875$}
\end{subfigure}
\caption{The choice in $\eta_3$ dictates the orientation of the problem, in each plot we have fixed $\eta_1=4$. The case for $\eta_3=1$ is special due to the two bifurcations overlapping and the unstable equilibrium vanishing.}
\label{fig:Stommel_bif_plots}
\end{figure}

\indent Much is known about the Stommel model in the case where $\eta_2$ is fixed but realistically this is not the case. In \cite{rahmstorf2000thermohaline}, this parameter is described as the influx of freshwater into the Atlantic and the changing nature of $\eta_2$ is justified by a positive feedback loop for salinity that drives the THC to move high-salinity water towards deep oceans. This loop causes the abrupt smooth bifurcation but then afterwards, a salinity deficit causes the parameter to decrease back towards the non-smooth bifurcation.

\indent This type of behavior is known as hysteresis, where there is some bi-stability region that the solution cycles through and observes both states of the equilibria. A similar analysis to the Stommel model's hysteresis can be found in \cite{roberts2017relaxation}. The phenomena of hysteresis appears in many physical systems, for example \cite{jung1990scaling,hohl1995scaling,joshi2005dynamical}. The smooth component of the hysteresis curve has been studied in a reduced one component model, see \cite{zhu2015tipping}. In this thesis we complement these results with an analysis of the non-smooth component.

\section*{Numerical Methods}

To obtain numerical solutions to the ordinary differential equations studied in this thesis, we choose to use both 2nd order and 4th order Runge-Kutta methods. The 2nd order method are used for the one component model and the 4th order method for the two component model. The choice in these methods comes from using the simplest scheme since numerical sensitivity is not present in our problem. We use the numerical solutions to compare our approximations to the observed state transition.

\chapter{One-Dimensional Model}
\label{chap:oneD}
We consider a simpler system to give insight into the more complex two component Stommel model. Here we use a toy system to build the analysis on and the spatial variables may not have a physical interpretation. This system is the following one component model in terms of the variables $x$ and $\mu$

\begin{equation}\label{eq:oneD_canonical}
\begin{aligned}
\dot{x}=-\mu+2|x|&-x|x|+A\sin(\Omega t),\\
\dot{\mu}=&-\epsilon,
\end{aligned}
\end{equation}

\begin{equation*}
x(0)=x^0,\quad\mu(0)=\mu^0,
\end{equation*}

where the fixed parameters are the slow variation rate of $\epsilon \ll 1$, the amplitude of oscillation $A$ and the frequency of oscillation $\Omega$. We also assume the initial conditions to be ${x^0=1-\sqrt{1+\mu^0}}$ and $\mu^0>\mu_{\text{ns}}$ which focuses our calculations on the lower equilibrium branch where $x<0$ and study nearby behavior. The value $\mu_{\text{ns}}$ refers to the non-smooth bifurcation which is discussed below in \autoref{sec:oneD_static}.

\indent The system \eqref{eq:oneD_canonical} is generalized from a basic model that contains both a smooth and non-smooth saddle-node bifurcation. This structure is similar to the Stommel model and hence a good model to test features like slow variation or oscillatory forcing. The slow variation is clear, but we use oscillatory forcing here in preparation for the two component model of the next chapter. In each case, emphasis is put on the non-smooth component of the model to study the non-smooth bifurcation and the role it plays in the hysteresis curve we anticipate in the Stommel model.

\section{Static Bifurcations}
\label{sec:oneD_static}

The foundation to our understanding comes from the simplest structure lying within the canonical system \eqref{eq:oneD_canonical} which is the bifurcation structure. This means finding the general form for the equilibria in \eqref{eq:oneD_canonical} with $A=0$ and $\epsilon=0$, which is our basic model with a static $\mu$ and no forcing. As we have a fixed parameter value, we search for a point or set of points that the solution relaxes to as $t\to \infty$. We call these points the equilibrium points and they are either stable or unstable. Since we are considering all possible $\mu$, we want all of the equilibrium points for each $\mu$ and thus we call these the equilibrium branches.

\indent To find all equilibrium branches, we search for when the solution has come to a rest, which is equivalent to setting the derivative of $x$ to zero. Thus we set \eqref{eq:oneD_canonical} to zero with

\begin{equation}\label{eq:oneD_static_equil}
0=-\mu +2|x|-x|x|.
\end{equation}

Solving \eqref{eq:oneD_static_equil} results in 3 solutions where the stability of each is characterized by small perturbations to the equilibrium either linearly growing or decaying. We denote the stable equilibria as $x_l$ and $x_u$ for the lower and upper branches respectively, and a single unstable middle branch, $x_{m}$. These are given by

\begin{equation*}
x_l=1-\sqrt{1+\mu},\quad x_u=1+\sqrt{1-\mu},\quad
x_{m}=1-\sqrt{1-\mu}.
\end{equation*}

\indent We note that $x_l$ is valid for $\mu\ge 0$ and both $x_u$ and $x_{m}$ for $\mu\le 1$. Thus this system has a stable equilibrium for each value of the parameter and has a region of bi-stability for $0\le \mu\le 1$. The boundaries of this region are ${(\mu_{\text{ns}},x_{\text{ns}})=(0,0)}$ and $(\mu_{\text{smooth}},x_{\text{smooth}})=(1,1)$ which are the non-smooth and smooth saddle-node bifurcations respectively. Both are saddle-node due to pairs of equilibria annihilating at these locations which is shown in figure~\ref{fig:oneD_static_bifdiagram}.

\begin{figure}[H]
\centering
\includegraphics[width=.7\textwidth]{oneD/bif_diagram.jpg}
\caption{The one component bifurcation diagram with the upper and lower equilibrium branches as well as the unstable middle branch. The non-smooth bifurcation occurs at (0,0) denoted by the circle and the smooth bifurcation occurs at (1,1) by the box. }
\label{fig:oneD_static_bifdiagram}
\end{figure}


\section{Slowly Varying Bifurcation Parameter}
\label{sec:oneD_slow}

To develop a method for the slowly varying Stommel model, we consider \eqref{eq:oneD_canonical} with $\epsilon\ll 1$ and $A=0$. Under these conditions, $\mu(t)$ is a function of time and thus a bifurcation no longer occurs. Instead, it is expected that a tipping point occurs nearby the static bifurcation points as long as $\epsilon$ is small. Also, due to $\mu(t)$ being a function of time, we will find equilibria that are also functions in time, which we call pseudo-equilibira. The smooth case is well understood, see \cite{zhu2015tipping}, so we consider the behavior of the non-smooth bifurcation with $x<0$. From \cite{haberman1979slowly} as well as the smooth model \cite{zhu2015tipping}, it is common practice to rescale time in a model with slow variation to put the dynamics on the same order and allow for algebraic solutions to be found. Here the parameter $\mu(t)$ is slowly varying in time so it makes sense to rescale using this as our slow time, $\tau=\epsilon t$. Applying both $x<0$ and this slow time approach to the system \eqref{eq:oneD_canonical} then gives

\begin{equation}\label{eq:oneD_slow_scaled}
\begin{aligned}
\epsilon x_\tau=&-\mu(\tau)-2x+x^2,\\
\mu_\tau=&-1.
\end{aligned}
\end{equation}

\indent A standard approach to extracting information out of complicated models is to find reduced equations by separating the behavior at each order of the slow time. This approach is known as using an asymptotic expansion and further details can be found in Murray's \textit{Asymptotic Analysis} \cite{murray2012asymptotic}. With $\epsilon$ being the small quantity that dictates our slow time, we choose to use an asymptotic expansion of $x$ with

\begin{equation}\label{eq:oneD_slow_asympexpan}
x(\tau)\sim x_0(\tau)+\epsilon x_1(\tau)+\epsilon^2 x_2(\tau)+O(\epsilon^3).
\end{equation}

This approach captures the slowly varying behavior of the solution in terms of this small quantity $\epsilon$ and aims to relate the slow variation to the solution. We substitute the expansion \eqref{eq:oneD_slow_asympexpan} into the scaled system \eqref{eq:oneD_slow_scaled} to get

\begin{equation*}
\epsilon {x_0}_\tau +\epsilon^2 {x_1}_\tau+\ldots= -\mu(\tau) -2x_0+x_0^2+\epsilon(-2x_1+2x_1x_0)+\epsilon^2(-2x_1+2x_2x_0+x_1^2)+\ldots
\end{equation*}

Once we separate the equations at each order of $\epsilon$, we find the following system of equations

\begin{align}
\label{eq:oneD_slow_outerO1}
O(1):& \quad 0=-\mu(\tau)-2x_0+x_0^2,\\
\label{eq:oneD_slow_outerO2}
O(\epsilon):& \quad 0=-{x_0}_\tau-2x_1+2x_1 x_0,\\
\label{eq:oneD_slow_outerO3}
O(\epsilon^2):& \quad 0=-{x_1}_\tau-2x_2+2x_2x_0+x_1^2.
\end{align}

Each of the equations \eqref{eq:oneD_slow_outerO1}-\eqref{eq:oneD_slow_outerO3} gives the respective order's pseudo-equilibrium. Thus we solve each equation progressively to find the terms of our asymptotic expansion \eqref{eq:oneD_slow_asympexpan} as

\begin{equation}\label{eq:oneD_slow_outersoln}
x(t)\sim 1-\sqrt{1+\mu(t)}+ \frac{\epsilon}{4(1+\mu(t))}-\frac{3\epsilon^2}{32(1+\mu(t))^{5/2}}+O(\epsilon^3).
\end{equation}

We call \eqref{eq:oneD_slow_outersoln} the outer solution as it approximates the solution well for values of $x(t)$ away from the bifurcation value $\mu_{\text{ns}}$. Since the dynamics of the system \eqref{eq:oneD_canonical} change at $x=0$ due to the non-smooth bifurcation of the underlying static system, this solution is valid only for $x<0$ and $\mu>0$.


\indent It is a key assumption of an asymptotic expansion that the terms are clearly separated by order of $\epsilon$. We search for a scaling of $\mu$ and $x$ for which \eqref{eq:oneD_slow_outersoln} is no longer valid under this assumption of order separation. This assumption fails when $x_0\sim \epsilon x_1$ which occurs here for $\mu\sim O(\epsilon)$. To confirm, we conduct a simple scale analysis to determine the appropriate scaling for the local analysis about $x=0$. Hence we consider the general scales

\begin{equation*}
x=\epsilon^\alpha y,\quad \mu = \epsilon^\beta m,
\end{equation*}

with $\alpha>0$ and $\beta>0$ for an inner scaling. We apply these local variables in \eqref{eq:oneD_canonical} to get the system

\begin{equation}\label{eq:oneD_slow_scalesearch}
\begin{aligned}
\epsilon^\alpha \dot{y}=&-\epsilon^\beta m +\epsilon^\alpha 2|y|-\epsilon^{2\alpha}y|y|,\\
 \epsilon^\beta \dot{m}=&-\epsilon.
\end{aligned}
\end{equation}

We balance the leading order terms $\epsilon^\alpha\dot{y}$ with $\epsilon^\beta m$ to find $\alpha=\beta$. Here the equation for $m$ calls for $\beta=1$, thus we have the scaling for the local analysis

\begin{equation}\label{eq:oneD_slow_scales}
x=\epsilon y,\quad \mu=\epsilon m.
\end{equation}

We have found that the scalings in \eqref{eq:oneD_slow_scales} apply to all $x$ and thus we consider the region of $x>0$. Substituting the local variables \eqref{eq:oneD_slow_scales} into the original model \eqref{eq:oneD_canonical} we find the following inner system for the region of $x>0$

\begin{equation}\label{eq:oneD_slow_innereq}
\begin{aligned}
\dot{y}=&-m(t)+2 y-\epsilon y^2,\\
\dot{m}=&-1.
\end{aligned}
\end{equation}

We recall that we are searching for a link between $y$ and $m$, and from \cite{haberman1979slowly} we use that it is then convenient to change the differentiation on $y$ to be with respect to the slowly varying parameter $m$. This incorporates the behavior of $m(t)$ directly into the equation we solve and gives us a direct method for finding the tipping point. Then the leading order equation is

\begin{equation}\label{eq:oneD_slow_innerm}
y_m = m-2y.
\end{equation}

The leading order solution to \eqref{eq:oneD_slow_innerm} is found explicitly as follows

\begin{equation*}
y(m) = C e^{-2m}+\frac{m}{2}-\frac{1}{4}+O(\epsilon).
\end{equation*}

With the inner solution found in terms of the parameter $m$, we write this in terms of the original variables with

\begin{equation}\label{eq:oneD_slow_innersoln}
x(t)\sim Ce^{-2\mu(t)/\epsilon}+\frac{\mu(t)}{2}+O(\epsilon).
\end{equation}

We call the solution \eqref{eq:oneD_slow_innersoln} the inner solution as it approximated the solution well near the bifurcation value $\mu_{\text{ns}}$. Since the inner solution behaves exponentially, the tipping point, $\mu_{\text{slow}}$, occurs when the exponential term begins to grow rapidly. Here we consider tipping to occur when the solution becomes $O(1/\epsilon)$. Then we find the tipping point $\mu_{\text{slow}}$ to take the form

\begin{equation}\label{eq:oneD_slow_tipping}
\mu_{\text{slow}}= \frac{1}{2}\epsilon \log (\epsilon).
\end{equation}

\indent Thus we have the tipping point for the slowly varying model. Notice that for small values of $\epsilon$, $\mu_{\text{slow}}<\mu_{\text{ns}}$ and this is consistent with considering the inner equation \eqref{eq:oneD_slow_innereq} for the region $x>0$ as we found in the analysis. Hence we find that a slowly varying bifurcation parameter causes a delay in the rapid transition to the upper branch and we expect the solution to remain near the lower branch for longer than in the static problem. In terms of hysteresis, then slow variation allows for a longer period before the states switch from the lower to the upper branch.

\indent In figure~\ref{fig:oneD_slow_numerics} (a,b), two examples of this tipping is shown for different sizes of $\epsilon$ along with the standard bifurcation diagram where (c) demonstrates the tipping approximation across a range of $\epsilon$. The concavities match as well as agreement in the estimation of the tipping point as $\epsilon$ goes to 0.

\begin{figure}[H]
\centering
\begin{subfigure}{.5\textwidth}
 \centering
 \includegraphics[width=\linewidth]{oneD/slow_bif_diagram.jpg}
 \caption{}
\end{subfigure}%
\begin{subfigure}{.5\textwidth}
 \centering
 \includegraphics[width=\linewidth]{oneD/slow_bif_diagram_zoom.jpg}
 \caption{}
\end{subfigure}
\begin{subfigure}{.5\textwidth}
\centering
\includegraphics[width=\linewidth]{oneD/slow_epscomp.jpg}
\caption{}
\label{fig:oneD_slow_comp}
\end{subfigure}
\caption{In (a) the numerical solutions (black dashed and dash-dotted lines) to \eqref{eq:oneD_canonical} are given with $A=0$ and $\epsilon=\{.01,.04\}$ respectively. The bifurcation plot is overlayed for convenience. In (b) a zoom in of what happens near the non-smooth bifurcation. The solid vertical lines (black) are tipping points where we use the tipping criterion $x>.5$ on the numerical solution. The dashed and dash-dotted vertical lines (blue) are the tipping estimates. In (c) a range of $\epsilon$ and their corresponding tipping (red stars) are compared to our estimate (solid black line) from \eqref{eq:oneD_slow_tipping}.}
\label{fig:oneD_slow_numerics}
\end{figure}


\subsection{Stability}
From the static model we know our outer solution \eqref{eq:oneD_slow_outersoln} to be stable, but to verify that the inner solution \eqref{eq:oneD_slow_innersoln} is stable we use a simple linear stability analysis on the inner system. Typically to do this, an analysis would be performed about an equilibrium to see if perturbations would grow or decay. Although, in this model there is a parameter that is allowed to vary and hence we must be careful to note the analysis is about the pseudo-equilibrium instead. In the first region of interest, $m(t)\ge 0$, the following inner equation and pseudo-equilibrium, $z^0(t)$, hold below the axis

\begin{equation}\label{eq:oneD_slow_stability1}
\dot{y}=-m(t)-2y=f(t,y), \quad z^0(t)=-\frac{m(t)}{2}.
\end{equation}

We then consider simple perturbations of the pseudo-equilibrium, $u$, in \eqref{eq:oneD_slow_stability1} with

\begin{equation*}
y(t)=z^0(t)+u(t), \quad \lVert u(t) \rVert \ll 1.
\end{equation*}

\indent Normally, a Taylor expansion would result in expressing the perturbations with their own equation that we could use to determine stability. Since $z^0(t)$ is not fixed, we must consider its contribution to the derivative in this region of the parameter space with $m(t)\ge 0$. Thus we find

\begin{equation}\label{eq:oneD_slow_stability2}
\begin{aligned}
\dot{y} =& \,\dot{z}^0+\dot{u},\\
\dot{z}^0= & \begin{cases}
-\frac{1}{2}\dot{m}=\frac{1}{2} & m(t)>0,\\
0 & m(t)=0.
\end{cases}
\end{aligned}
\end{equation}

Now we apply the standard Taylor expansion to see the behavior of these perturbations and with the contributions in \eqref{eq:oneD_slow_stability2}, the inner equation \eqref{eq:oneD_slow_stability1} becomes

\begin{equation}\label{eq:oneD_slow_perturbeq}
\begin{aligned}
\dot{y}=& f(t,z^0)+f_y(t,z^0)(y-z^0)
= f_y(t,z^0)u,\\
\dot{u}=&\begin{cases}
-\frac{1}{2}-2u, & m(t)>0,\\
-2u, & m(t)=0.
\end{cases}
\end{aligned}
\end{equation}

\indent If this were the static parameter problem, we would always have the second case in \eqref{eq:oneD_slow_perturbeq}, which is always stable due to the sign. Since we allow for a varying parameter, we learn that the solution is attracted to just below the pseudo-equilibrium $z^0(t)$. As this system always experiences the critical point $m=0$ due to the continuous decrease in $m(t)$, the slowly varying parameter eventually acts like the static parameter in \autoref{sec:oneD_static}. Hence we have that for $x<0$, the pseudo-equilibrium is hyperbolic and asymptotically stable. Here there is a critical point at $(\mu_{\text{ns}},x_{\text{ns}})=(0,0)$ which corresponds to a non-hyperbolic equilibrium point. Generally, non-hyperbolic behavior signals equilibrium structures to change. Here, this signals a transition in behavior for $x>0$ and helps identify that the tipping occurs in this region.

\indent For the second region of interest, $m(t)<0$, we found a solution that had the following inner equation which has the pseudo-equilibrium above the axis with

\begin{equation}\label{eq:oneD_slow_innerstability}
\dot{y}=-m(t)+2y, \quad z^0(t) = \frac{m(t)}{2}.
\end{equation}

In \eqref{eq:oneD_slow_innerstability} we find a contradiction, here $m(t)<0$ yet the solution of this region is above the axis $x=0$. Thus we may conclude that this inner equation has no equilibrium in this region and further verifies that the critical point $(\mu_{\text{ns}},x_{\text{ns}})$ was non-hyperbolic and tipping occurs for $m(t)<0$.

\section{High Frequency Oscillatory Forcing}
\label{sec:oneD_highfreqosc}

To understand the oscillatory forcing in the Stommel model, consider the canonical system \eqref{eq:oneD_canonical} with $A\sim O(1)$, $\Omega\gg 1$ and $\epsilon=0$, which gives high frequency oscillatory forcing in the system. Under these conditions, we have a static parameter and for each parameter value there is oscillatory forcing with solutions characterized by oscillations about a fixed point. Thus we should expect to find a bifurcation influenced by oscillations occurring under these conditions. Here we develop a method to find oscillatory solutions to determine what the effect of oscillatory forcing has on the bifurcation of \eqref{eq:oneD_canonical}. In \autoref{sec:oneD_slow}, we focused only on the slowly varying dynamics but here we have both a slow time scale $t$ and a fast time scale $T=\Omega t$. This naturally suggests a multiple scales approach where we search for a solution that is dependent on both of these scales, $x(t)=x(t,T)$. This method is commonly used in problems that have behavior observable on multiple scales, and we use it here to find a way to accurately analyze each scale and effectively combine their behavior into a single unifying solution. Further discussion on this method can be found in \cite{sanchez1996method}.

\indent Recall that our focus is on the non-smooth behavior and hence we restrict the solution to follow along the lower stable equilibrium branch where $x<0$. Using this multiple scales approach, our canonical system \eqref{eq:oneD_canonical} has the following form

\begin{equation}\label{eq:oneD_osc_multiscale}
x_T+\Omega^{-1}x_t=\Omega^{-1}\left(-\mu-2x+x^2+A\sin(T)\right).
\end{equation}

\textbf{Note:} We choose to use the subscript notation for partial derivatives, $\frac{\partial x}{\partial T}=x_T$. In \eqref{eq:oneD_osc_multiscale}, the small quantity $\Omega^{-1}$ appears which suggests an asymptotic expansion in powers of this quantity

\begin{equation}\label{eq:oneD_osc_asymptotic}
x(t,T)\sim x_0(t,T)+\Omega^{-1}x_1(t,T)+\Omega^{-2}x_2(t,T)+O(\Omega^{-3}).
\end{equation}
Substituting \eqref{eq:oneD_osc_asymptotic} into \eqref{eq:oneD_osc_multiscale}, we find

\begin{equation*}
{x_0}_T+\Omega^{-1}{x_0}_t+\Omega^{-1}{x_1}_T+\ldots=\Omega^{-1}(-\mu-2x_0+x_0^2+A\sin(T))+\Omega^{-2}(-2x_1+2x_1x_0)+\ldots
\end{equation*}

Here we separate by each order of $\Omega$ to find the following system of reduced equations

\begin{align}
\label{eq:oneD_osc_outerO1}
O(1):& \quad {x_0}_T=0, \\
\label{eq:oneD_osc_outerO2}
O(\Omega^{-1}):& \quad {x_1}_T+{x_0}_t =-\mu-2x_0+x_0^2+A\sin(T),\\
\label{eq:oneD_osc_outerO3}
O(\Omega^{-2}):& \quad {x_2}_T + {x_1}_t= -2x_1+2x_0 x_1.
\end{align}

With an equation at each order, we must be able to solve each equation to proceed to the next but we must also further restrict our solution from having resonant or linearly growing terms to prevent any multiplicity or exponential growth. This assures that the terms in the asymptotic expansion are compatible with one another and we are able to find a robust solution. A common method to guarantee compatible solutions with sublinear growth at each order is the Fredholm alternative. This provides a solvability condition for each equation of the form ${x_i}_T=R_i(t,T)$ with

\begin{equation*}
\lim\limits_{T\to\infty}\frac{1}{T}\int_0^T R_i(t,u)\,du=0,
\end{equation*}

although for this system we consider the periodic form of the Fredholm alternative

\begin{equation} \label{eq:Fredholm}
\frac{1}{2\pi}\int_0^{2\pi}R_i(t,T)\,dT=0.
\end{equation}

\indent Both the general and periodic form of the Fredholm alternative have been well studied and a more theoretic approach to the periodic version is discussed in Bensoussan's \textit{Asymptotic analysis for periodic structures} \cite{bensoussan2011asymptotic}. From \eqref{eq:oneD_osc_outerO1}, we learn the leading order term is only dependent on the slow time, $x_0=x_0(t)$. Applying the Fredholm alternative \eqref{eq:Fredholm} to \eqref{eq:oneD_osc_outerO2} gives an equation for the slow behavior which then implies an equation for the fast behavior with

\begin{equation}\label{eq:oneD_osc_outerO2soln}
\begin{aligned}
0=\frac{1}{2\pi}\int_0^{2\pi}&\left( -{x_0}_t(t) -\mu -2x_0(t)+x_0(t)^2+A\sin(T)\right)\,dT ,\\
{x_0}_t=& -\mu -2x_0+x_0^2 , \quad{x_1}_T = A\sin(T).
\end{aligned}
\end{equation}

Solving for the equilibrium solution of \eqref{eq:oneD_osc_outerO2soln} leads to the leading order solution, $x_0$, and also allows us to partially solve for the first correction term $x_1$ with

\begin{equation*}
\begin{aligned}
x_0 =& 1-\sqrt{1+\mu},\\
x_1(t,T) =& v_1(t) - A\cos(T).
\end{aligned}
\end{equation*}
Repeating this procedure in \eqref{eq:oneD_osc_outerO3}, as shown in \autoref{app:oneD}, results in the expansion \eqref{eq:oneD_osc_asymptotic} written in the original variables

\begin{equation}\label{eq:oneD_osc_outersoln}
x\sim 1-\sqrt{1+\mu}-\Omega^{-1} A \cos(\Omega t)+O(\Omega^{-2}).
\end{equation}

Once again, the explicit outer solution \eqref{eq:oneD_osc_outersoln} performs well for $x$ away from the axis $x=0$, we search for when the assumptions of the asymptotic series fail indicating where an inner analysis is needed. This is when $x_0\sim \epsilon x_1$ which occurs for $\mu\sim O(\Omega^{-1})$.

\indent We consider a general scaling in the form of $x=\Omega^{-\alpha}y$ and $\mu = \Omega^{-\beta}m$ where $\alpha>0$ and $\beta>0$ allow for an inner equation to be found. Applying these local variables to \eqref{eq:oneD_canonical} results in

\begin{equation}\label{eq:oneD_osc_generalinner}
\dot{y} = -\Omega^{\alpha-\beta}m+2|y|-\Omega^{-\alpha}y|y|+\Omega^{\alpha}A\sin(\Omega t).
\end{equation}
In the local system \eqref{eq:oneD_osc_generalinner}, we find similar behavior occurring across multiple scales and thus we are able to use the same time scales from the outer analysis, $t=t$ and $T=\Omega t$. Then we have $y(t)=y(t,T)$ and hence a similar multiple scales argument in \eqref{eq:oneD_osc_generalinner} leads to

\begin{equation}\label{eq:oneD_osc_innergeneralmulti}
y_T+\Omega^{-1}y_t = - \Omega^{\alpha-\beta-1}m+\Omega^{-1}2|y|-\Omega^{-\alpha-1}y|y|+\Omega^{\alpha-1}A\sin(T).
\end{equation}

With a standard balancing argument between the leading order terms in \eqref{eq:oneD_osc_innergeneralmulti} $y_T$ and $\Omega^{\alpha-1} A\sin(T)$, we see that $\alpha=1$. We also want to see the terms $\Omega^{\alpha-\beta-1}m$ balance with $\Omega^{-1}2|y|$, which gives us that $\beta=1$ as well. This results in the inner equation

\begin{equation}\label{eq:oneD_osc_naivemultiscales}
y_T+\Omega^{-1}y_t = \Omega^{-1}\left(-m+2|y|\right)-\Omega^{-2}y|y|+A\sin(T).
\end{equation}

Similarly to the outer equation, we approximate the solution with an asymptotic expansion in terms of $\Omega^{-1}$ 

\begin{equation}\label{eq:oneD_osc_innerasymptotic}
y(t,T)\sim y_0(t,T)+\Omega^{-1}y_1(t,T)+O(\Omega^{-2}).
\end{equation}

Substituting the expansion \eqref{eq:oneD_osc_innerasymptotic} into the inner equation \eqref{eq:oneD_osc_naivemultiscales} we find

\begin{equation*}
{y_0}_T+\Omega^{-1}{y_0}_t+\Omega^{-1}{y_1}_T+\ldots =\begin{aligned}[t]\Omega^{-1}&(-m+2|y_0+\Omega^{-1}y_1+\ldots|)+A\sin(T)\\
&+\Omega^{-2}(y_0+\Omega^{-1}y_1+\ldots)|y_0+\Omega^{-1}y_1+\ldots|.
\end{aligned}
\end{equation*}

Here we then find the following system of equations at each order of $\Omega$

\begin{align}
\label{eq:oneD_osc_innerO1}
O(1):\quad & {y_0}_T = A\sin(T),\\
\label{eq:oneD_osc_innerO2}
O(\Omega^{-1}):\quad & {y_1}_T+{y_0}_t = -m+2|y_0|.
\end{align}

Solving the leading order equation \eqref{eq:oneD_osc_innerO1} gives that the leading order term has the form, $y_0(t,T)=v_0(t)-A\cos(T)$. Applying the Fredholm alternative \eqref{eq:Fredholm} to \eqref{eq:oneD_osc_innerO2} leads to

\begin{equation}\label{eq:oneD_osc_integral}
{v_0}_t(t)=-m+\frac{1}{\pi}\int_0^{2\pi} |v_0(t)-A\cos(T)|\,dT.
\end{equation}

\indent In this setting, we must consider two cases for $v_0(t)$ that determine the nature of this integrand. Case I: if $v_0(t)$ is large enough to keep $y_0$ from ever changing sign and Case II: if $v_0(t)$ is too small and $y_0$ crosses the $x=0$ axis. In figure~\ref{fig:oneD_osc_cases} we show the range of each case, the region on the right is following under case I, the green dotted vertical line defining the parameter range between the cases, the middle region for case II and the blue vertical line giving the bifurcation, $\mu_{\text{osc}}$, which is determined below.

\begin{figure}[H]
\centering
\includegraphics[width=.7\textwidth]{oneD/osc_cases.jpg}
\caption{The parameter ranges for each case are shown here with $A=2$. For reference, the original bifurcation diagram is overlayed.}
\label{fig:oneD_osc_cases}
\end{figure}

\subsection{Case I: $v_0(t) \le -|A|$}
\label{subsec:oneD_osc_CaseI}

We call this the 'below axis' case as the solution stays far from the axis $x=0$ for most of the oscillation and thus the behavior is not influenced by the non-smooth dynamics. We do not expect to see the bifurcation occur under these conditions but instead we find the parameter range for each of these cases. Here the integral in equation \eqref{eq:oneD_osc_integral} is straightforward to evaluate as $v_0(t)$ is a constant with respect to the fast time $T$, thus we find the inner equation and equilibrium

\begin{equation*}
{v_0}_t=-m-2v_0,\quad v_0=-\frac{m}{2}.
\end{equation*}

This gives the leading order equilibrium solution with oscillations of the local variables for this case which we write in terms of the original variables

\begin{equation}\label{eq:oneD_osc_caseIsoln}
\begin{aligned}
y(t,T)\sim& -\frac{m}{2}-A\cos(T)+O(\Omega^{-1}),\\ 
x(t)\sim& -\frac{\mu}{2}-\Omega^{-1} A\cos(\Omega t)+O(\Omega^{-2}).
\end{aligned}
\end{equation}

The condition $v_0(t)\le -|A|$ combined with the equilibrium allows us to establish when \eqref{eq:oneD_osc_caseIsoln} holds

\begin{equation}\label{eq:oneD_osc_caseboundary}
\mu\ge \frac{2|A|}{\Omega}.
\end{equation}

Following the equilibrium to \eqref{eq:oneD_osc_caseboundary} leads us to case II where the oscillations cross the axis and the assumptions of this case no longer hold.

\subsection{Case II: $|v_0(t)|< |A|$}
\label{subsec:oneD_osc_CaseII}

We call this the 'crossing' case; here the equilibrium is small enough that the oscillations can now push the solution above the axis. Under these conditions, the solution spends time near the axis $x=0$ and thus experiences non-smooth influence. As the crossing continues, the non-smooth behavior drives the solution to gradually grow. Therefore we expect to find the bifurcation here. From \eqref{eq:oneD_osc_caseboundary}, we have a range of $\mu$ when this case applies, $\mu<\frac{2|A|}{\Omega}$. It is important to note that the integrand in \eqref{eq:oneD_osc_integral} is non-trivial when $|v_0(t)|<|A|$. In order to deal with the sign changing inside the integral, we break the integration into regions based on the sign. Recall that we are searching for equilibrium behavior, and so we may make the assumption that we are dealing with a fixed value of $v_0$ such that $|v_0|\le |A|$. In figure~\ref{fig:oneD_osc_t1t2_graphic} we observe the function that we are integrating.

\begin{figure}[H]
\centering
\includegraphics[width=0.7\textwidth]{oneD/t1t2_graphic.jpg}
\caption{The non-smooth function $|y_0(T)|=|v_0-A\cos(T)|$ that we integrate is shown as a solid red line. We also show an example of $v_0$ as a horizontal blue dotted line. Here the value of $|v_0|\le|A|$, which causes kinks to appear at the roots of $|y_0|$: $T_1$ and $T_2$ respectively. These are the vertical black dashed dotted lines. }
\label{fig:oneD_osc_t1t2_graphic}
\end{figure}

\indent From figure~\ref{fig:oneD_osc_t1t2_graphic}, the roots of the integrand are 

\begin{equation*}
T_1=\arccos (v_0/A),\qquad T_2= 2\pi - \arccos (v_0/A).
\end{equation*}

Here we notice $0<T_1<T_2<2\pi$ and that the sign of the integrand stays the same on each interval. We only assume that the first interval $[0,T_1]$  observes a solution while the center is still negative, thus the integrand will also be negative. From this, the integral in \eqref{eq:oneD_osc_integral} is computed as

\begin{equation}\label{eq:oneD_osc_expandedintegral}
\begin{aligned}
\int_0^{2\pi}|v_0-A\cos(T)|\,dT=&-\int_0^{T_1}(v_0-A\cos(T))\,dT+\\
&\int_{T_1}^{T_2}(v_0-A\cos(T))\,dT-\int_{T_2}^{2\pi}(v_0-A\cos(T))\,dT.
\end{aligned}
\end{equation}

Evaluating \eqref{eq:oneD_osc_expandedintegral} and using a trigonometric identity, $\sin(\arccos(x))=\sqrt{1-x^2}$, we find the integral to be

\begin{equation*}
\int_0^{2\pi}|v_0-A\cos(T)|\,dT=\frac{2}{\pi}\left(\arcsin(v_0/A)v_0+\sqrt{A^2-v_0^2}\right).
\end{equation*}

Notice that our argument above is simple for fixed $v_0$, but we have used a multiple scales approach for our fast time with $t\ll T$ implying that $v_0(t)$ is approximately fixed over $T\in [0,2\pi]$. This holds true due to having a high frequency $\Omega$ and otherwise would not be a valid approximation. Thus we can evaluate \eqref{eq:oneD_osc_integral} to find the inner equation 

\begin{equation}\label{eq:oneD_osc_caseIIexact}
{v_0}_t=-m+\frac{4}{\pi}\left(\arcsin(v_0/A)v_0+\sqrt{A^2-v_0^2}\right).
\end{equation}

\indent In its current form, \eqref{eq:oneD_osc_caseIIexact} prevents $v_0(t)$ to be found analytically, so we use a quadratic Taylor approximation to be able to solve this equation explicitly. This then gives

\begin{equation}\label{eq:oneD_osc_caseIItaylor}
{v_0}_t \approx -m + \frac{4|A|}{\pi} + \frac{2}{\pi |A|}v_0^2,
\end{equation}

which has the following equilibrium with positive constant $C$

\begin{equation}\label{eq:oneD_osc_caseIIequil}
v_0=-C\sqrt{m-\frac{4|A|}{\pi}}.
\end{equation}

Thus we have the leading order inner equilibrium \eqref{eq:oneD_osc_caseIIequil} and writing this in the original variables gives

\begin{equation}\label{eq:oneD_osc_innersoln}
\begin{aligned}
y\sim& -C\sqrt{m-\frac{4|A|}{\pi}}-A\cos(T)+O(\Omega^{-1}),\\ 
x(t)\sim& -C\sqrt{\Omega \left(\mu-\frac{4|A|}{\pi \Omega}\right)}-\Omega^{-1} A\cos(\Omega t)+O(\Omega^{-2}).
\end{aligned}
\end{equation}

It then is clear that the bifurcation, $\mu_{\text{osc}}$, occurs when \eqref{eq:oneD_osc_innersoln} fails to be real valued. Thus we find $\mu_{\text{osc}}$ to take the form

\begin{equation}\label{eq:oneD_osc_bif}
\mu_{\text{osc}}=\frac{4|A|}{\pi \Omega}.
\end{equation}

\indent From the result \eqref{eq:oneD_osc_bif}, we gather that the oscillatory forcing in the system causes the bifurcation to occur sooner, $\mu_{\text{osc}}>\mu_{\text{ns}}$, and this is controlled by the size of $A$ and $\Omega$. Heuristically, the model experiences the non-smooth behavior sooner in $\mu$ with the oscillations, but we can see as $\Omega\to\infty$ then $\mu_{\text{osc}}\to\mu_{\text{ns}}$. This effect is contrary to the slow variation where the solution experienced a delayed tipping, $\mu_{\text{slow}}<\mu_{\text{ns}}$. Although our assumption that $\Omega \gg 1$ must be met for our analysis to hold, we may still see a medium-range $\Omega$. From figure~\ref{fig:oneD_osc_numerics} (d) we may have $\Omega=5$ and get reasonable approximations with $\mu_{\text{osc}}$. This means that our approximation $\mu_{\text{osc}}$ improves with increasing $\Omega$, but it still provides a reasonable approximation for bifurcations away from $\mu_{\text{ns}}$ for a range of $\Omega$. Advanced bifurcation also indicates that the region of bi-stability is shrunk with oscillatory forcing and thus can be used to eliminate the region entirely with $A$ and $\Omega$ chosen properly, effectively destroying any hysteresis. We compare our estimate to numerical results for varying sizes of $\Omega^{-1}$.

\begin{figure}[H]
\centering
\begin{subfigure}{.5\textwidth}
 \centering
 \includegraphics[width=\linewidth]{oneD/osc_timeseries.jpg}
 \caption{}
\end{subfigure}%
\begin{subfigure}{.5\textwidth}
 \centering
 \includegraphics[width=\linewidth]{oneD/osc_bif_diagram.jpg}
 \caption{}
\end{subfigure}
\begin{subfigure}{.5\textwidth}
 \centering
 \includegraphics[width=\linewidth]{oneD/osc_bif_diagram_zoom.jpg}
 \caption{}
\end{subfigure}%
\begin{subfigure}{.5\textwidth}
\centering
\includegraphics[width=\linewidth]{oneD/osc_Omegacomp.jpg}
\caption{}
\label{fig:oneD_osc_comp}
\end{subfigure}
\caption{In (a) the numerical time series solutions to \eqref{eq:oneD_canonical} are given from bottom to top with $\mu=\{.8,.33,.15\}$ in case I, case II and $\mu<\mu_{\text{osc}}$ in \eqref{eq:oneD_osc_bif} respectively with $A=2$, $\Omega=10$ and $\epsilon=0$. In (b) we show the time series on the bifurcation diagram. In (c), a zoom in closer to the non-smooth bifurcation of (b), where the dotted vertical lines dictate the region between case I and case II (green) as well as the bifurcation estimate (blue) respectively. In (d) a range of $\Omega^{-1}$ and the corresponding numerical bifurcations (red stars) are compared to our estimate of the bifurcations (black solid line). We consider the bifurcation criterion to be when the numerical solution has passed $x>.5$.}
\label{fig:oneD_osc_numerics}
\end{figure}

\indent In figure~\ref{fig:oneD_osc_numerics} an example is given of the effect oscillatory forcing has on the solution given a choice of $A$ and $\Omega$ with (a), (b) and (c), but (d) shows the bifurcation approximation across a range of $\Omega^{-1}$. There is an allowed range of $\Omega$ from our assumption of $\Omega \gg 1$ and in this region we see agreement between the numerical results and our approximations. The concavity is well represented and the behavior as $\Omega^{-1}\to 0$ converges to the static bifurcation. Thus we expect that our methodology is valuable for the Stommel model.

\subsection{Stability}

Once more, the outer solution \eqref{eq:oneD_osc_outersoln} is stable from the static model in \autoref{sec:oneD_static}. In this section, we have two regions of interest and establish their stabilities agree with our analysis. Each region has a particular version of the same inner equation dictating the solution's behavior, namely

\begin{equation}\label{eq:oneD_osc_stabilityI}
{v_0}_t=-m+\frac{1}{\pi}\int_0^{2\pi}|v_0-A\cos(T)|\,dT.
\end{equation}

\subsubsection{Case I: $v_0(t)< -|A|$}

In this region we did not find any bifurcation behavior and \eqref{eq:oneD_osc_stabilityI} simplifies to the inner equation with equilibrium $z^0$ as follows

\begin{equation*}
{v_0}_t=-m-2v_0=f(v_0), \quad z^0=-\frac{m}{2}.
\end{equation*}

Similarly to \autoref{sec:oneD_slow}, we have a fixed parameter equation that we have shown to cause perturbations to decay exponentially and hence we find the equilibrium to be hyperbolic and asymptotically stable in the paramter range found in the analysis \eqref{eq:oneD_osc_caseboundary}

\begin{equation*}
\mu \ge \frac{2|A|}{\Omega}.
\end{equation*}

\subsubsection{Case II: $|v_0(t)|<|A|$}

For this region we found the bifurcation and hence we should expect to lose stability here. The Taylor approximation \eqref{eq:oneD_osc_caseIItaylor} for the inner equation with equilibrium $z^0$ is

\begin{equation}\label{eq:oneD_osc_stabilityII}
{v_0}_t= -m +\frac{4|A|}{\pi}+\frac{2}{\pi |A|}v_0^2,\quad z^0=-C \sqrt{m-\frac{4|A|}{\pi}}.
\end{equation}

We consider a simple linear perturbation of \eqref{eq:oneD_osc_stabilityII}, $v_0(t)=z^0+u(t)$ with $\lVert u(t) \rVert \ll 1$. Applying the standard Taylor expansion to determine the equation for the perturbations, we find

\begin{equation}\label{eq:oneD_osc_innerpertubation}
\begin{aligned}
{v_0}_t =& f(z^0)+f_{v_0}(z^0)(v_0-z^0)+ O(\lVert v_0-z^0\rVert^2),\\ 
u_t =& -2\sqrt{m-\frac{4|A|}{\pi}} u .
\end{aligned}
\end{equation}

\indent The sign of \eqref{eq:oneD_osc_innerpertubation} gives that the perturbations decay exponentially and hence the equilibrium is hyperbolic and asymptotically stable as long as $m>\frac{4|A|}{\pi}$ or equivalently $\mu>\frac{4|A|}{\pi \Omega}$. We find that once $\mu$ reaches this value in \eqref{eq:oneD_osc_bif}, then the stability of \eqref{eq:oneD_osc_innerpertubation} is non-hyperbolic. When the stability switches like this, we expect a bifurcation. Thus we have further evidence to support that \eqref{eq:oneD_osc_bif} is the oscillatory bifurcation we seek.

\section{Slowly Varying and Oscillatory Forcing}
\label{sec:oneD_slowosc}

Now that we have established an approach for each feature of the model individually, we combine them in the full one component model \eqref{eq:oneD_canonical} where $\epsilon\ll 1$ and ${A\sim O(1)}$. Due to the slow variation in $\mu$, we do not see a bifurcation occur under these conditions but rather a tipping point. Hence we must find the behavior of the solution and search for when a rapid transition towards the upper branch occurs. Since the high frequency could be approximated by a power of slow variation, we choose to relate these mechanisms with a generic polynomial $\Omega = \epsilon^{-\lambda}$ for a parameter $\lambda>0$. With a general $\lambda$, we classify regions of behavior by ranges of $\lambda$ and are able to determine where mixed behavior occurs or when one of the mechanisms becomes dominant. With both mechanisms in effect, we again choose to use a multiple scales approach to capture both slow behavior and fast oscillations. Although now, we truly have slow behavior, the slowly varying parameter $\mu(t)$, as well as fast behavior, the rapid oscillations $\sin(\Omega t)$. The choice in time scales is then $\tau=\epsilon t$ and $T=\epsilon^{-\lambda} t$, which leads to the system

\begin{equation*}
\begin{aligned}
x_T+\epsilon^{\lambda+1}x_\tau =&\, \epsilon^\lambda (-\mu(\tau)+2|x|-x|x|+A\sin(T)),\\
\mu_\tau=&-1.
\end{aligned}
\end{equation*}

\indent Once again, we assume initial conditions satisfying $x<0$ and starting far enough away from $x=0$, before any crossing occurs, to find the outer solution. Thus we have the system

\begin{equation}\label{eq:oneD_slowosc_outereq}
\begin{aligned}
x_T+\epsilon^{\lambda+1}x_\tau =& \, \epsilon^\lambda (-\mu(\tau)-2x+x^2+A\sin(T)),\\
\mu_\tau(\tau)=&-1.
\end{aligned}
\end{equation}

We perform an asymptotic expansion in terms of the small quantity $\epsilon^\lambda$, where we note that this is the same as $\Omega^{-1}$,

\begin{equation}\label{eq:oneD_slowosc_asympexpansion}
x(\tau,T)\sim x_0(\tau,T)+\epsilon^\lambda x_1(\tau,T)+O(\epsilon^{1+\lambda},\epsilon^{2\lambda}).
\end{equation}

Introducing the expansion \eqref{eq:oneD_slowosc_asympexpansion} into the outer multi-scaled equation \eqref{eq:oneD_slowosc_outereq} gives 

\begin{equation}\label{eq:oneD_slowosc_outerexplicit}
{x_0}_T+\epsilon^{\lambda+1}{x_0}_\tau+\epsilon^\lambda {x_1}_T+\ldots=\epsilon^\lambda(-\mu(\tau)-2x_0+x_0^2+A\sin(T))+\epsilon^{2\lambda}(-2x_1+x_1x_0)+\ldots
\end{equation}

Here we separate \eqref{eq:oneD_slowosc_outerexplicit} at each order of $\epsilon^\lambda$ to find the following system of equations

\begin{align}
\label{eq:oneD_slowosc_outerO1}
O(1):\quad & {x_0}_T=0, \\
\label{eq:oneD_slowosc_outerO2}
O(\epsilon^\lambda):\quad& {x_1}_T=-\mu(\tau)-2x_0+x_0^2+A\sin(T),\\
\label{eq:oneD_slowosc_outerO3}
O(\epsilon^{2\lambda}):\quad& {x_2}_T+\epsilon^{1-\lambda}{x_0}_\tau= -2x_1+2x_0x_1.
\end{align}

Depending on the value of $\lambda$, $O(\epsilon^{\lambda+1})$ may be the next order before $O(\epsilon^{2\lambda})$, although considering either produce the same equation at their respective order and hence our choice in $\lambda$ does not change the calculations up to this correction term. Thus for the outer solution, we consider the system with $O(\epsilon^{2\lambda})$. Each equation gives the behavior of each order of the solution; \eqref{eq:oneD_slowosc_outerO1} indicates that the leading order term is only slow time dependent, $x_0=x_0(\tau)$. In \autoref{app:oneD} we apply the Fredholm alternative \eqref{eq:Fredholm} to \eqref{eq:oneD_slowosc_outerO2} and \eqref{eq:oneD_slowosc_outerO3} to find the first few terms of the expansion in \eqref{eq:oneD_slowosc_asympexpansion} explicitly. The resulting solution is

\begin{equation}\label{eq:oneD_slowosc_outersoln}
x(t)\sim 1-\sqrt{1+\mu(t)}-\frac{\epsilon}{4(1+\mu(t))}-\epsilon^\lambda A \cos(\Omega t)+O(\epsilon^{1+\lambda},\epsilon^{2\lambda}).
\end{equation}

In the outer solution \eqref{eq:oneD_slowosc_outersoln}, we consider when the terms violate the assumptions of the expansion to find where we need to use an inner equation. This happens either when $x_0\sim O(\epsilon)$ or when $x_0\sim O(\epsilon^\lambda)$ which is when $\mu\sim O(\epsilon)$ or $\mu\sim O(\epsilon^\lambda)$ respectively and depends on the value of $\lambda$.

\indent To find an inner equation we use a general scaling for both $x$ and $\mu$ given the ambiguity of the choice in $\mu$ with

\begin{equation}\label{eq:oneD_slowosc_general_scaling}
x(t)=\epsilon^\alpha y(t) ,\quad \mu(t)=\epsilon^\beta m(t),
\end{equation}

where $\alpha>0$ and $\beta>0$ allow for inner equations to be found. Applying the local variables \eqref{eq:oneD_slowosc_general_scaling} to the canonical equation \eqref{eq:oneD_canonical} gives

\begin{equation}\label{eq:oneD_slowosc_innerscaled}
\begin{aligned}
\epsilon^\alpha \dot{y}=& -\epsilon^\beta m(t)+\epsilon^\alpha 2|y| - \epsilon^{2\alpha}y|y| +A\sin(\epsilon^{-\lambda}t),\\
\dot{m}=&-\epsilon^{1-\beta}.
\end{aligned}
\end{equation}

From \eqref{eq:oneD_slowosc_innerscaled} we find the fast time still appears but the slow time has multiple choices depending on $\lambda$. For convenience we choose to take a multiple scales approach with scales $t$ and $T=\epsilon^{-\lambda}t$ in \eqref{eq:oneD_slowosc_innerscaled} to find

\begin{equation}\label{eq:oneD_slowosc_innergeneral}
\begin{aligned}
\epsilon^{\alpha-\lambda} y_T+\epsilon^{\alpha}y_t=& -\epsilon^{\beta}m(t)+\epsilon^{\alpha}2|y|-\epsilon^{2\alpha}y|y|+A\sin(T),\\
m_t=&-\epsilon^{1-\beta}.
\end{aligned}
\end{equation}

To determine the correct scalings in \eqref{eq:oneD_slowosc_general_scaling}, we balance the leading order terms on both sides of \eqref{eq:oneD_slowosc_innergeneral} $\epsilon^{\alpha-\lambda}y_T$ and $A\sin(T)$, which gives us that $\alpha=\lambda$. This suggests that the oscillatory term persists in the inner asymptotic expansion of \eqref{eq:oneD_canonical} regardless of the choice in $\lambda$.

\indent We now consider the same scales $t$ and $T=\epsilon^{-\lambda}t$ on the canonical system \eqref{eq:oneD_canonical} 

\begin{equation}\label{eq:oneD_slowosc_general_outermulti}
\begin{aligned}
x_T+\epsilon^{\lambda}x_t =& -\epsilon^{\lambda+\beta}m(t)+\epsilon^{\lambda}2|x|-\epsilon^{\lambda}x|x|+\epsilon^{\lambda}A\sin(T),\\
m_t =&-\epsilon^{1-\beta}.
\end{aligned}
\end{equation}

Here we use the expansion 

\begin{equation*}
x(t,T) = \epsilon^{\lambda}y_0(t,T) +\ldots ,
\end{equation*}

where the next terms of this expansion depend on whether $\lambda\le1$ or $\lambda> 1$. We consider these ranges in case I and case II respectively.

\subsection{Case I: $\lambda \le 1$}
\label{subsec:oneD_slowosc_caseI}

We call this the 'mixed effects' case a both slow variation and oscillatory forcing causing noticeable effects on the solution for this range of $\lambda$. Hence we consider the expansion

\begin{equation}\label{eq:oneD_slowosc_caseI_expansion}
x(t,T)\sim \epsilon^{\lambda} y_0(t,T)+\epsilon^q y_1(t,T)+\ldots,
\end{equation}

with the next term $q>\lambda$ to be consistent with the scale analysis above that determined inner behavior to start at $O(\epsilon^\lambda)$. Substituting \eqref{eq:oneD_slowosc_caseI_expansion} into \eqref{eq:oneD_slowosc_general_outermulti} gives

\begin{equation*}
\begin{aligned}
{y_0}_T+\epsilon^{\lambda}{y_0}_t+\epsilon^{q-\lambda} {y_1}_T+\epsilon^{q} {y_1}_t+\ldots={} & -\epsilon^{\beta}m(t)+\epsilon^\lambda 2|y_0 +\epsilon^{q-\lambda} y_1+\ldots|+ A\sin(T) \\
& + \epsilon^{2\lambda}( y_0 +\epsilon^{q-\lambda} y_1+\ldots)|y_0 +\epsilon^{q-\lambda} y_1+\ldots |.
\end{aligned}
\end{equation*}

Separating by distinct orders of $\epsilon$ then gives the following equations at each order

\begin{align} \label{eq:oneD_slowosc_caseI_O1}
O(1): & \,{y_0}_T = A\sin(T),\\ \label{eq:oneD_slowosc_caseI_O2}
O(\epsilon^\lambda):  &\, \epsilon^{q-2\lambda}{y_1}_T+{y_0}_t=-\epsilon^{\beta-\lambda}m(s)+2|y_0|.
\end{align}

\indent In \eqref{eq:oneD_slowosc_caseI_O2} we find that the appropriate next term in the expansion \eqref{eq:oneD_slowosc_caseI_expansion} is with $q=2\lambda$. This choice in $q$ keeps the equations balanced but $q$ implies that $\lambda> \frac{1}{2}$ for an expansion to be found. Otherwise, the quadratic terms must be included and we no longer find local equations. This indicates that the range of $\lambda\le\frac{1}{2}$ behaves differently. We discuss this further in \autoref{chap:conclusion}. There is also the choice between $\beta=\lambda$ or $\beta=1$ and each has a particular appeal. With $\beta=\lambda$, the form of \eqref{eq:oneD_slowosc_caseI_O2} is simple, but the equation for the slow variation is $m_t=-\epsilon^{1-\lambda}$. This then suggests a slower time scale to approach the problem. We instead choose to allow $\beta=1$ for convenience and track a small coefficient on $m(t)$ in exchange for keeping the same time scale with $m_t=-1$. This is valid as long as we are tracking small coefficients and not large ones as this would suggest we used an incorrect scaling, but both of these choices lead to the same conclusion. Using \eqref{eq:oneD_slowosc_caseI_O1} gives the appropriate form, $y_0(t,T)=v_0(t)-A\cos(T)$. We then apply the Fredholm alternative \eqref{eq:Fredholm} to \eqref{eq:oneD_slowosc_caseI_O2} which gives a similar equation to the integral \eqref{eq:oneD_osc_integral} in \autoref{sec:oneD_highfreqosc} with

\begin{equation}\label{eq:oneD_slowosc_caseIintegral}
{v_0}_t = -\epsilon^{1-\lambda}m(t)+\frac{1}{\pi}\int_0^{2\pi} |v_0(t)-A\cos(T)|\,dT.
\end{equation}

The approach developed in \autoref{sec:oneD_highfreqosc} is applied here to \eqref{eq:oneD_slowosc_caseIintegral}, where we separate the behavior of the integral based on the relative size of $v_0(t)$ to $A$. We have the following situations, sub-case I: $v_0(t)\le -|A|$ and sub-case II: $|v_0(t)|<|A|$.

\subsubsection{Sub-Case I: $v_0(t) \le -|A|$} 
\label{subsubsec:oneD_slowosc_subcaseI}

Once more, we call this the 'below axis' sub-case and we do not expect tipping to occur under these conditions since the solution is entirely negative and far from the axis $x=0$ for most of the oscillation. Under these conditions, \eqref{eq:oneD_slowosc_caseIintegral} gives the simple inner equation

\begin{equation}\label{eq:oneD_slowosc_innersubcaseI}
{v_0}_t= -\epsilon^{1-\lambda}m(t)-2v_0.
\end{equation}

Solving \eqref{eq:oneD_slowosc_innersubcaseI} can be done under our assumptions much like in \autoref{subsec:oneD_osc_CaseI} but instead we focus on the pseudo-equilibrium. This choice results in finding the effective parameter range for $\mu$ which distinguishes these sub-cases and helps to determine when the solution enters sub-case II. Since $m(t)$ is allowed to vary, this must be thought of more as a pseudo-equilibrium and we are only interested in when the pseudo-equilibrium violates the assumptions of this case. Finding the pseudo-equilibrium of \eqref{eq:oneD_slowosc_innersubcaseI} gives

\begin{equation*}
v_0(t)=-\epsilon^{1-\lambda}\frac{m(t)}{2}.
\end{equation*}

Using the condition $v_0(t)\le -|A|$ gives that $m(t)\ge \epsilon^{\lambda-1}2|A|$. Writing this result in original variables gives us the parameter range

\begin{equation}\label{eq:oneD_slowosc_subcaseboundary}
\mu(t)\ge \frac{2 |A|}{\Omega},
\end{equation}

for sub-case I which agrees with the range from \eqref{eq:oneD_osc_caseboundary} in \autoref{sec:oneD_highfreqosc}. Following the pseudo-equilibrium to the boundary \eqref{eq:oneD_slowosc_subcaseboundary}, we eventually reach sub-case II where we see the oscillations crossing the axis.

\subsubsection{Sub-Case II: $|v_0(t)|< |A|$}
\label{subsubsec:oneD_slowosc_subcaseII}

Again, we call this the 'crossing' sub-case. Here the behavior of the solution depends strongly on the sign of the solution similarly to \autoref{sec:oneD_highfreqosc}. We seek the relationship between slow variation and oscillatory forcing on the tipping point. As the pseudo-equilibrium gets closer to the $x=0$ axis, the solution spends more time above this axis and more complicated contributions from the sign changing appear. We expect tipping to happen under these conditions.

\indent The methodology of solving the integral in \eqref{eq:oneD_slowosc_caseIintegral} holds identically to that of \autoref{subsec:oneD_osc_CaseII}. Here, we have a slow time function $v_0(t)$ that is approximately fixed with respect to the fast time $T$ under the multiple scales approach. Thus we evaluate the integral by separating the sign of the integrand with the values $T_1=\arccos(v_0/A)$ and $T_2 = 2\pi-\arccos(v_0/A)$ to find

\begin{equation}\label{eq:oneD_slowosc_subcaseIIexact}
{v_0}_t=-\epsilon^{1-\lambda}m(t)+\frac{4}{\pi}\left(\arcsin(v_0/A)v_0+\sqrt{A^2-v_0^2}\right).
\end{equation}

We then choose to find an explicit analytic expression by approximating \eqref{eq:oneD_slowosc_subcaseIIexact} with a quadratic Taylor expansion. This gives

\begin{equation}\label{eq:oneD_slowosc_subcaseIItaylor}
\begin{aligned}
{v_0}_t =& -\epsilon^{1-\lambda}m(t) + \frac{4|A|}{\pi} + \frac{2}{\pi |A|}v_0^2,\\
m_t =& -1.
\end{aligned}
\end{equation}

With \eqref{eq:oneD_slowosc_subcaseIItaylor} in terms of slow time, it restricts any analytical approaches that link the effects of the varying parameter. Instead we take the same approach from \cite{haberman1979slowly} and switch the differentiation onto the slow varying parameter $m$ with

\begin{equation}\label{eq:oneD_slowosc_subcaseIItaylorm}
{v_0}_m = \epsilon^{1-\lambda}m - \frac{4|A|}{\pi} - \frac{2}{\pi |A|}v_0^2.
\end{equation}

It is here where we take advantage of the form of \eqref{eq:oneD_slowosc_subcaseIItaylorm} with the form from \eqref{eq:intro_Zhuairy} to solve, resulting in

\begin{equation*}
v_0(m)\sim \epsilon^{(1-\lambda)/3}\left( \frac{\pi |A|}{2} \right)^{2/3}\frac{Ai'\left(\epsilon^{2(\lambda-1)/3}\left(\frac{2}{\pi |A|}\right)^{1/3}(\epsilon^{1-\lambda}m-\frac{4|A|}{\pi})\right)}{Ai\left(\epsilon^{2(\lambda-1)/3}\left(\frac{2}{\pi |A|}\right)^{1/3}(\epsilon^{1-\lambda}m-\frac{4|A|}{\pi})\right)}.
\end{equation*}

With the solution to \eqref{eq:oneD_slowosc_caseI_expansion} we rewrite back into the original variables

\begin{equation}\label{eq:oneD_slowosc_caseIsoln}
\begin{aligned}
y_0(t,T)\sim& C\frac{Ai'\left(\epsilon^{2(\lambda-1)/3}\left(\frac{2}{\pi |A|}\right)^{1/3}(\epsilon^{1-\lambda}m(t)-\frac{4|A|}{\pi})\right)}{Ai\left(\epsilon^{2(\lambda-1)/3}\left(\frac{2}{\pi |A|}\right)^{1/3}(m(t)-\frac{4|A|}{\pi})\right)}-\epsilon^\lambda A\cos(T)+\ldots,\\
x(t) \sim& C\frac{Ai'\left(\left(\frac{\Omega}{\epsilon^2}\right)^{1/3}\left(\frac{2}{\pi |A|}\right)^{1/3}(\mu(t)-\frac{4|A|}{\pi \Omega})\right)}{Ai\left(\left(\frac{\Omega}{\epsilon^2}\right)^{1/3}\left(\frac{2}{\pi |A|}\right)^{1/3}(\mu(t)-\frac{4|A|}{\pi \Omega})\right)}-\epsilon^\lambda A\cos(\Omega t) +\ldots
\end{aligned}
\end{equation}

\indent Given the inner solution \eqref{eq:oneD_slowosc_caseIsoln}, we search for the singularity of this solution in order to identify tipping. Recall from \eqref{eq:intro_Zhuresult} that the singularity relates to the first root of the Airy equation. Here we find the singularity $\mu_{\text{mixed}}$ to be

\begin{equation}\label{eq:oneD_slowosc_uglymu}
\mu_{\text{mixed}}=\left(\frac{\epsilon^2}{\Omega}\right)^{1/3}\left(\frac{\pi |A|}{2}\right)^{1/3}(-2.33811\ldots)+\frac{4|A|}{\pi \Omega}.
\end{equation}

The value $\mu_{\text{mixed}}$ which causes this singularity is our tipping point. We rewrite \eqref{eq:oneD_slowosc_uglymu} to emphasize the contributions from the slow variation of the parameter and the oscillatory forcing

\begin{equation}\label{eq:oneD_slowosc_caseItipping}
\mu_{\text{mixed}} = \left(\frac{\pi |A|}{2\Omega}\right)^{1/3} \mu_{\text{smooth}}+\mu_{\text{osc}},
\end{equation}

with $\mu_{\text{smooth}}=\epsilon^{2/3}\left(-2.33811\ldots\right)$, similarly to the smooth problem from \cite{zhu2015tipping}, and $\mu_{\text{osc}}$ from \eqref{eq:oneD_osc_bif} respectively.

\indent The resulting tipping approximation \eqref{eq:oneD_slowosc_caseItipping} indicates that the size of the amplitude $A$ determines whether the tipping occurs early or late relative to the bifurcation. Naturally we see a larger amplitude cause more contribution from the oscillations and hence an earlier tipping. On the other hand, larger values in $\epsilon$ cause this tipping to occur later. So these effects have opposite pulls on the tipping and can effectively cancel one another out under proper conditions. It would even be possible to break the hysteresis cycle by eliminating the region of bi-stability in this model with sufficiently large amplitude and small $\epsilon$. The tipping point holds for any $\lambda\in (\frac{1}{2},1]$ and we see different behavior for larger $\lambda$.

\subsection{Case II: $\lambda>1$}
\label{subsec:oneD_slowosc_caseII}

We call this the 'slowly varying dominant' case as this is when we see that the oscillations contribute less than the slow variation. For this range of $\lambda$ the scaling for $\mu$ is simple, $\mu=\epsilon m$. Thus we expect to see integer powers in the leading order along with powers of $\lambda$ so we choose the expansion

\begin{equation}\label{eq:oneD_slowosc_caseII_expansion}
x(t,T) \sim \epsilon^\lambda y_0(t,T)+\epsilon y_1(t,T)+\epsilon^q y_2(t,T)+\ldots,
\end{equation}


where $q>\lambda$ to allow for consistency with the scale analysis but not necessarily the same value as in case I. Substituting \eqref{eq:oneD_slowosc_caseII_expansion} into \eqref{eq:oneD_slowosc_general_outermulti} gives

\begin{equation*}
\begin{aligned}
\epsilon {y_0}_T+\epsilon^{\lambda+1} {y_0}_t+\epsilon^\lambda {y_1}_T+\epsilon^q {y_2}_T+\ldots=&-\epsilon^{\lambda+1}m(t)+\epsilon^{\lambda+1} 2|y_0+\epsilon^{\lambda-1} y_1 +\ldots|\\
&+ \epsilon^{\lambda+2}( y_0+\epsilon^{\lambda-1} y_1 +\ldots)| y_0+\epsilon^{\lambda-1} y_1 +\ldots|\\
&+\epsilon^\lambda A\sin(T) 
\end{aligned}
\end{equation*}

Here we separate out each order of $\epsilon$ to find the equations at each order

\begin{align} \label{eq:oneD_slowosc_caseII_O1}
O(\epsilon): &\, {y_0}_T=0,\\ \label{eq:oneD_slowosc_caseII_O2}
O(\epsilon^\lambda):  &\, {y_1}_T = A\sin(T),\\ \label{eq:oneD_slowosc_caseII_O3}
O(\epsilon^{\lambda+1}): &\, \epsilon^{q-\lambda-1}{y_2}_T+ {y_0}_t = -m(t) +2|y_0+\epsilon^{\lambda-1}y_1|.
\end{align}

We learn in \eqref{eq:oneD_slowosc_caseII_O3} that $q=\lambda+1$ keeps the terms balanced. From \eqref{eq:oneD_slowosc_caseII_O1} we find that the dominant behavior for this case is only slow time dependent, $y_0=y_0(t)$ and from \eqref{eq:oneD_slowosc_caseII_O2} that the oscillatory behavior occurs in $y_1$ with $y_1(t,T)=v_1(t)-A\cos(T)$. Since we have $y_1$ as a correction to $y_0$, we may absorb the slow behavior into $y_0$. Thus we treat $y_0(t)=y_0(t)+\epsilon^{\lambda+1} v_1(t)\approx y_0(t)$. Applying Fredholm to \eqref{eq:oneD_slowosc_caseII_O3} gives 

\begin{equation}\label{eq:oneD_slowosc_caseII_integral}
\begin{aligned}
{y_0}_t=& -m(t)+\frac{1}{\pi}\int_0^{2\pi}|y_0(t)-\epsilon^{\lambda-1}A\cos(T)|\,dT.
\end{aligned}
\end{equation}

\indent With $\lambda\approx1$, we see nearly identical behavior in \eqref{eq:oneD_slowosc_caseII_integral} as that of what we explored in \autoref{subsec:oneD_slowosc_caseI}. As long as the amplitude of oscillations inside the integral are $\epsilon^{\lambda-1}A\sim O(1)$, then this integral is similar to the integral in Case I \eqref{eq:oneD_slowosc_caseIintegral}. To see this, we follow the same approach as to integrate \eqref{eq:oneD_slowosc_caseII_integral} with $T_1=\arccos(y_0/\epsilon^{\lambda-1}A)$ and $T_2=2\pi- \arccos(y_0/\epsilon^{\lambda-1}A)$ which gives

\begin{equation}\label{eq:oneD_slowosc_caseIIexact}
{y_0}_t=-m(t)+\frac{4}{\pi}\left(\arcsin(y_0/\epsilon^{\lambda-1}A)y_0+\sqrt{(\epsilon^{\lambda-1}A)^2-y_0^2}\right).
\end{equation}

Here we apply the same quadratic Taylor approximation to \eqref{eq:oneD_slowosc_caseIIexact} to find

\begin{equation}\label{eq:oneD_slowosc_caseII_taylor}
\begin{aligned}
{y_0}_t=&-m(t)+\epsilon^{\lambda-1}\frac{2|A|}{\pi}+\epsilon^{1-\lambda}\frac{2}{\pi |A|}y_0^2,\\
m=&-1.
\end{aligned}
\end{equation}

We again use the result from \eqref{eq:intro_Zhuresult} to find the tipping, which we then write into original variables

\begin{equation*}
\begin{aligned}
m_{\text{mixed}}=&\epsilon^{(\lambda-1)/3}\left(\frac{\pi |A|}{2}\right)^{1/3}(-2.33811\ldots)+\epsilon^{\lambda-1}\frac{4|A|}{\pi},\\
\mu_{\text{mixed}}=&\left(\frac{\pi |A|}{2\Omega}\right)^{1/3} \mu_{\text{smooth}}+\mu_{\text{osc}}.
\end{aligned}
\end{equation*}

\indent Thus we conclude that there is a natural transition into case II from case I with almost the same behavior and identical tipping as in \eqref{eq:oneD_slowosc_caseItipping}. As $\lambda$ continues to grow, the amplitude of oscillation in \eqref{eq:oneD_slowosc_caseII_integral} decays and the contribution from the oscillations weaken. This allows us to say that the integral is approaching

\begin{equation}\label{eq:oneD_slowosc_caseII_inner}
{y_0}_t = -m(t) +2|y_0|.
\end{equation}

With \eqref{eq:oneD_slowosc_caseII_inner} taking the same form as in \autoref{sec:oneD_slow}, this allows us to use the results there to find the solution. We write this in terms of the original variables

\begin{equation}\label{eq:oneD_slowosc_caseIIsoln}
\begin{aligned}
y_0(t,T)\sim& C e^{-2m(t)}+\frac{m(t)}{2}-1/4 +\epsilon^\lambda A\cos(T),\\
x(t)\sim& C e^{-2\mu(t)/\epsilon}+\frac{\mu(t)}{2} -\epsilon^\lambda A\cos(\Omega t)+O(\epsilon^{2\lambda}).
\end{aligned}
\end{equation}

This then leads to the same tipping as in the slowly varying model with 

\begin{equation*}
\mu_{\text{slow}}=\frac{1}{2}\epsilon\log\epsilon.
\end{equation*}

Thus we find that in this case and with $\lambda$ near 1, the same tipping point $\mu_{\text{mixed}}$ in \eqref{eq:oneD_slowosc_caseItipping} from \autoref{subsec:oneD_slowosc_caseI} is found. For large $\lambda$, the oscillations have less of an impact and the solution tips entirely like $\mu_{\text{slow}}$ in \eqref{eq:oneD_slow_tipping} from \autoref{sec:oneD_slow}. For convenience, this is summarized in the following table.

\begin{center}
\begin{table}[H]\label{table:oneD_tipping}
\centering
\begin{tabular}{|c|c|}
\hline 
 \multicolumn{2}{|c|}{One Component Tipping Points} \\ 
\hline
$\epsilon>0$ and $A=0$: & $\mu_{\text{slow}}=\epsilon\ln(\epsilon)/2$ \\ 
\hline 
$\epsilon=0$ and $A\not=0$ with $\Omega\gg1$: & $\mu_{\text{osc}}=\frac{4|A|}{\pi \Omega}$\\ 
\hline 
$\epsilon>0$, $A\not=0$ and $\lambda\le 1$: & $\mu_{\text{mixed}}=\left(\frac{\pi |A|}{2\Omega}\right)^{1/3} \mu_{\text{smooth}}+\mu_{\text{osc}}$ \\ 
\hline 
$\epsilon>0$, $A\not=0$ and $\lambda> 1$ with $\lambda\approx 1$: & $\mu_{\text{mixed}}=\left(\frac{\pi |A|}{2\Omega}\right)^{1/3} \mu_{\text{smooth}}+\mu_{\text{osc}}$ \\
\hline
$\epsilon>0$, $A\not=0$ and $\lambda>1$: & $ \mu_{\text{slow}}=\epsilon\ln(\epsilon)/2$\\
\hline
\end{tabular} 
\caption{Overview of the tipping points of the one component model for each mechanism and case.}
\end{table}
\end{center}

\indent In figure~\ref{fig:oneD_slowosc_numerical_small}, we see an example of the numerical solution to the canonical system \eqref{eq:oneD_canonical} with slow variation and oscillatory forcing. This example has a tipping point occurring in case I due to $\lambda\in (\frac{1}{2},1]$ allowing the slow variation and oscillatory forcing to produce a mixed effect on the tipping point. Although we see noticeable contributions from the slow varying parameter the tipping point still occurs near the oscillatory bifurcation. This tells us that for these choices in the values the strongest effect is the oscillatory forcing. It is possible to find values of $\epsilon$, $A$ and $\lambda$ that cause the non-smooth tipping to occur at the same place as the smooth bifurcation. This would eliminate the region of bi-stability and destroy the hysteresis curve entirely for this model.

\begin{figure}[H]
\centering
\begin{subfigure}{.5\textwidth}
 \centering
 \includegraphics[width=\linewidth]{oneD/slowosc_bif_diagram_small.jpg}
 \caption{}
\end{subfigure}%
\begin{subfigure}{.5\textwidth}
 \centering
 \includegraphics[width=\linewidth]{oneD/slowosc_bif_diagram_small_zoom.jpg}
 \caption{}
\end{subfigure}
\caption{On the left, one can see the bifurcation diagram for the canonical system \eqref{eq:oneD_canonical} with the numerical solution (black dotted line). On the right, a zoom in around the non-smooth bifurcation. The dotted vertical line is the tipping point $\mu_{\text{mixed}}$ \eqref{eq:oneD_slowosc_uglymu} (blue). The vertical line (black) is the when the numerical solution has passed the tipping criterion $x>.5$. The parameter values are $\epsilon=.05$, $\lambda=.8$ and $A=4$.}
\label{fig:oneD_slowosc_numerical_small}
\end{figure}

\indent In figure~\ref{fig:oneD_slowosc_numerical_medium}, we see an example of $\lambda$ falling into case II yet close enough to 1 that we see mixed behavior in the tipping. Here the slow variation is dominant and the oscillations are only noticeable in the zoom in. The green dotted line is the tipping point approximation \eqref{eq:oneD_slow_tipping} from \autoref{sec:oneD_slow} and $\mu_{\text{mixed}}$ is still approximating the tipping point well.

\begin{figure}[H]
\centering
\begin{subfigure}{.5\textwidth}
 \centering
 \includegraphics[width=\linewidth]{oneD/slowosc_bif_diagram_medium.jpg}
 \caption{}
\end{subfigure}%
\begin{subfigure}{.5\textwidth}
 \centering
 \includegraphics[width=\linewidth]{oneD/slowosc_bif_diagram_medium_zoom.jpg}
 \caption{}
\end{subfigure}
\caption{On the left, one can see the bifurcation diagram for the canonical system \eqref{eq:oneD_canonical} with the numerical solution (black dotted line). On the right, a zoom in about the non-smooth bifurcation. The dotted vertical lines are the tipping point $\mu_{\text{mixed}}$ \eqref{eq:oneD_slowosc_uglymu} (blue) and slowly varying tipping $\mu_{\text{slow}}$ \eqref{eq:oneD_slow_tipping} (green). The vertical line (black) is the when the numerical solution has passed the tipping criterion $x>.5$. The parameter values are $\epsilon=.05$, $\lambda=1.05$ and $A=4$.}
\label{fig:oneD_slowosc_numerical_medium}
\end{figure}

\indent In figure~\ref{fig:oneD_slowosc_numerical_large}, we see an example of $\lambda$ falling into case II but large enough that we see almost entirely slow behavior in the tipping. Even upon closer inspection it is hardly noticeable that oscillations are present in the model. The green dotted line is the tipping approximation \eqref{eq:oneD_slow_tipping} from \autoref{sec:oneD_slow}, and it is clear that this is a better approximation than the mixed tipping point.

\begin{figure}[H]
\centering
\begin{subfigure}{.5\textwidth}
 \centering
 \includegraphics[width=\linewidth]{oneD/slowosc_bif_diagram_large.jpg}
 \caption{}
\end{subfigure}%
\begin{subfigure}{.5\textwidth}
 \centering
 \includegraphics[width=\linewidth]{oneD/slowosc_bif_diagram_large_zoom.jpg}
 \caption{}
\end{subfigure}
\caption{On the left, one can see the bifurcation diagram for the canonical system \eqref{eq:oneD_canonical} with the numerical solution(black dotted line). On the right, a zoom in around the non-smooth bifurcation. The dotted vertical lines are the tipping point $\mu_{\text{mixed}}$ \eqref{eq:oneD_slowosc_uglymu} (blue) and slowly varying tipping $\mu_{\text{slow}}$ \eqref{eq:oneD_slow_tipping} (green). The vertical line (black) is the when the numerical solution has passed the tipping criterion $x>.5$. The parameter values are $\epsilon=.05$, $\lambda=1.6$ and $A=4$.}
\label{fig:oneD_slowosc_numerical_large}
\end{figure}

\indent In figure~\ref{fig:oneD_slowosc_lambdacomp} we compare the tipping between case I and case II with the numerical tipping. For smaller $\lambda$, the frequency $\Omega$ gets smaller and the case I tipping becomes more predominant. For the analysis performed in this section, $\Omega\gg 1$ and for $\lambda\le\frac{1}{2}$ we have $\Omega\sim O(1)$. We do not consider a low frequency corresponding to $\lambda\le\frac{1}{2}$ in this thesis. The larger $\lambda$ becomes, the less effect we see due to the oscillatory forcing until it is negligible for some $\lambda>1$. This is also seen in the asymptotic solution for each case, \eqref{eq:oneD_slowosc_outersoln}, \eqref{eq:oneD_slowosc_caseIsoln}, and \eqref{eq:oneD_slowosc_caseIIsoln}, where the oscillatory component of the term has a $\epsilon^\lambda$ coefficient and shrinks the effects as $\lambda$ grows.

\begin{figure}[H]
\centering
\includegraphics[width=0.7\textwidth]{oneD/slowosc_lambdacomp.jpg}
\caption{An example of numerical tipping (red stars) as the numerical solution to \eqref{eq:oneD_canonical} passes the tipping criterion $x=.5$ for the last time. The parameter values are $\epsilon=.01$ and $A=4$. The lines are the case I tipping estimate \eqref{eq:oneD_slowosc_uglymu} (black solid line) and the case II tipping estimate \eqref{eq:oneD_slow_tipping} (blue dotted line).}
\label{fig:oneD_slowosc_lambdacomp}
\end{figure} 

\indent The performance of our estimates are seen in figure~\ref{fig:oneD_slowosc_epscomp}. For case I tipping, the range of appropriate $\epsilon$ is highly dependent on the choice in $\lambda$. Often, the range is quite small for accurate estimates, but with this in mind both case approximations have good performance over a reasonable range of $\epsilon$.


\begin{figure}[H]
\centering
\begin{subfigure}{.5\textwidth}
 \centering
 \includegraphics[width=\linewidth]{oneD/slowosc_epscomp_case2.jpg}
 \caption{$\lambda=.8$}
\end{subfigure}%
\begin{subfigure}{.5\textwidth}
 \centering
 \includegraphics[width=\linewidth]{oneD/slowosc_epscomp_case3.jpg}
 \caption{$\lambda=1.3$}
\end{subfigure}
\caption{The numerical tipping (red stars) follows the appropriate case depending on $\lambda$. The case I tipping estimate (black solid line) and the case II tipping estimate (blue dotted line) are shown.}
\label{fig:oneD_slowosc_epscomp}
\end{figure}

\subsection{Stability}

Similarly to \autoref{sec:oneD_highfreqosc}, there are two ranges for $\lambda$ that govern the stability of solutions found in our analysis, namely $\lambda\le 1$ and $\lambda>1$.

\subsubsection{Case I: $\lambda\le 1$}

Recall that for this case, we must have $\lambda\in (\frac{1}{2},1]$ and for this range, we found the inner equation

\begin{equation}\label{eq:oneD_slowosc_innerintegral}
{v_0}_t= -\epsilon^{1-\lambda}m(t)+\frac{1}{\pi}\int_0^{2\pi}| v_0(t)- A\cos(T)|\,dT=f(t,v_0).
\end{equation}

As in our previous analysis, we consider two regions of the solution $v_0(t)$ for this integral, in \autoref{subsec:oneD_slowosc_caseI} with sub-case I: $v_0(t)\le - |A|$ and in \autoref{subsec:oneD_slowosc_caseII} with sub-case II: $|v_0(t)|\le |A|$ where each these sub-cases deal with the respective size of $v_0(t)$ to the amplitude of the effective oscillations.

\subsubsection*{Sub-Case I: $v_0(t)\le - |A|$}

Recall from the analysis that equation \eqref{eq:oneD_slowosc_innerintegral} simplifies in this region of $v_0(t)$ and has the following inner equation and pseudo-equilibrium $z^0(t)$

\begin{equation}\label{eq:oneD_slowosc_stabilitysubcaseI}
{v_0}_t= -\epsilon^{1-\lambda}m(t) -2v_0=f(t,v_0), \quad z^0(t)=-\epsilon^{1-\lambda}\frac{m(t)}{2}.
\end{equation}

As we saw in \autoref{sec:oneD_slow}, special treatment of the pseudo-equilibrium stability analysis is needed with linear perturbations $v_0(t)=z^0(t)+u(t)$, where $\lVert u(t)\rVert \ll 1$ and $z^0_t = -\epsilon^{1-\lambda}\frac{m_t}{2}=\frac{\epsilon^{1-\lambda}}{2}$. The resulting Taylor expansion is thus

\begin{equation*}
\begin{aligned}
{v_0}_t =& f(t,z^0)+f_{v_0}(t,v_0)(v_0(t)-z^0(t))+O(\lVert v_0(t)-z^0(t) \rVert^2),\\
u_t =&-\frac{\epsilon^{1-\lambda}}{2}-2u.
\end{aligned}
\end{equation*}

This leads to the conclusion that equation \eqref{eq:oneD_slowosc_stabilitysubcaseI} causes perturbations to decay exponentially to just below the pseudo-equilibrium. Hence we find the pseudo-equilibrium to be a hyperbolic and asymptotically attracting solution.

\subsubsection*{Sub-Case II: $v_0(t)\le |A|$}

With the Taylor approximation from the analysis \eqref{eq:oneD_slowosc_subcaseIItaylor}, we have the following inner equation and pseudo-equilibrium $z^0(t)$

\begin{equation}\label{eq:oneD_slowosc_stabilitysubcaseII}
{v_0}_t= -\epsilon^{1-\lambda}m(t) +\frac{4|A|}{\pi}+\frac{2}{\pi |A|}v_0^2=f(t,v_0),\quad z^0(t)=-C \sqrt{\epsilon^{1-\lambda}m(t)-\frac{4|A|}{\pi}}.
\end{equation}

We consider simple linear perturbations to this pseudo-equilibrium \eqref{eq:oneD_slowosc_stabilitysubcaseII}, $v_0(t)=z^0(t)+u(t)$ with $\lVert u(t) \rVert \ll 1$. Treating the pseudo-equilibrium carefully, we find that the slowly varying component of the equilibrium contributes to the derivative. Thus we have

\begin{equation}
\begin{aligned}
{v_0}_t =& z^0_t(t) +u_t,\\
z^0_t(t) = & \begin{cases}
\frac{\epsilon^{1-\lambda}}{2C\sqrt{\epsilon^{1-\lambda}m(t)-\frac{4|A|}{\pi}}} & \epsilon^{1-\lambda}m(t)> \frac{4|A|}{\pi},\\
0 & \epsilon^{1-\lambda}m(t) =\frac{4|A|}{\pi}.
\end{cases}
\end{aligned}
\end{equation}

Now applying a Taylor expansion, we find the following behavior of perturbations

\begin{equation}\label{eq:oneD_slowosc_stabilitysubcaseIIeq}
\begin{aligned}
{v_0}_t =& f(t,z^0)+f_{v_0}(t,z^0)(v_0-z^0(t))+O(\lVert v_0(t)-z^0(t) \rVert),\\
u_t =&\begin{cases}
-\frac{\epsilon^{1-\lambda}}{2C\sqrt{\epsilon^{1-\lambda}m(t)-\frac{4|A|}{\pi}}}-2\sqrt{\epsilon^{1-\lambda}m(t)-\frac{4|A|}{\pi}} u & \epsilon^{1-\lambda}m(t)>\frac{4|A|}{\pi},\\
0 & \epsilon^{1-\lambda}m(t)=\frac{4|A|}{\pi}.
\end{cases}
\end{aligned}
\end{equation}

\indent From \eqref{eq:oneD_slowosc_stabilitysubcaseIIeq}, we find that the perturbations decay to a fixed negative quantity. This indicates, much like in \autoref{sec:oneD_slow}, that there is an attracting solution below the pseudo-equilibrium. The negative sign describes exponential decay and hence this equilibrium is hyperbolic and asymptotically stable for $\epsilon^{1-\lambda}m(t)>\frac{4|A|}{\pi}$ or $\mu(t)>\frac{4|A|}{\pi \Omega}$. For $\epsilon^{1-\lambda}m(t)=\frac{4|A|}{\pi}$ or $\mu(t) =\mu_{\text{osc}}$, the stability of \eqref{eq:oneD_slowosc_stabilitysubcaseIIeq} suddenly becomes non-hyperbolic. This tells us that we lose stability at the oscillatory bifurcation but the tipping point occurs afterwards, which agrees with the conclusion in the tipping approximation from \eqref{eq:oneD_slowosc_caseItipping}.

\subsubsection{Case II: $\lambda>1$}

From the analysis, we discovered that as long as $\epsilon^{\lambda-1}A\sim O(1)$, then we have a similar behavior in the tipping point. With the Taylor approximation from the analysis \eqref{eq:oneD_slowosc_caseII_integral}, the inner equation and pseudo-equilibrium $z^0(t)$ are

\begin{equation}\label{eq:oneD_slowosc_stabilitycaseII}
\begin{aligned}
y_0=&-m(t) +\epsilon^{\lambda-1}\frac{2|A|}{\pi}+\epsilon^{1-\lambda}\frac{2}{\pi |A|}y_0^2=f(t,y),\\
z^0(t)=&-\epsilon^{\lambda-1}C\sqrt{m(t)-\epsilon^{\lambda-1}\frac{4|A|}{\pi}}.
\end{aligned}
\end{equation}

\indent Similarly to Case I, we consider simple linear perturbations to this pseudo-equilibrium \eqref{eq:oneD_slowosc_stabilitycaseII}, $y_0(t)=z^0(t)+u(t)$ with $\lVert u(t) \rVert \ll 1$. Treating the pseudo-equilibrium carefully, we find that the slowly varying component of the equilibrium contributes to the derivative. Thus we have

\begin{equation}
\begin{aligned}
{y_0}_t =& z^0_t(t) +u_t,\\
z^0_t(t) = & \begin{cases}
\frac{\epsilon^{\lambda-1}}{2C\sqrt{m(t)-\epsilon^{\lambda-1}\frac{4|A|}{\pi}}} & m(t)> \epsilon^{\lambda-1}\frac{4|A|}{\pi},\\
0 & m(t) =\epsilon^{\lambda-1}\frac{4|A|}{\pi}.
\end{cases}
\end{aligned}
\end{equation}

Now applying a Taylor expansion, we find the following behavior of perturbations

\begin{equation}\label{eq:oneD_slowosc_stabilitycaseIIeq}
\begin{aligned}
{y_0}_t =& f(t,z^0)+f_{y_0}(t,z^0)(y_0-z^0(t))+O(\lVert y_0-z^0(t) \rVert^2),\\
u_t =&\begin{cases}
-\frac{\epsilon^{\lambda-1}}{2C\sqrt{m(t)-\epsilon^{\lambda-1}\frac{4|A|}{\pi}}}-2\sqrt{m(t)-\epsilon^{\lambda-1}\frac{4|A|}{\pi}} u & m(t)>\epsilon^{\lambda-1}\frac{4|A|}{\pi},\\
0 & m(t)=\epsilon^{\lambda-1}\frac{4|A|}{\pi}.
\end{cases}
\end{aligned}
\end{equation}

The conclusions from case I still apply to \eqref{eq:oneD_slowosc_stabilitycaseIIeq} and thus we still have an attracting solution until $\mu(t)=\mu_{\text{osc}}$ and expect to see tipping occurring after the oscillatory bifurcation which is consistent with our tipping approximation for this case.

\indent On the other hand, for large $\lambda$ the integral \eqref{eq:oneD_slowosc_caseII_integral} approaches

\begin{equation}\label{eq:oneD_slowosc_stabilitycaseII}
{y_0}_t=-m(t)+2|y_0|.
\end{equation}

This is the same type of behavior from \autoref{sec:oneD_slow}, where we found that for $m(t)\ge 0$ our pseudo-equilibrium was attracting and for $m(t)<0$ searching for the pseudo-equilibrium caused a contradiction. Thus, we conclude that the tipping point occurs in the region of $m(t)<0$ which agrees with \eqref{eq:oneD_slow_tipping}.


\chapter{Two-Dimensional Model}
\label{chap:twoD}
With the methods and approaches developed in \autoref{chap:oneD} for the one-dimensional model, we have an expectation of the behavior of the two-dimensional Stommel model around the non-smooth bifurcation under similar conditions. With the bifurcation structure we explored in \autoref{chap:background}, we consider a generalization with the canonical model

\begin{equation}\label{eq:twoD_canonical}
  \begin{aligned}
   \dot{V} & =  \eta_1-\eta_2+\eta_3(T-V)-T-V|V|+A\sin(\Omega t), \\
   \dot{T}     & =  \eta_1-T(1+|V|)+B\sin(\Omega t),  \\
  \dot{\eta_2}  & =  -\epsilon\\
  T(0)&=T_i,\quad V(0)=V_i,\quad \eta_2(0)={\eta_2}_i>\eta_1\eta_3,
  \end{aligned}
\end{equation}

with slow variation $\epsilon\ll 1$, high frequency $\Omega\gg 1$, amplitudes of oscillation $A$ and $B$, and model parameters $\eta_1$ and $\eta_3$ to be fixed positive constants. This is the generalized two-dimensional Stommel model with two additional features. First, we allow for slow variation in the bifurcation parameter, this has been shown to be realistic as the $\eta_2$ is related to the freshwater flux and therefore not a fixed parameter; the same assumption is made in Roberts \cite{roberts2017relaxation}. Second, we consider periodic forcing in the additive parameters ($\eta_1,\eta_2$) to account for oscillations in the observed behavior in Huybers \cite{huybers2005obliquity}. To fully understand the effects of each component on the model, we build them individually before putting them together.

For the remainder of this paper, we make two assumptions: first that $\eta_3<1$ which causes \eqref{eq:twoD_canonical} to admit the smooth bifurcation in the positive $V$ region. The value $\eta_3$ describes the relative strength of the temperature relaxation to that of salinity, and it is frequently assumed that salinity's is much longer, giving $\eta_3<1$. Although not necessary, this will align our focus and give a case to analyze in depth, the case of $\eta_3>1$ follows similarly. The second assumption we make is that even though we have a two-dimensional model, the variable $V$ is leading the dynamics of the system with it's nonlinear behavior. This assumption makes it clear that we will want to understand the non-smooth behavior in $V$ where $T$ follows in response to the effects of $V$. Again, also not necessary as the opposite situation may be considered, there is evidence to show that changes in temperature respond to changes in salinity and this assumption allows for better physical agreement. With this we have the ability to solve behavior of $T$ in terms of $V$ to find equations in only one variable.

\section{Slowly Varying Parameter}
\label{sec:twoD_slow}

We consider only the slow variation mechanism to understand it's effects on the canonical system \eqref{eq:twoD_canonical} with $\epsilon>0$ and $A=B=0$, where we allow the bifurcating parameter to slowly vary but no oscillatory forcing. With the parameter $\eta_2$ slowly varying, we expect to find a tipping point in the neighborhood of the aforementioned non-smooth bifurcation. With the choice of $\eta_3<1$, the lower branch with $V<0$ is the branch we focus on in order to approach the non-smooth behavior, thus \eqref{eq:twoD_canonical} becomes

\begin{equation}\label{eq:twoD_slow_negative}
 \begin{aligned}
   \dot{V} & =  \eta_1-\eta_2(t)+\eta_3(T-V)-T+V^2, \\
   \dot{T} & =  \eta_1-T(1-V),  \\
  \dot{\eta_2}  & =  -\epsilon.
  \end{aligned}
\end{equation}

From \autoref{sec:oneD_slow}, we learned that the one-dimensional model had a solution that began to act differently on a smaller scale. It was this approach that gave insight into the tipping. Here we search for an outer solution to \eqref{eq:twoD_slow_negative} that helps us understand the behavior of system.  Since we have slow variation in the parameter, we choose to scale the system \eqref{eq:twoD_slow_negative} with the 'slow' time $\tau=\epsilon t$

\begin{equation}\label{eq:twoD_slow_slowsystem}
\begin{aligned}
\epsilon V_\tau =&\eta_1-\eta_2(\tau)+\eta_3(T-V)-T+V^2, \\
\epsilon T_\tau & =  \eta_1-T(1-V),  \\
  {\eta_2}_\tau  & =  -1.
\end{aligned}
\end{equation}

We also form an asymptotic expansions in terms of the small quantity $\epsilon$ to get an expression that separates the dynamics by their contribution to the solution. Here we choose

\begin{equation}\label{eq:twoD_slow_outerexpansion}
\begin{aligned}
V(\tau)\sim &V_0(\tau)+\epsilon V_1(\tau)+\epsilon^2 V_2+\ldots\\
T(\tau)\sim & T_0(\tau)+\epsilon T_1(\tau)+\epsilon^2 T_2(\tau)+\ldots
\end{aligned}
\end{equation}

and substitute \eqref{eq:twoD_slow_outerexpansion} into \eqref{eq:twoD_slow_slowsystem} to find

\begin{equation*}
\begin{aligned}
 \epsilon{V_0}_\tau+\epsilon^2{V_1}_\tau+\ldots =&\begin{aligned}[t]
\eta_1-&\eta_2(\tau)+\eta_3(T_0-V_0)-T_0+V_0^2\\
&+\epsilon(\eta_3(T_1-V_1)-T_1-2V_1V_0)+\ldots
\end{aligned}\\
\epsilon{T_0}_\tau+\epsilon^2{T_1}_\tau+\ldots=&\eta_1-T_0(1-V_0)+\epsilon(-T_1(1-V_0)+V_1T_0)+\ldots
\end{aligned}
\end{equation*}

Which once we separate at each order of $\epsilon$ we find the equations

\begin{align}
\label{eq:twoD_slow_outerO1}
O(1):\quad & \begin{cases}
	0 =& \eta_1-\eta_2(\tau)+\eta_3(T_0-V_0)-T_0+V_0^2 , \\
	0 =&  \eta_1-T_0(1-V_0),\\
\end{cases}\\
\label{eq:twoD_slow_outerO2}
O(\Omega^{-1}):\quad & \begin{cases}
	{V_0}_\tau = & \eta_3(T_1-V_1)-T_1+2V_1V_0,\\
	{T_0}_\tau =&  -T_1(1-V_0)+V_1T_0,
\end{cases}
\end{align}

Where we solve \eqref{eq:twoD_slow_outerO1} simultaneously for the pseudo-equilibria and we choose to solve $T_0$ in terms of $V_0$ from our assumption that $T$ responds to $V$ and thus we find the equation for $V_0$ with

\begin{equation}\label{eq:twoD_slow_equilibria}
\begin{aligned}
T_0(V_0)=&\frac{\eta_1}{1-V_0},\\
0=\eta_1-\eta_2(\tau)-T_0(V_0)&+\eta_3(T_0(V_0)-V_0)+V_0^2.
\end{aligned}
\end{equation}

With $T_0$ and $V_0$ found, we use \eqref{eq:twoD_slow_outerO2} to search for the solution to $T_1$ and $V_1$ but first we note that with $\eta_\tau =-1$ we have

\begin{equation*}
\begin{aligned}
&{T_0}_\tau({V_0}_\tau)=\frac{{V_0}_\tau T_0(V_0)}{1-V_0},\\
{V_0}_\tau = &\frac{(1-V_0)}{(\eta_3-2V_0)(1-V_0)+(1-\eta_3)T_0(V_0)}.
\end{aligned}
\end{equation*}

Thus we can solve \eqref{eq:twoD_slow_outerO2} for $T_1$ in terms of $V_1$ with the same assumption as before and this results in

\begin{equation}\label{eq:twoD_slow_equilcorrec}
\begin{aligned}
&T_1(V_1) = \frac{{T_0}_\tau-T_0(V_0)V_1}{1-V_0},\\
V_1 =& \frac{-(1-V_0){V_0}_\tau+(1-\eta_3){T_0}_\tau({V_0}_\tau)}{(1-\eta_3)T_0(V_0)+(\eta_3-2V_0)(1-V_0)}.
\end{aligned}
\end{equation}

Which we have the first few terms of the asymptotic expansion \eqref{eq:twoD_slow_outerexpansion} with \eqref{eq:twoD_slow_equilibria} and \eqref{eq:twoD_slow_equilcorrec}, although the method of using an asymptotic expansion to find when the inner dynamics form isn't as feasible in the two-dimensional problem as even the leading order term is complex. But the method of scaling the system to find an inner equation from the one-dimensional model in should hold just the same here in the two-dimensional case.

We perform a separate analysis analogous to \autoref{sec:oneD_slow} to determine the appropriate scaling, and since we know our non-smooth bifurcation to occur at $\eta_2=\eta_1\eta_3$ when $V=0$ and $T=\eta_1$, it makes the most sense to rescale \eqref{eq:twoD_canonical} around these values. This results in the scalings

\begin{equation}\label{eq:twoD_slow_rescale}
\begin{aligned}
\eta_2=&\eta_1\eta_3+\epsilon n,\\
V=&\epsilon X,\\
T=&\eta_1+\epsilon Y.
\end{aligned}
\end{equation}

We introduce these scalings \eqref{eq:twoD_slow_rescale} into the canonical system \eqref{eq:twoD_canonical} to find the following inner system

\begin{equation}\label{eq:twoD_slow_inner}
\begin{aligned}
   \dot{X} & =  -n(t)-\eta_3 X-(1-\eta_3)Y-\epsilon X|X|, \\
   \dot{Y} & =  -\eta_1 |X|-Y-\epsilon |X|Y,  \\
  \dot{n}  & =  -1.
  \end{aligned}
\end{equation}

Generally speaking, the parameters $\eta_1$ and $\eta_3$ have quite an effect on the behavior of a solution. We already determined from the introduction that $\eta_3$ will determine the orientation of the problem, but here we find a relationship between the parameters $\eta_1$ and $\eta_3$ by viewing \eqref{eq:twoD_slow_inner} as a $2\times 2$ system of spatial coordinates

\begin{equation}\label{eq:twoD_slow_innermatrix}
\begin{pmatrix}
\dot{X}\\
\dot{Y}
\end{pmatrix}=
\begin{pmatrix}
-\eta_3 & -(1-\eta_3) \\ 
-\eta_1\text{sgn}(X) & -1
\end{pmatrix}
\begin{pmatrix}
X\\
Y
\end{pmatrix}-
\begin{pmatrix}
n(t)+\epsilon X|X|\\
\epsilon |X|Y
\end{pmatrix}.
\end{equation}

Where the system \eqref{eq:twoD_slow_innermatrix} has eigenvalues that are either real or complex depending on the choice in $\eta_1$ and $\eta_3$ dictated by

\begin{equation}\label{eq:twoD_slow_eigen}
\lambda_{1,2} = -\frac{\eta_3+1}{2}\pm \frac{1}{2}\sqrt{(\eta_3+1)^2-4(\eta_3-\eta_1(1-\eta_3)\text{sgn}(X))}.
\end{equation}

It is important to notice that there is non-smooth behavior in the discriminant of \eqref{eq:twoD_slow_eigen}, telling us that there is different behavior for these eigenvalues for the respective sign of $X$. If we consider $X<0$ like in \autoref{sec:oneD_slow}, then the parameters $\eta_1$ and $\eta_3$ must adhere to

\begin{equation}\label{eq:twoD_parameters}
0<\eta_3\le 1-4\eta_1,\quad 0<\eta_1<\frac{1}{4}
\end{equation}

to admit real eigenvalues. Under the conditions in \eqref{eq:twoD_parameters}, the discriminant in \eqref{eq:twoD_slow_eigen} is positive and less than ${(\eta+1)^2}$ which causes both eigenvalues to be negative. Thus we have that the solutions in this region decay exponentially and hence stable. If the conditions of \eqref{eq:twoD_parameters} are not met, we see qualitatively different behavior with complex eigenvalues but from \eqref{eq:twoD_slow_eigen} we see the real component is still negative and the exponential decay is still present where stability arises from this. The eigenvalues in \eqref{eq:twoD_slow_eigen} describe the type of behavior we see near the non-smooth bifurcation, when $V<0$ and for real eigenvalues we see pure decay where complex eigenvalues will result in decaying spiral behavior around the equilibrium. In both cases, no erratic behavior and no sign of tipping occur up to leading order.

In \autoref{sec:oneD_slow} we found that the non-smooth bifurcation was a critical point and that the region immediately after contained the tipping. For our two-dimensional model, this critical point is $(\eta_2,V,T)=(\eta_1\eta_3,0,\eta_1)$, which results in the eigenvalues from \eqref{eq:twoD_slow_eigen} as $\lambda_{1,2}=\{-1,-\eta_3\}$, which are both negative real valued and hence we still have stability. Thus we expect our tipping to occur just after the standard non-smooth bifurcation here as well.

We now consider $V>0$ and with \eqref{eq:twoD_slow_innermatrix} we find the inner system

\begin{equation}\label{eq:twoD_slow_positiveinner}
 \begin{aligned}
   \dot{X} & =  -n(t)-\eta_3 X-(1-\eta_3)Y-\epsilon X^2, \\
   \dot{Y} & =  -\eta_1 X-Y-\epsilon XY,  \\
  \dot{n}  & =  -1.
  \end{aligned}
\end{equation}

Following the approach from \autoref{sec:oneD_slow}, we relate the solution directly to the parameter to find their relationship. Thus we swap the respective differentiation onto $n$; for convenience we write this as a $2\times 2$ system

\begin{equation*}
\begin{pmatrix}
X_n\\
Y_n
\end{pmatrix}=
\begin{pmatrix}
\eta_3 & 1-\eta_3 \\ 
\eta_1 & 1
\end{pmatrix}
\begin{pmatrix}
X\\
Y
\end{pmatrix} +
\begin{pmatrix}
n+\epsilon X^2\\
\epsilon XY
\end{pmatrix}.
\end{equation*}

We seek a leading order solution in this region, thus we are permitted to drop the $\epsilon$ order terms to give 

\begin{equation}\label{eq:twoD_slow_uppermatrix}
\begin{pmatrix}
X_n\\
Y_n
\end{pmatrix}=
\begin{pmatrix}
\eta_3 & 1-\eta_3 \\ 
\eta_1 & 1
\end{pmatrix}
\begin{pmatrix}
X\\
Y
\end{pmatrix} +
\begin{pmatrix}
n\\
0
\end{pmatrix}.
\end{equation}

For the system \eqref{eq:twoD_slow_uppermatrix}, we find the following eigenvalues using \eqref{eq:twoD_slow_eigen}

\begin{equation}\label{eq:twoD_slow_uppereigen}
\lambda_{1,2}=\frac{\eta_3+1}{2}\pm\frac{1}{2}\sqrt{(1+\eta_3)^2+4(\eta_1(1-\eta_3)-\eta_3)}.
\end{equation}

These eigenvalues in \eqref{eq:twoD_slow_uppereigen} must be real as $\eta_3<1$ guarantees the discriminant is always positive. But the stability can also be determined here, as $\lambda_1<0<\lambda_2$ causes the solution to be unstable, which confirms tipping to occur in the region $V>0$. With real eigenvalues, the solution in the $V>0$ region takes the following exponential form with constants $K_{i,j}$ being the jth component of the corresponding ith eigenvector 

\begin{equation}\label{eq:twoD_slow_inner}
\begin{aligned}
X(n)\sim& K_{1,1}e^{\lambda_1 n}+K_{2,1}e^{\lambda_2 n}+C_1 n+C_2,\\
Y(n)\sim& K_{1,2}e^{\lambda_1 n}+K_{2,2}e^{\lambda_2 n}+C_3 n+C_4.\\
\end{aligned}
\end{equation}

Translating both solutions in \eqref{eq:twoD_slow_inner} back to our original coordinates we find

\begin{equation}\label{eq:twoD_slow_inneroriginal}
\begin{aligned}
V(t)\sim&  \eta_2(t)-\eta_1\eta_3+ \epsilon K_{1,1}e^{\lambda_1(\eta_2(t)-\eta_1\eta_3)/\epsilon}+\epsilon K_{2,1}e^{\lambda_1(\eta_2(t)-\eta_1\eta_3)/\epsilon}+O(\epsilon),\\
T(t)\sim& \eta_2(t)-\eta_1+ \epsilon K_{1,2}e^{\lambda_1 (\eta_2(t)-\eta_1\eta_3)/\epsilon}+\epsilon K_{2,2}e^{\lambda_2 (\eta_2(t)-\eta_1\eta_3)/\epsilon}+O(\epsilon).
\end{aligned}
\end{equation}

With \eqref{eq:twoD_slow_inneroriginal} admitting the solution in the region $V>0$, we determine the system to tip once one of these exponentials becomes large (i.e $O(1/\epsilon)$), which causes the system to diverge away from our lower branch towards the upper stable branch. This can be seen with

\begin{equation}\label{eq:twoD_slow_tipping}
{\eta_2}_{\text{slow}}= \min\{\eta_1\eta_3-\epsilon\ln\epsilon/\lambda_i\},\quad i=1,2.
\end{equation}

Thus we have the tipping for this problem with \eqref{eq:twoD_slow_tipping} and this has a noticeably similar form to the tipping from \autoref{sec:oneD_slow}. As we found from \eqref{eq:twoD_slow_uppereigen}, one of the eigenvalues is always negative and it is with this we find that the tipping is delayed with respect to the bifurcation. This in turn allows for the region of bi-stability to be extended and with more bi-stability, the hysteresis of the Stommel model allows the solution to spend more time around the lower branch before transitioning to the upper branch. These effects shrink as $\epsilon\to 0$ until we return to the static problem with $\epsilon=0$,

\begin{figure}[H]
\centering
\begin{subfigure}{.5\textwidth}
  \centering
  \includegraphics[width=\linewidth]{twoD/slow_bif_diagram.jpg}
  \caption{}
\end{subfigure}%
\begin{subfigure}{.5\textwidth}
  \centering
  \includegraphics[width=\linewidth]{twoD/slow_bif_diagram_zoom.jpg}
  \caption{}
\end{subfigure}
\caption{In (a) the numerical solution (black dotted line) to \eqref{eq:twoD_canonical} is given with $\eta_1=4$, $\eta_3=.375$, and $\epsilon=.01$. In (b) a zoom in closer to the non-smooth bifurcation region where the blue vertical line is the prediction \eqref{eq:twoD_slow_tipping} against the black dotted vertical line which is the numerical tipping point.}
\label{fig:twoD_slow_Vnumerics}
\end{figure}

In figure~\ref{fig:twoD_slow_Vnumerics} an example of the slowly varying is given for a choice of $\epsilon$ in (a) and we zoom in around the non-smooth bifurcation in (b). Here we use the tipping criteria to be whenever $V>0.5$, this is large enough that the solution is going towards the upper branch indefinitely and is around the point of the smooth bifurcation. The effect of seeing a delay moving towards the upper branch in the $V$ solution causes a similar delay in $T$ seen in figure~\ref{fig:twoD_slow_Tnumerics}, where the delay causes the maximum value of $T$ to never be achieved. Here notice that after the tipping occurs, the numerical solution passes entirely over the unstable branch and even some of the upper stable branch before it resumes following closely.

\begin{figure}[H]
\centering
\begin{subfigure}{.5\textwidth}
  \centering
  \includegraphics[width=\linewidth]{twoD/slow_bif_Tplot.jpg}
  \caption{}
\end{subfigure}%
\begin{subfigure}{.5\textwidth}
  \centering
  \includegraphics[width=\linewidth]{twoD/slow_bif_Tplot_zoom.jpg}
  \caption{}
\end{subfigure}
\caption{In (a) we have the numerical solution (black dotted) over the standard equilibrium plot for $V$ vs. $T$. In (b) a zoom of the bifurcation area.}
\label{fig:twoD_slow_Tnumerics}
\end{figure}

In figure~\ref{fig:twoD_slow_epscomp} we compare the numerical tipping to the predicted tipping in \eqref{eq:twoD_slow_tipping} over a range of epsilon. Here we see performance even better than in section \autoref{sec:oneD_slow} as even for relatively large $\epsilon$ the prediction has small error. This is an artifact of having a higher dimensional problem, where now two exponentials in \eqref{eq:twoD_slow_inneroriginal} are dominating the behavior of the solution in the $V>0$ region. As in \autoref{sec:oneD_slow}, the concavity of the predicted tipping against the numerical tipping match very well and we can expect the prediction to hold for reasonably small values of $\epsilon$.

\begin{figure}[H]
\centering
\includegraphics[width=\linewidth]{twoD/slow_epscomp.jpg}
\caption{The numerical tipping vs the estimate with $\eta_1=4$ and $\eta_3=\frac{3}{8}$. The tipping criteria is $V>.5$.}
\label{fig:twoD_slow_epscomp}
\end{figure}

\section{High Frequency Oscillatory Forcing}
\label{sec:twoD_highfreqosc}

It is a physical behavior of the THC to see oscillations occurring in the dynamics that is not originally encompassed by the Stommel model \cite{alley2003abrupt,huybers2005obliquity,marotzke2000abrupt,rahmstorf2000thermohaline,rahmstorf2002ocean,stastna2007box}. We choose to allow $\eta_1$ and $\eta_2$ to exhibit oscillatory behavior to account for this. As these parameters appear in both equations for $V$ and $T$, here we consider the canonical system \eqref{eq:twoD_canonical} with $A,B\sim O(1)$, $\Omega\gg 1$ and $\epsilon=0$ which is the purely oscillatory forcing problem. Under these conditions, we expect to find oscillations about some stable behavior; this stable behaior should act like the equilibria of a reduced inner problem. Thus our analysis must locate these equilibria for each variable up to the oscillations in order to find the bifurcation.

To begin our analysis, we note that there is behavior happening on multiple time scales, a 'slow' $t$ and 'fast' $R = \Omega t$. Following the multiple scales method, we consider $V(t)=V(t,R)$ and $T(t)=T(t,R)$ and substituting this into \eqref{eq:twoD_canonical}, we get the system

\begin{equation}\label{eq:twoD_osc_multiscaleouter}
\begin{aligned}
V_R+\Omega^{-1}V_t = & \Omega^{-1}\left(\eta_1-\eta_2+\eta_3(T-V)-T-V|V|+A\sin(R)\right),\\
T_R+\Omega^{-1}T_t = & \Omega^{-1}\left(\eta_1-T(1+|V|)+B\sin(R)\right).
\end{aligned}
\end{equation}

As per our typical approach to the non-smooth behavior, we follow the lower branch to isolate this dynamic. Thus we consider the system \eqref{eq:twoD_osc_multiscaleouter} with $V<0$

\begin{equation}\label{eq:twoD_osc_multiscaleouterlower}
\begin{aligned}
V_R+\Omega^{-1}V_t = & \Omega^{-1}\left(\eta_1-\eta_2+\eta_3(T-V)-T+V^2+A\sin(R)\right),\\
T_R+\Omega^{-1}T_t = & \Omega^{-1}\left(\eta_1-T(1-V)+B\sin(R)\right).
\end{aligned}
\end{equation}

From \eqref{eq:twoD_osc_multiscaleouterlower}, it makes sense to consider an asymptotic expansion for both $V$ and $T$ in terms of the small quantity that appears, $\Omega^{-1}$, with

\begin{equation}\label{eq:twoD_osc_outerexpansion}
\begin{aligned}
V(t,R)\sim V_0(t,R) +\Omega^{-1}V_1(t,R) +\Omega^{-2}V_2(t,R)+O(\Omega^{-3}),\\
T(t,R)\sim T_0(t,R) +\Omega^{-1}T_1(t,R) +\Omega^{-2}T_2(t,R)+O(\Omega^{-3}).
\end{aligned}
\end{equation}

Substituting \eqref{eq:twoD_osc_outerexpansion} into the system \eqref{eq:twoD_osc_multiscaleouterlower} gives

\begin{equation*}
\begin{aligned}
{V_0}_R+\Omega^{-1}{V_0}_t+\Omega^{-1}{V_1}_R+\ldots=&\begin{aligned}[t]\Omega^{-1}&(\eta_1-\eta_2+\eta_3(T_0-V_0)-T_0+V_0^2+A\sin(R))\\
+&\Omega^{-2}(\eta_3(T_1-V_1)-T_1+2V_1V_0)+\ldots
\end{aligned}\\
{T_0}_R+\Omega^{-1}{T_0}_t+\Omega^{-1}{T_1}_R+\ldots=&\begin{aligned}[t]  \Omega^{-1}&(\eta_1-T_0(1-V_0)+B\sin(R))\\
+&\Omega^{-2}(-T_1(1-V_0)+T_0V_1)+\ldots
\end{aligned}
\end{aligned}
\end{equation*}

Which we then find the following equations separated by order of $\Omega^{-1}$ with

\begin{align}
\label{eq:twoD_osc_outerO1}
O(1):\quad & \begin{cases}
	{V_0}_R =&  0, \\
	{T_0}_R =&  0,\\
\end{cases}\\
\label{eq:twoD_osc_outerO2}
O(\Omega^{-1}):\quad & \begin{cases}
	{V_1}_R+{V_0}_t = & \eta_1-\eta_2+\eta_3(T_0-V_0)-T_0+V_0^2+A\sin(R),\\
	 {T_1}_R +{T_0}_t =&  \eta_1-T_0(1-V_0)+B\sin(R),
\end{cases}\\
\label{eq:twoD_osc_outerO3}
O(\Omega^{-2}):\quad & \begin{cases}
	{V_2}_R+{V_1}_t = & \eta_3(T_1-V_1)-T_1+2V_0V_1,\\
	 {T_2}_R +{T_1}_t =&  -T_1(1-V_0)+T_0V_1.
\end{cases}
\end{align}

We learn from \eqref{eq:twoD_osc_outerO1} that both our leading order terms are purely dependent on the slow variable, $V_0=V_0(t)$, $T_0=T_0(t)$. But much like \autoref{sec:oneD_highfreqosc}, we must introduce a solvability condition on the resonant terms to be able to solve for the correction terms. This secures terms that are both consistent with one another as well as less than linear in their growth, making for a stable expansion. Here, we use the Fredholm alternative \eqref{eq:Fredholm} on \eqref{eq:twoD_osc_outerO2}-\eqref{eq:twoD_osc_outerO3} and search for the equilibrium solutions, the work for this can be found in \autoref{app:twoD}. This leads to the outer solution in original coordinates

\begin{equation}\label{eq:twoD_osc_outersoln}
\begin{aligned}
V\sim& V_0-\Omega^{-1} A\cos(\Omega t)+\dots\\
%\Omega^{-2}\left(V_2(t)+\left(A(\eta_3-2V_0)+B(1-\eta_3)\right)\sin(\Omega t)\right),\\
T\sim& T_0-\Omega^{-1} B\cos(\Omega t)+\ldots%\Omega^{-2}\left(T_2(t)+(1-V_0)B-T_0A)\sin(\Omega t)\right).
\end{aligned}
\end{equation}

Where $V_0$ and $T_0$ are the same equilibrium solutions from the static model in the introduction with

\begin{equation*} \label{eq:twoD_lowerleadingorder}
\begin{aligned}
T_0(V_0)=&\frac{\eta_1}{1-V_0},\\
0=\eta_1-\eta_2+\eta_3&(T_0(V_0)-V_0)-T_0(V_0)+V_0^2.
\end{aligned}
\end{equation*}

From the one-dimensional model in \autoref{sec:oneD_highfreqosc}, we discovered that in order to access the bifurcation we needed to scale both the coordinate $x$ as well as the parameter $\mu$ and analyze the behavior occurring around axis for $x=0$. Since we still have non-smooth behavior occurring at the axis $V=0$, we expect this approach to hold for the two-dimensional model as well. But the outer solution \eqref{eq:twoD_osc_outersoln} is too complex for us to search for when the assumptions of the asymptotic series break that would help find the appropriate scaling. Instead, a separate scales analysis analogous to  \autoref{sec:oneD_highfreqosc} leads to the following scaling

\begin{equation}\label{eq:twoD_osc_scales}
\begin{aligned}
V=&\Omega^{-1}X,\\
T=& \eta_1 +\Omega^{-1}Y,\\
\eta_2=&\eta_1\eta_3+\Omega^{-1} n.
\end{aligned}
\end{equation}

Substituting \eqref{eq:twoD_osc_scales} into \eqref{eq:twoD_canonical} leads to the following inner system

\begin{equation}\label{eq:twoD_osc_innersystem}
\begin{aligned}
\dot{X}=& -n+\eta_3(Y-X)-Y-\Omega^{-1}X|X| +\Omega A\sin(\Omega t),\\
\dot{Y}=& -\eta_1|X|-Y -\Omega^{-1}|X|Y +\Omega A \sin(\Omega t).
\end{aligned}
\end{equation}

Where we still see behavior on the same time scales in \eqref{eq:twoD_osc_innersystem}, the 'slow' $t$ and the 'fast' $R=\Omega t$. Considering both $X(t)=X(t,R)$ and $Y(t)=Y(t,R)$ gives the multiple scales inner system

\begin{equation}\label{eq:twoD_osc_innermulti}
\begin{aligned}
X_R+\Omega^{-1}X_t =& \Omega^{-1}\left(-n +\eta_3(Y-X)-Y\right)-\Omega^{-2}X|X|+A\sin(R),\\
Y_R+\Omega^{-1}Y_t =& \Omega^{-1}\left(-\eta_1|X|-Y\right)-\Omega^{-2}|X|Y+ B\sin(R).
\end{aligned}
\end{equation}

Once more, as we see the small quantity $\Omega^{-1}$ appearing in \eqref{eq:twoD_osc_innermulti}, then we choose an expansion of the form

\begin{equation}\label{eq:twoD_osc_innerexpan}
\begin{aligned}
X(t,R)\sim& X_0(t,R)+\Omega^{-1}X_1(t,R)+O(\Omega^{-2}),\\
Y(t,R)\sim& Y_0(t,R)+\Omega^{-1}Y_1(t,R)+O(\Omega^{-2}),
\end{aligned}
\end{equation}

where we then substitute \eqref{eq:twoD_osc_innerexpan} into \eqref{eq:twoD_osc_innermulti} to give the equation containing dynamics of all orders

\begin{equation*}
\begin{aligned}
{X_0}_R+\Omega^{-1}{X_0}_t+\Omega^{-1}{X_1}_R+\ldots=&\begin{aligned}[t]\Omega^{-1}&(-n+\eta_3(Y_0-X_0)-Y_0)+A\sin(R)\\
+&\Omega^{-2}(X_0|X_0+\Omega^{-1}X_1+\ldots|+\eta_3(Y_1-X_1)-Y_1)+\ldots
\end{aligned}\\
{Y_0}_R+\Omega^{-1}{Y_0}_t+\Omega^{-1}{Y_1}_R+\ldots=&\begin{aligned}[t]\Omega^{-1}&(-\eta_1|X_0+\Omega^{-1}X_1+\ldots|-Y_0)+B\sin(R)\\
+&\Omega^{-2}(-|X_0+\Omega^{-1}X_1+\ldots|Y_0-Y_1)+\ldots
\end{aligned}
\end{aligned}
\end{equation*}

We then separate the dynamics by their respective contribution to the solution, here by order of $\Omega^{-1}$, to find the following equations

\begin{align}
\label{eq:twoD_osc_innerO1}
O(1):\quad & \begin{cases}
	{X_0}_R =& A\sin(R), \\
	{Y_0}_R =& B\sin(R),\\
\end{cases}\\
\label{eq:twoD_osc_innerO2}
O(\Omega^{-1}):\quad & \begin{cases}
	{X_1}_R+{X_0}_t =& -n-\eta_3X_0-(1-\eta_3)Y_0, \\
	{Y_1}_R+{Y_0}_t =& -\eta_1|X_0|-Y_0.\\
\end{cases}
\end{align}

From \eqref{eq:twoD_osc_innerO1} we find that the leading order terms of \eqref{eq:twoD_osc_innerexpan} have the form in terms of the time scales

\begin{equation}\label{eq:twoD_osc_innerO1soln}
X_0=P_0(t)-A\cos(R),\quad Y_0=Q_0(t)-B\cos(R).
\end{equation}
 
Substituting \eqref{eq:twoD_osc_innerO1soln} into \eqref{eq:twoD_osc_innerO2}, we apply the Fredholm alternative \eqref{eq:Fredholm} to solve in terms of the separate time scales. This results in

\begin{equation}\label{eq:twoD_osc_innerintegral}
\begin{aligned}
{P_0}_t =& -n -\eta_3P_0-(1-\eta_3)Q_0,\\
{Q_0}_t =& -\frac{\eta_1}{2\pi}\int_0^{2\pi}|P_0-A\cos(R)|dR-Q_0.
\end{aligned}
\end{equation}

As we are concerned with finding the bifurcation, we search for the equilibrium solutions to \eqref{eq:twoD_osc_innerintegral} but we find a similar integral equation to the inner equation in \autoref{sec:oneD_highfreqosc}. This leads us to the similar two case argument. Case I: $|P_0(t)|\le -|A|$ which prevents the sign of the integrand from changing, and Case II: $|P_0(t)|<|A|$ where the integrand experiences the sign flipping and the integral has a non-trivial solution. These cases can be seen in figure~\ref{fig:twoD_osc_cases}.

\begin{figure}[H]
\centering
\includegraphics[scale=.25]{twoD/osc_cases.jpg}
\caption{Here we have the parameter ranges for Case I and Case II shown as the right most green vertical line and the bifurcation at the left blue vertical line respectfully.}
\label{fig:twoD_osc_cases}
\end{figure}

\subsection{Case I: $P_0(t)\le -|A|$}
\label{subsec:twoD_highfreqosc_caseI}

We call this the entirely below axis case; here with the size of the solution $X_0$ being far enough from the axis to see no bifurcating behavior. This case helps to simplify the integration in \eqref{eq:twoD_osc_innerintegral} but also helps us to determine when the solution begins to act differently. Here we use the equilibria to define a boundary between Case I and Case II. Under the conditions of this case, the system \eqref{eq:twoD_osc_innerintegral} simplifies to

\begin{equation}\label{eq:twoD_osc_caseIsystem}
\begin{aligned}
{P_0}_t(t) =& -n -\eta_3P_0(t)-(1-\eta_3)Q_0(t),\\
{Q_0}_t(t) =& \eta_1P_0(t)-Q_0(t).
\end{aligned}
\end{equation}

Solving for the equilibria in \eqref{eq:twoD_osc_caseIsystem} results in

\begin{equation*}
Q_0(P_0)=\eta_1P_0\quad,P_0=-\frac{n}{\eta_1(1-\eta_3)+\eta_3}. 
\end{equation*}

Together with these equilibria and with the condition of the case, $P_0(t)\le -|A|$, we find the parameter range that distinguishes between Case I and Case II in terms of our inner parameter, which we then rewrite in original coordinates with

\begin{equation}\label{eq:twoD_osc_boundary}
\begin{aligned}
n\ge& (\eta_1(1-\eta_3)+\eta_3)|A|,\\
\eta_2\ge&\eta_1\eta_3+ \frac{(\eta_1(1-\eta_3)+\eta_3)|A|}{\Omega}.
\end{aligned}
\end{equation}

So for values of $\eta_2$ less that the boundary given in \eqref{eq:twoD_osc_boundary}, we begin to see the oscillations crossing above the axis and hence use a separate case to deal with this behavior.

\subsection{Case II: $|P_0(t)|<|A|$}
\label{subsec:twoD_highfreqosc_caseII}

We call this the crossing case; here the solution is beginning to oscillate about the axis while the center of the oscillations approach the axis. With this behavior, we expect a bifurcation to occur in this region and thus we must find the equilibria for \eqref{eq:twoD_osc_innerintegral} that will help determine the location. While this problem is two-dimensional in nature, the integral in \eqref{eq:twoD_osc_innerintegral} is nearly identical to the integral of difficulty in \autoref{sec:oneD_highfreqosc}. So we may use the ideas of that section here to form an approximate solution. Thus, under the assumptions of this case, we fix a value of $P_0$ and integrate \eqref{eq:twoD_osc_innerintegral} over the regions where the integrand take the same sign with

\begin{equation*}
R_1=\arccos(P_0/A),\quad R_2=2\pi-\arccos(P_0/A).
\end{equation*}

We make the same assumption in \autoref{sec:oneD_highfreqosc} that the solution to \eqref{eq:twoD_osc_innerintegral} is negative for $P_0(t)$ the region $R\in[0,R_1]$ and alternates sign for the regions $R\in (R_1,R_2]$ and $R\in (R_2,2\pi]$. With this assumption, we also follow the same procedure of integrating over each region to get the exact form for \eqref{eq:twoD_osc_innerintegral} with

\begin{equation}\label{eq:twoD_osc_caseIIexact}
\begin{aligned}
{P_0}_t=&-n- \eta_3 P_0(t)-(1-\eta_3)Q_0,\\
{Q_0}_t=&-\frac{2\eta_1}{\pi}\left(\arcsin(P_0/A)P_0+\sqrt{A^2-P_0^2}\right)-Q_0.
\end{aligned}
\end{equation}

Although this is the explicit inner equation from \eqref{eq:twoD_osc_caseIIexact}, this is analytically too complex to find an explicit form for any bifurcating behavior and thus we use a second order Taylor approximation to find the solvable system

\begin{equation}\label{eq:twoD_osc_taylor}
\begin{aligned}
{P_0}_t=&-n -\eta_3 P_0-(1-\eta_3)Q_0,\\
{Q_0}_t=&-\frac{2\eta_1|A|}{\pi}-Q_0-\frac{\eta_1}{\pi|A|}P_0^2.
\end{aligned}
\end{equation}

Recalling that the equilibria will lead to the bifurcation, we solve \eqref{eq:twoD_osc_taylor} to give the inner solution. For simplicity, define $a=\frac{\eta_1}{\pi|A|}$, and $ c=\frac{2\eta_1|A|}{\pi}$,

\begin{equation}\label{eq:twoD_osc_equilibria}
\begin{aligned}
Q_0(P_0)=&-aP_0^2-c,\\
0=-n+\frac{2\eta_1(|A|}{\pi}&-\eta_3 P_0+\frac{\eta_1}{\pi |A|}P_0^2.
\end{aligned}
\end{equation}

Where the equation for $P_0$ in \eqref{eq:twoD_osc_equilibria} is a quadratic that would have two solutions, we recall the lower branch as being what we follow for this analysis, so we choose the negative solution with

\begin{equation}\label{eq:twoD_osc_innersolution}
P_0=\frac{\eta_3}{2a(1-\eta_3)}- \frac{1}{2a(1-\eta_3)}\sqrt{\eta_3^2+4a(1-\eta_3)(n-c(1-\eta_3))}.
\end{equation}

With the equilibrium for $P_0$ in \eqref{eq:twoD_osc_innersolution} containing a square root, this fails once the discriminant becomes negative. It is with this sudden failure that we have the bifurcation, here occurring for

\begin{equation*}
n_{osc} = \frac{\eta_1(1-\eta_3)|A|}{\pi}\left[2-\left(\frac{\pi\eta_3}{2\eta_1(1-\eta_3)}\right)^2\right].
\end{equation*}

Where we have the inner equilibria found, we write \eqref{eq:twoD_osc_innersolution} along with the oscillations as well as the oscillatory bifurcation in original coordinates

\begin{equation}\label{eq:twoD_osc_innersolnoriginal}
\begin{aligned}
V(t)\sim& \Omega^{-1}\left(P_0-A\cos(\Omega t)\right),\\
T(t)\sim& \eta_1-\Omega^{-1}\left(\frac{\eta_1}{\pi|A|}P_0^2+\frac{2\eta_1|A|}{\pi}+B\cos(\Omega t)\right),
\end{aligned}
\end{equation}

\begin{equation}\label{eq:twoD_osc_bifurcation}
{\eta_2}_{osc} = \eta_1\eta_3+\frac{\eta_1(1-\eta_3)|A|}{\pi\Omega}\left[2-\left(\frac{\pi\eta_3}{2\eta_1(1-\eta_3)}\right)^2\right].
\end{equation}

With \eqref{eq:twoD_osc_bifurcation} we have found the bifurcation induced with the addition of oscillatory forcing in the Stommel model. As we learned from the one-dimensional model in \autoref{sec:oneD_highfreqosc}, the effect of oscillatory forcing is early bifurcations. Our result in \eqref{eq:twoD_osc_bifurcation} agrees with this heuristic under the caveat that we restrict the parameters with 

\begin{equation*}
\eta_3 <\frac{2\sqrt{2}\eta_1}{\pi+2\sqrt{2}\eta_1},
\end{equation*}

which is the condition to guarantee the second term in \eqref{eq:twoD_osc_bifurcation} is positive. This restriction is reasonable as generally the parameters have the behaviors of $\eta_3<1$ and $\eta_3\ll \eta_1$ where the thermal variation is much larger in real ocean dynamics than the ratio of relaxation times.

In figure~\ref{fig:twoD_osc_Vnumerics} the numerical solution to \eqref{eq:twoD_canonical} for $V$ and a zoom of the solution around the numerical bifurcation is shown. The static bifurcation diagram is underlayed as well for comparison. We contrast the result in \eqref{eq:twoD_osc_bifurcation} to these numerics and find the bifurcation prediction from our analysis agrees in $V$. Notice that there is an early bifurcation for this problem, where certain values of the lower branch are never achieved.
In figure~\ref{fig:twoD_osc_Tnumerics} the numerical solution to \eqref{eq:twoD_canonical} for $T$ and a zoom in around the bifurcation is shown. Once more, the static bifurcation diagram is underlayed and we have two interesting features to note. Due to the early bifurcation in $V$, the maximum value of $T$ is never reached in (b). But due to the early bifurcation, there is a region of both the lower and the upper branch in $V$ that is never followed and we see that range being skipped over by the numerics in (a).

\begin{figure}[H]
\centering
\begin{subfigure}{.5\textwidth}
  \centering
  \includegraphics[width=\linewidth]{twoD/osc_Vtimeseries.jpg}
  \caption{}
\end{subfigure}%
\begin{subfigure}{.5\textwidth}
  \centering
  \includegraphics[width=\linewidth]{twoD/osc_bif_diagram.jpg}
  \caption{}
\end{subfigure}
\begin{subfigure}{.5\textwidth}
  \centering
  \includegraphics[width=\linewidth]{twoD/osc_bif_diagram_zoom.jpg}
  \caption{}
\end{subfigure}
\caption{In (a) the numerical time series solutions to \eqref{eq:twoD_canonical} is given with parameters in each qualitatively different case of $\eta_2$ with $\eta_1=4$, $\eta_3=.375$, $A=10$ and $\Omega = 10$. In (b) these same solutions are shown on the phase plane. In (c) a zoom in closer to the non-smooth bifurcation region where the blue vertical line is the prediction \eqref{eq:twoD_slow_tipping} against the black dotted vertical line which is the numerical bifurcation.}
\label{fig:twoD_osc_Vnumerics}
\end{figure}

\begin{figure}[H]
\centering
\begin{subfigure}{.5\textwidth}
  \centering
  \includegraphics[width=\linewidth]{twoD/osc_Ttimeseries.jpg}
  \caption{}
\end{subfigure}%
\begin{subfigure}{.5\textwidth}
  \centering
  \includegraphics[width=\linewidth]{twoD/osc_bif_Tplot.jpg}
  \caption{}
\end{subfigure}
\begin{subfigure}{.5\textwidth}
  \centering
  \includegraphics[width=\linewidth]{twoD/osc_bif_Tplot_zoom.jpg}
  \caption{}
\end{subfigure}
\caption{In (a) we have the numerical time series solutions for a qualitatively different cases of $\eta_2$. In (b) we plot these solutions over the standard equilibrium plot for $V$ vs. $T$. In (c) a zoom of the bifurcation area.}
\label{fig:twoD_osc_Tnumerics}
\end{figure}

To evaluate the performance of this prediction, we compare \eqref{eq:twoD_osc_bifurcation} to the numerical tipping criteria over a range of $\Omega^{-1}$. In figure~\ref{fig:twoD_osc_epscomp} we allow for this range to be $\Omega^{-1}\in (0,.5)$. For small values, the two agree very well and as we expect, they begin to diverge once the values of $\Omega^{-1}$ become too large from the assumption that $\Omega\gg 1$ and the asymptotics cannot capture the behavior for low frequency oscillations. But under our assumptions, the prediction is performing quite well and resembles the performance of \autoref{sec:oneD_highfreqosc}.

\begin{figure}[H]
\centering
\includegraphics[scale=.25]{twoD/osc_Omegacomp.jpg}
\caption{The numerical tipping vs the estimate with $\eta_1=4$ and $\eta_3=\frac{3}{8}$. The tipping criteria is $V>.5$.}
\label{fig:twoD_osc_epscomp}
\end{figure}

\subsection{Stability}

Although the eigenvalues from \autoref{sec:twoD_slow} apply here, since we have a non-autonomous system when $A\not=0$, we must approach the stability with a linearized form about the equilibria much like in \autoref{sec:oneD_highfreqosc}. To do this, recall from our analysis that we found the system

\begin{equation}\label{eq:twoD_osc_stabilityequation}
\begin{aligned}
{P_0}_t =& -n -\eta_3 P_0-(1-\eta_3)Q_0,\\
{Q_0}_t =& -\frac{\eta_1}{2\pi}\int_0^{2\pi}|P_0-A\cos(R)|\, dR - Q_0.
\end{aligned}
\end{equation}

We must consider the stability of solutions over the relative sizes of $P_0(t)$ with Case I: $P_0(t)\le -|A|$ and Case II: $|)_0(t)|\le |A|$.

\subsubsection{Case I: $v_0(t)\le -|A|$}

From the analysis, we called this case the entirely below-axis case where the solution spends most of its time away from the axis $V=0$. We expected this case to be entirely controllable and thus we should see stability here. Under these conditions, the inner equation \eqref{eq:twoD_osc_stabilityequation} simplifies to a much simpler equation

\begin{equation}\label{eq:twoD_osc_stability_caseI_fullinner}
\begin{aligned}
{P_0}_t =& -m -\eta_3 P_0-(1-\eta_3)Q_0,\\
{Q_0}_t =& \eta_1 P_0 - Q_0.
\end{aligned}
\end{equation}

The equilibria of \eqref{eq:twoD_osc_stability_caseI_fullinner} is found with $Q_0(P_0)=\eta_1 P_0$, and thus we find the following one-dimensional equation and equilibrium

\begin{equation}\label{eq:twoD_osc_stability_caseI_red}
{P_0}_t = -n -(\eta_3 +(1-\eta_3))Q_0=f(P_0),\quad Z^0 = -\frac{n}{\eta_3+\eta_1(1-\eta_3)}.
\end{equation}

Now we consider a simple linear perturbation about this equilibrium with $P_0(t)= Z^0+U(t)$ where $\lVert U(t) \rVert \ll 1$. Our standard Taylor expansion about this equilibrium results in

\begin{equation}\label{eq:twoD_osc_stability_caseI_perteq}
\begin{aligned}
f(P_0)=&f(Z^0)+f_{P_0}(Z^0)(P_0-Z^0)+O((P_0-Z^0)^2),\\
U_t =& -(\eta_3+\eta_1(1-\eta_3))U + O(\lVert U\rVert ^2).
\end{aligned}
\end{equation}

From \eqref{eq:twoD_osc_stability_caseI_perteq} we now conclude the equilibrium $Z^0$ is hyperbolic and asymptotically stable due to the exponential decay in perturbations. Thus we find that no tipping will occur for this case, and from the analysis, this holds true until for the parameter range of $\eta_2$ from \eqref{eq:twoD_osc_boundary}

\begin{equation*}
\eta_2 \ge \eta_1\eta_3 +\frac{(\eta_3+\eta_1(1-\eta_3))|A|}{\Omega}.
\end{equation*}


\subsubsection{Case II: $|P_0(t)|<|A|$}

We called this case the crossing case and here the solution experiences all of the non-smooth behavior when it crosses $V=0$. We expect the stability to fail under these conditions and we discovered in the analysis that the crossing pattern caused the system \eqref{eq:twoD_osc_stabilityequation} to behave like

\begin{equation}\label{eq:twoD_osc_stability_caseII_full}
\begin{aligned}
{P_0}_t =& -n-\eta_3 P_0-(1-\eta_3)Q_0,\\
{Q_0}_t =& -\frac{2\eta_1|A|}{\pi}-\frac{\eta_1}{\pi |A|}P_0^2-Q_0.
\end{aligned}
\end{equation}

As we search for the equilibria of \eqref{eq:twoD_osc_stability_caseII_full}, we find the equilibrium for $Q_0$ in terms of $P_0$  

\begin{equation*}
Q_0(P_0)=-\frac{2\eta_1|A|}{\pi}-\frac{\eta_1}{\pi |A|}P_0^2
\end{equation*}

which then gives the following inner equation with the negatively chosen equilibrium for $P_0$

\begin{equation}\label{eq:twoD_osc_stability_caseII_inner}
\begin{aligned}
{P_0}_t=&-n+\frac{2\eta_1(|A|}{\pi}-\eta_3 P_0+\frac{\eta_1}{\pi |A|}P_0^2=f(P_0),\\
Z^0=&\frac{\pi |A|}{2\eta_1(1-\eta_3)}\left(\eta_3-\sqrt{n-n_{\text{osc}}}\right).
\end{aligned}
\end{equation}

For simplicity we write the argument of the square root in terms of oscillatory bifurcation we found in the analysis with \eqref{eq:twoD_osc_bifurcation}. We now consider a simple linear perturbation about this equilibrium in \eqref{eq:twoD_osc_stability_caseII_inner} with $P_0(t)=Z^0+U(t)$ where $\lVert U(t)\rVert\ll 1$. The standard Taylor expansion about the equilibrium is thus

\begin{equation}\label{eqLtwoD_osc_stability_caseII_perturb}
\begin{aligned}
f(P_0)=&f(Z^0)+f_{P_0}(Z^0)(P_0-Z^0)+O((P_0-Z^0)^2),\\
U_t=&\left(-\eta_3+\frac{2\eta_1(1-\eta_3)}{\pi |A|}Z^0\right)U,\\
U_t=&- \sqrt{n-n_{\text{osc}}}U.
\end{aligned}
\end{equation}

Thus with \eqref{eqLtwoD_osc_stability_caseII_perturb} we learn that the perturbations $U(t)$ decay exponentially as long as the square-root is non-zero and makes sense. Thus we have that $Z^0$ is a hyperbolic and asymptotically stable equilibrium for $P_0$ and thus we find stability in $Q_0$ as well. This gives stability for this region but we lose this stability once the square-root becomes zero, here when

\begin{equation*}
n_{osc} = \frac{\eta_1(1-\eta_3)|A|}{\pi}\left[2-\left(\frac{\pi\eta_3}{2\eta_1(1-\eta_3)}\right)^2\right].
\end{equation*}

This says the equilibrium $Z^0$ at this point is non-hyperbolic which is indicative of a bifurcation. This is in agreement with out analysis and thus we can say that the value found in \eqref{eq:twoD_osc_bifurcation} is the bifurcation under the oscillatory forcing.


\section{Slow Variation with Oscillatory Forcing}
\label{sec:twoD_slowosc}

With the one-dimensional model solved and both the slowly varying and high oscillatory two-dimensional components analyzed, we have all of the tools needed to analyze the full system \eqref{eq:twoD_canonical} with both $\epsilon \ll 1$ and $A,B\sim O(1)$ simultaneously. This is the most general setting we discuss in this paper where we account for both a slowly varying $\eta_2$ that leads to abrupt changes seen in \cite{alley2003abrupt,marotzke2000abrupt,rahmstorf2000thermohaline} as well as oscillatory forcing in both equations in \eqref{eq:twoD_canonical} seen in \cite{roberts2017relaxation,huybers2005obliquity}. We care to find the interaction of mechanisms in the physical Stommel model. Under the framework of slowly varying parameters we expect to find tipping instead of a bifurcation and hence our method for finding the tipping point follows a mixture of both \autoref{sec:twoD_slow} and \autoref{sec:twoD_highfreqosc}. This procedure dictates that we search for inner behavior about the non-smooth bifurcation and to do so we need to solve the inner equation and estimate when this solution becomes uncontrollable. Ultimately, we provide a realistic solution for where the abrupt change from the lower stable branch to the upper occurs in the full two-dimensional Stommel model.

To begin, we take our standard approach of following the lower branch towards the non-smooth behavior with $V<0$ in \eqref{eq:twoD_canonical} which gives the following system  

\begin{equation}\label{eq:twoD_slowosc_outereqs}
 \begin{aligned}
   \dot{V} & =  \eta_1-\eta_2-T+\eta_3(T-V)+V^2+A\sin(\Omega t), \\
   \dot{T} & =  \eta_1-T(1-V)+B\sin(\Omega t),  \\
  \dot{\eta_2}  & =  -\epsilon.
  \end{aligned}
\end{equation}

Much like in \autoref{sec:oneD_slowosc}, we assume there is a generic polynomial relationship between the frequency of our problem and the slow variation, $\Omega = \epsilon^{-\lambda}$ with strength parameter $\lambda>0$. It is this assumption that helps us to find the interaction between the slowly varying parameter and fast oscillations in the tipping. We notice in \eqref{eq:twoD_slowosc_outereqs} that there is behavior on a slow scale in $\eta_2(t)$ and on a fast scale in $\sin(\Omega t)$, so this suggests a multiple scales approach with 'slow' time $\tau = \epsilon t$ and 'fast' time $R=\epsilon^{-\lambda}t$. These scales cause \eqref{eq:twoD_slowosc_outereqs} to take the form

\begin{equation}\label{eq:twoD_slowosc_multiouter}
 \begin{aligned}
V_R+\epsilon^{\lambda+1}V_\tau & = \epsilon^{\lambda} \left(\eta_1-\eta_2-		T+\eta_3(T-V)+V^2+A\sin(R)\right), \\
T_R+\epsilon^{\lambda+1}T_\tau & = \epsilon^{\lambda}\left( \eta_1-T(1-		  V)+B\sin(R)\right),  \\
	{\eta_2}_\tau  & =  -1.
\end{aligned}
\end{equation}
  
From these multiple-scaled outer equations in \eqref{eq:twoD_slowosc_multiouter}, we find an asymptotic expansion in terms of our small recurring quantity $\epsilon^\lambda$ to be a good choice for separating the dynamics by order. But we must consider both multiples of $\lambda$ as well as integer powers as we have not specified the range of $\lambda$ and both could be significant, thus our expansion is

\begin{equation}\label{eq:twoD_slowosc_outerexpansion}
	\begin{aligned}
		V(\tau,R)\sim& V_0(\tau,R)+\epsilon^\lambda 	V_1(\tau,R)+O(\epsilon^{2\lambda},\epsilon^{\lambda+1}),\\
        T(\tau,R)\sim& T_0(\tau,R)+\epsilon^\lambda T_1(\tau,R)+O(\epsilon^{2\lambda},\epsilon^{\lambda+1}).
	\end{aligned}
\end{equation}

Which substituting \eqref{eq:twoD_slowosc_outerexpansion} into \eqref{eq:twoD_slowosc_multiouter} results in the governing dynamics for all orders of $\epsilon$ with

\begin{equation*}
\begin{aligned}
{V_0}_R+\epsilon^{\lambda+1}{V_0}_\tau+\epsilon^\lambda {V_1}_R+\ldots=&\begin{aligned}[t]&
\epsilon^\lambda \left(\eta_1-\eta_2-T+\eta_3(T_0-V_0)+V_0^2+A\sin(R)\right)\\
&+\epsilon^{2\lambda}\left(-\eta_3 V_1-(1-\eta_3)T_1+2V_0V_1\right)+\ldots
\end{aligned}\\
{T_0}_R+\epsilon^{\lambda+1}{T_0}_\tau+\epsilon^\lambda {T_1}_R+\ldots=&\begin{aligned}[t]&
\epsilon^\lambda\left( \eta_1-T_0(1-V_0)+B\sin(R)\right)\\
&+\epsilon^{2\lambda}\left(-T_1+T_0V_1+T_1V_0\right)+\ldots
\end{aligned}
\end{aligned}
\end{equation*}

Here we consider the next order in \eqref{eq:twoD_slowosc_outerexpansion} to be $O(\epsilon^{2\lambda})$ as the equations at $O(\epsilon^{\lambda+1})$ and $O(\epsilon^{2\lambda})$ result in the same information, so it does not matter whether we consider a $\lambda$ value that would cause $O(\epsilon^{2\lambda})$ to come before $O(\epsilon^{\lambda+1})$ or vice-versa. With this in mind, we find the following sets of equations at each order

\begin{align}
\label{eq:twoD_slowosc_outerO1}
O(1):\quad & \begin{cases}
	{V_0}_R =&  0, \\
	{T_0}_R =&  0,\\
\end{cases}\\
\label{eq:twoD_slowosc_outerO2}
O(\epsilon^\lambda):\quad & \begin{cases}
	{V_1}_R = & \eta_1-\eta_2(\tau) +\eta_3(T_0-V_0)-T_0+V_0^2+A\sin(R),\\
	{T_1}_R =&  \eta_1-T_0(1-V_0)+B\sin(R),
\end{cases}\\
\label{eq:twoD_slowosc_outerO3}
O(\epsilon^{2\lambda}):\quad & \begin{cases}
	{V_2}_R+\epsilon^{1-\lambda}{V_0}_\tau = & \eta_3(T_1-V_1)-T_1+2V_0V_1,\\
	{T_2}_R +\epsilon^{1-\lambda}{T_0}_\tau =&  -T_1(1-V_0)+T_0V_1,
\end{cases}
\end{align}

From \eqref{eq:twoD_slowosc_outerO1} the leading order terms in our expansion are purely slow dependent, $V_0=V_0(\tau)$ and $T_0=T_0(\tau)$. This allows the solutions of \eqref{eq:twoD_slowosc_outerO2} and \eqref{eq:twoD_slowosc_outerO3} to be found, the work for this is found in \autoref{app:twoD}. This results in the following outer solutions 

\begin{equation}\label{eq:twoD_slowosc_outersoln}
\begin{aligned}
V\sim& V_0 + \frac{\epsilon({V_0}_\tau(1-V_0)+(1-\eta_3){T_0}_\tau)}{(1-\eta_3)T_0+(2V_0-\eta_3)(1-V_0)}-\epsilon^\lambda A \cos(\Omega t),\\
T\sim& T_0 + \frac{\epsilon {T_0}_\tau}{1-V_0}-\frac{\epsilon T_0({V_0}_\tau(1-V_0)+(1-\eta_3){T_0}_\tau)}{(1-\eta_3)T_0(1-V_0)+(2V_0-\eta_3)(1-V_0)^2}-\epsilon^\lambda B \cos(\Omega t),
\end{aligned}
\end{equation}

where $V_0$ and $T_0$ are the same leading order solutions from the slowly varying Stommel model in \autoref{sec:twoD_slow}. Unfortunately the common theme of the two-dimensional model is that these outer solutions are too complex to see directly when an inner scaling is required so we perform a separate scale analysis analogous to that of \autoref{sec:oneD_slowosc} to find the appropriate inner scaling.

We make the assumption that the inner scaling for both $V$ and $T$ are the same which just makes this calculation slightly easier, but this isn't necessary to arrive at the same conclusion. Hence we chose a general scaling about the bifurcation point $(V,T,\eta_2)=(0,\eta_1,\eta_1\eta_3)$ with

\begin{equation}\label{eq:twoD_slowosc_general_scaling}
V=\epsilon^\alpha X, \quad T=\eta_1+\epsilon^\alpha Y ,\quad \eta_2(t)=\eta_1\eta_3+\epsilon^\beta n(t),
\end{equation}

where both $\alpha>0$ and $\beta>0$ allow for this to be an inner scaling. Applying the scalings in \eqref{eq:twoD_slowosc_general_scaling} to the full two-dimensional model \eqref{eq:twoD_canonical} gives

\begin{equation}\label{eq:twoD_slowosc_innerscaled}
\begin{aligned}
\epsilon^\alpha \dot{X}=& -\epsilon^\beta n(t)-\epsilon^\alpha (X+(1-\eta_3)Y) - \epsilon^{2\alpha}X|X| +A\sin(\epsilon^{-\lambda}t),\\
\epsilon^\alpha \dot{Y}=&-\epsilon^\alpha(\eta_1|X|+Y)+\epsilon^{2\alpha}|X|Y +B\sin(\epsilon^{-\lambda} t)\\
\dot{n}=&-\epsilon^{1-\beta}.
\end{aligned}
\end{equation}

From \eqref{eq:twoD_slowosc_innerscaled} it is apparent that fast behavior is still occurring on different time scales and to truly flesh out the particular choice in $\alpha$, we then take a multiple scales approach to capture the 'fast' behavior with scales $t$ and $R=\epsilon^{-\lambda}t$. Also note that we choose to have the 'slow' behavior has been moved into regular time with the scalings we applied to the space variables. This choice comes with the ambiguity in $\beta$ and is discussed further below. Applying the multiple scales in \eqref{eq:twoD_slowosc_innerscaled} results in

\begin{equation}\label{eq:twoD_slowosc_innergeneral}
\begin{aligned}
\epsilon^{\alpha-\lambda} X_R+\epsilon^{\alpha}X_t=& -\epsilon^{\beta}n(t)-\epsilon^{\alpha}(X+(1-\eta_3)Y)-\epsilon^{2\alpha}X|X|+A\sin(R),\\
\epsilon^{\alpha-\lambda}Y_R + \epsilon^{\alpha}Y_t =&- \epsilon^\alpha(\eta_1|X|+Y)-\epsilon^{2\alpha}|X|Y +B\sin(R)\\
n_t=&-\epsilon^{1-\beta}.
\end{aligned}
\end{equation}

Here we balance the leading order terms in each equation of \eqref{eq:twoD_slowosc_innergeneral}, $\epsilon^{\alpha-\lambda}X_R$ and $A\sin(R)$ as well as $\epsilon^{\alpha-\lambda}Y_R$ with $B\sin(R)$, which gives us that $\alpha=\lambda$ and confirms that the scales for each variable are the same. The scaling for $\eta_2$ has still yet to be determined and could have multiple possibilities depending on $\lambda$, but due to this choice in $\alpha$ we expect the oscillatory term to persist in the inner asymptotic expansion of \eqref{eq:twoD_canonical} regardless of choice in $\lambda$ and we have an effective means of tracking this behavior with this scaling.

We now consider the scales $t$ and $R=\epsilon^{-\lambda}t$ on the canonical system \eqref{eq:twoD_canonical} along with the general scaling on $\eta_2$ which gives

\begin{equation}\label{eq:twoD_slowosc_general_outermulti}
\begin{aligned}
V_R+\epsilon^{\lambda}V_t =& -\epsilon^{\lambda+\beta}n(t)-\epsilon^{\lambda}(\eta_1-\eta_1\eta_3+\eta_3(T-V)-T-V|V|+A\sin(R)),\\
T_R+\epsilon^{\lambda}T_t =& \epsilon^\lambda(\eta_1-T(1+|V|)+B\sin(R)),\\
n_t =&-\epsilon^{1-\beta}.
\end{aligned}
\end{equation}

We learned in \autoref{sec:oneD_slowosc} that the analysis from here depends on the relative size of the slow variation with respects to the oscillations. The distinction in these behaviors are when $\lambda\le1$ where a mixture between the slow variation and oscillations occur or $\lambda>1$ where the slow variation dominates the solution. We then consider a separate asymptotic expansion for the following Case I: $\lambda\le 1$ and Case II: $\lambda >1$ to find an accurate classification of behavior for the full two-dimensional Stommel model.

\subsection{Case I: $\lambda \le 1$}

We call this the mixed effects case where there is significant influence from both slow variation and fast oscillations due to the size of $\lambda$. Under this range for $\lambda$, there are two choices for the scaling on $\eta_2$, $\beta=1$ or $\beta=\lambda$. But we are unsure whether there should be integer powers in an asymptotic expansion due to the uncertainty of the scale $\beta$. Thus we choose a rather general expansion with

\begin{equation}\label{eq:twoD_slowosc_caseI_expansion}
\begin{aligned}
V(t,R)\sim& \epsilon^{\lambda} X_0(t,R)+\epsilon^q X_1(t,R)+\ldots\\
T(t,R)\sim& \eta_1+\epsilon^{\lambda} Y_0(t,R)+\epsilon^q Y_1(t,R)+\ldots
\end{aligned}
\end{equation}

with $q>\lambda$ to be consist with our analysis thus far. Substituting \eqref{eq:twoD_slowosc_caseI_expansion} into \eqref{eq:twoD_slowosc_general_outermulti} then gives the governing dynamics for this case

\begin{equation*}
\begin{aligned}
 {X_0}_R+\epsilon^{\lambda}{X_0}_t+\epsilon^{q-\lambda} {X_1}_R\ldots={} & -\epsilon^{\beta}n(t)-\epsilon^{\lambda} (\eta_3X_0+(1-\eta_3)Y_0) \\
&-\epsilon^{2\lambda}(X_0+\epsilon^{q-\lambda} X_1+\ldots)|X_0+\epsilon^{q-\lambda} X_1+\ldots|\\
& - \epsilon^{q}(\eta_3X_1+(1-\eta_3)Y_1) + A\sin(R) +\ldots
\end{aligned}
\end{equation*}

\begin{equation*}
\begin{aligned}
{Y_0}_R+\epsilon^{\lambda}{Y_0}_t+\epsilon^{q-\lambda} {Y_1}_R+\ldots= &-\epsilon^\lambda(\eta_1| X_0 +\epsilon^{q-\lambda} X_1+\ldots|+ Y_0+\epsilon^{q-\lambda} Y_1+\ldots)\\
&+\epsilon^{2\lambda}|X_0 +\epsilon^{q-\lambda} X_1+\ldots|(Y_0+\epsilon^{q-\lambda} Y_1+\ldots)\\
&+ B\sin (R)+\ldots
\end{aligned}
\end{equation*}

Where we now separate by the distinct orders of $\epsilon$ to find the following equations at each order

\begin{align} \label{eq:twoD_slowosc_caseI_O1}
O(1):\, &\begin{cases}
	{X_0}_R =&  A\sin(R), \\
	{Y_0}_R =&  B\sin(R),\\
\end{cases}\\ \label{eq:twoD_slowosc_caseI_O2}
O(\epsilon^\lambda): \, & \begin{cases}
	\epsilon^{q-2\lambda}{X_1}_R+{X_0}_t =&  -\epsilon^{\beta-\lambda} n(t) -\eta_3 X_0-(1-\eta_3)X_0, \\
	\epsilon^{q-2\lambda}{Y_1}_R+{Y_0}_t =&  -\eta_1|X_0|-Y_0.\\
\end{cases}
\end{align}

We learn from \eqref{eq:twoD_slowosc_caseI_O2} that $q= 2\lambda$ prevents terms from being unbalanced, which implies that $\lambda> \frac{1}{2}$ for an expansion to be found, otherwise we would need to include the quadratic terms at $O(\epsilon^\lambda)$ and our equations would be too complicated to solve analytically. This $q$ also suggests that there is no need for integer powers in the expansion \eqref{eq:twoD_slowosc_caseI_expansion} and all behavior is captures by the interaction between the the slow variation and frequency, $\epsilon^\lambda$. Since there is a choice in the scaling for $\eta_2$ where $\beta=\lambda$ or $\beta=1$ we must choose here to dictate the future of the analysis. The advantage to choosing $\beta=\lambda$ is that these inner equations are rather simple, but the slow variation is still small. On the other hand, $\beta=1$ keeps a small coefficient on the parameter but makes the slow variation simple. Both of these choices result in the same equations, so here we choose $\beta=1$ for simplicity. From \eqref{eq:twoD_slowosc_caseI_O1} we find the appropriate form of the leading order terms, $X_0=P_0(t)-A\cos(R)$ and $Y_0=Q_0(t)-B\cos(R)$. Using these forms for the leading order term and applying the Fredholm alternative \eqref{eq:Fredholm} to \eqref{eq:twoD_slowosc_caseI_O2} we find 

\begin{equation}\label{eq:twoD_slowosc_caseI_fullinner}
\begin{aligned}
{P_0}_t =&  -\epsilon^{1-\lambda} n(t) -\eta_3 P_0-(1-\eta_3)Q_0, \\
{Q_0}_t =&  -\frac{\eta_1}{2\pi}\int_0^{2\pi}|P_0(t)-A\cos(R)|\,dR-Q_0,\\
n_t =& -1.
\end{aligned}
\end{equation}

We learned from \autoref{sec:twoD_osc} that we must approach the integration with the relative size of $P_0(t)$ to the amplitude of oscillation $A$ in mind as these sizes determine the difficulty of integration in \eqref{eq:twoD_slowosc_caseI_fullinner}. We consider these sizes of $P_0(t)$ as Sub-case I: $P_0(t)\le -|A|$ and Sub-Case II:$|P_0(t)|<|A|$ much like in \autoref{sec:twoD_highfreqosc}. These cases keep the integrand from ever crossing the axis or consider the integrand as its allowed to cross the axis respectively.

\subsubsection{Sub-Case I: $P_0(t)\le -|A|$}

We call this the below-axis sub-case where the solution $P_0(t)$ is entirely below the axis $V=0$ and the full solution $X_0$ has center of oscillations far from crossing. Under these conditions we don't expect any tipping behavior as the solution is far from the non-smooth behavior, but we may use this to find the range of $\eta_2$ that distinguishes these cases. With $P_0(t)\le -|A|$, we find \eqref{eq:twoD_slowosc_caseI_fullinner} simplifies to

\begin{equation}\label{eq:twoD_slowosc_subcaseI_full}
\begin{aligned}
{P_0}_t =&  -\epsilon^{1-\lambda} n(t) -\eta_3 P_0-(1-\eta_3)Q_0, \\
{Q_0}_t =&  \eta_1 P_0-Q_0.
\end{aligned}
\end{equation}

We have the means available to solve \eqref{eq:twoD_slowosc_subcaseI_full} as it takes the form of an equation we have seen in \autoref{sec:twoD_slow}, but instead we search for when the pseudo-equilibrium fails the assumption of this sub-case. This results in the parameter range between these sub-cases and taking this approach is more convenient than solving the system. Here the form of the pseudo-equilibria is simple to find as $Q_0(P_0) = \eta_1P_0$ and thus 

\begin{equation*}
P_0(t) = -\epsilon^{1-\lambda}\frac{n(t)}{\eta_3+\eta_1(1-\eta_3)}.
\end{equation*}

But we recall that for this sub-case $P_0(t)\le -|A|$, which gives the range for $n$ and we rewrite this in the original coordinates of $\eta_2$ with

\begin{equation}\label{eq:twoD_slowosc_subcaseboundary}
\begin{aligned}
\epsilon n \ge & \epsilon^\lambda (\eta_3+\eta_1(1-\eta_3))|A|,\\
\eta_2 \ge& \eta_1\eta_3 +\frac{(\eta_3+\eta_1(1-\eta_3))|A|}{\Omega}.
\end{aligned}
\end{equation}

With the parameter range \eqref{eq:twoD_slowosc_subcaseboundary}, we now have an effective region for Sub-Case I and know when the crossings of Sub-Case II begin in terms of the parameter $\eta_2$.

\subsubsection{Sub-Case II: $|P_0(t)|\le |A|$}

We call this the crossing sub-case and under these conditions we see more complexity arise from the integral in \eqref{eq:twoD_slowosc_caseI_fullinner}. As the crossings continue, there is an increasing effect on the system and it is here that we anticipate the tipping to occur. In \autoref{sec:oneD_slowosc}, we found a similar integral to \eqref{eq:twoD_slowosc_caseI_fullinner} that we could evaluate with the assumption that $A\sim O(1)$. We have that the assumptions that allowed for the integration to make sense in the one-dimensional model still hold here with a 'fast' time $R$ that is sufficiently large due to the high frequency. We then follow the approach from the one-dimensional model by integrating with $R_1=\arccos(P_0/A)$ and $R_2 = 2\pi-\arccos(P_0/A)$ and then taking a quadratic Taylor approximation to find the system

\begin{equation}\label{eq:twoD_slowosc_subcaseII_taylor}
\begin{aligned}
{P_0}_t =& -\epsilon^{1-\lambda}n(t)-\eta_3 P_0(s)-(1-\eta_3)Q_0\\
{Q_0}_t =&-\frac{2\eta_1|A|}{\pi}-\frac{\eta_1}{\pi|A|}P_0^2-Q_0.
\end{aligned}
\end{equation}

Where in it's current form \eqref{eq:twoD_slowosc_subcaseII_taylor} is known as a quadratic two-dimensional Riccati-type equation which are notoriously difficult to solve analytically. We recall that allowing for the solution to the equation for $T$ to be in terms of $V$ was realistic to the delayed behavior of the THC, so any behavior that we are interested in lies within the dynamics for $V$, or it's inner counterpart $X$. For this reason, we choose to reduce the system \eqref{eq:twoD_slowosc_subcaseII_taylor} to a one-dimensional model by assuming our equation for $Q_0$ is in pseudo-equilibrium with 

\begin{equation}\label{eq:twoD_slowosc_subcaseII_equilreduction}
{Q_0}(P_0) =  -\frac{2\eta_1|A|}{\pi}-\frac{\eta_1}{\pi|A|}P_0^2.
\end{equation}

The resulting reduced one-dimensional system from introducing the equilibrium \eqref{eq:twoD_slowosc_subcaseII_equilreduction} into the leading order inner equation \eqref{eq:twoD_slowosc_subcaseII_taylor} is then

\begin{equation}\label{eq:twoD_slowosc_subcaseII_reducedeq}
\begin{aligned}
{P_0}_t =& -\epsilon^{1-\lambda}n(t)+\frac{2\eta_1(1-\eta_3)|A|}{\pi}-\eta_3P_0+\frac{\eta_1}{\pi|A|}P_0^2,\\
n_t=&-1.
\end{aligned}
\end{equation}

To relate the slow variation directly to the solution of \eqref{eq:twoD_slowosc_subcaseII_reducedeq}, we swap the differentiation to be with respect to the parameter giving 

\begin{equation} \label{eq:twoD_slowosc_subcaseII_reducedn}
\begin{aligned}
{P_0}_n =& \epsilon^{1-\lambda} n -\frac{2\eta_1(1-\eta_3)|A|}{\pi}+\eta_3 P_0-\frac{\eta_1(1-\eta_3)}{\pi|A|}P_0^2.\\
\end{aligned}
\end{equation}

Now \eqref{eq:twoD_slowosc_subcaseII_reducedn} is in a form that the result from Zhu \& Kuske \eqref{eq:intro_Zhuresult} has the ability to solve. Thus we determine that \eqref{eq:twoD_slowosc_subcaseII_reducedn} is an Airy-type equation and that it's tipping follows with \eqref{eq:intro_Zhuresult}. We promptly write this into original coordinates and notice the relationship to tipping and bifurcating values we've found in previous sections with

\begin{equation}\label{eq:twoD_slowosc_subcaseII_tipping}
\begin{aligned}
n_{\text{tip}} =& -\epsilon^{(\lambda-1)/3}\left(\frac{\pi|A|}{\eta_1(1-\eta_3)}\right)^{1/3}(2.33810)+\epsilon^{\lambda-1}\frac{\eta_1(1-\eta_3)|A|}{\pi}\left(2-\left(\frac{\pi\eta_3}{2\eta_1(1-\eta_3)}\right)^2\right),\\
{\eta_2}_{\text{tip}} =& \epsilon^{(\lambda-1)/3}\left(\frac{\pi|A|}{\eta_1(1-\eta_3)}\right)^{1/3}\mu_{\text{Airy}}+{\eta_2}_{\text{osc}}
\end{aligned}
\end{equation}

We conclude that our tipping in \eqref{eq:twoD_slowosc_subcaseII_tipping} follows closely to the tipping found with \eqref{eq:oneD_slowosc_caseItipping} in \autoref{sec:oneD_slowosc} where we found a weighted average between the smooth tipping and the oscillatory bifurcation for this range of $\lambda$. We also discovered along the way that any $\lambda\le\frac{1}{2}$ causes fundamentally different inner equations which lead to unsolvable and incoherent asymptotics under this approach. This heuristically makes sense as for $\lambda\le \frac{1}{2}$ we have low frequency oscillations with our polynomial relationship and the contributions to the dynamics from this behavior require a different approach than presented in this paper. For more, see Zhu \& Kuske \cite{zhu2015tipping} for an example of a low-frequency method.

\subsection{Case II: $\lambda>1$}

We call this case the slow dominate case; here we expect integer powers of $\epsilon$ to appear due to the $O(\epsilon^\lambda)$ being quite small for this range of $\lambda$ and thus we choose the expansion 

\begin{equation}\label{eq:twoD_slowosc_caseII_expansion}
\begin{aligned}
V(t,R) \sim& \epsilon X_0(t,R)+\epsilon^\lambda X_1(t,R)+\epsilon^q X_2(t,R)+\ldots\\
T(t,R) \sim& \epsilon Y_0(s,R) + \epsilon^\lambda Y_1(t,R) +\epsilon^q Y_2(t,R)+\ldots
\end{aligned}
\end{equation}

Substituting \eqref{eq:twoD_slowosc_caseII_expansion} into \eqref{eq:twoD_slowosc_general_outermulti} then gives the full dynamics of this system with their respective orders of $\epsilon$

\begin{equation*}
\begin{aligned}
\epsilon {X_0}_R+\epsilon^{\lambda+1}{X_0}_t+\epsilon^\lambda {X_1}_R+\ldots={} & -\epsilon^{\lambda+\beta}n(t)-\epsilon^{\lambda+1} (\eta_3X_0+(1-\eta_3)Y_0) \\
&-\epsilon^{\lambda+2}(X_0+\epsilon^{q-\lambda} X_1+\ldots)|X_0+\epsilon^{q-\lambda} X_1+\ldots|\\
& - \epsilon^{2\lambda}(\eta_3X_1+(1-\eta_3)Y_1) + \epsilon^\lambda A\sin(R),
\end{aligned}
\end{equation*}

\begin{equation*}
\begin{aligned}
\epsilon {Y_0}_R+\epsilon^{\lambda+1}{Y_0}_t+\epsilon^\lambda {Y_1}_R+\ldots=& -\epsilon^{\lambda+1}(\eta_1|X_0 +\epsilon^{\lambda-1} X_1+\epsilon^{q-1} X_2+\ldots|- Y_0-\epsilon^{\lambda-1} Y_1+\ldots)\\
&+\epsilon^2|X_0 +\epsilon^{\lambda-1} X_1+\epsilon^{q-1} X_2+\ldots|(Y_0 +\epsilon^{\lambda-1} Y_1+\ldots)\\
&+\epsilon^\lambda B \sin (R).
\end{aligned}
\end{equation*}

Where we separate by each distinct order of $\epsilon$ to find the following equations at each order

\begin{align} \label{eq:twoD_slowosc_caseII_O1}
O(\epsilon):\, &\begin{cases}
	{X_0}_R =&  0, \\
	{Y_0}_R =&  0,\\
\end{cases}\\ \label{eq:twoD_slowosc_caseII_O2}
O(\epsilon^\lambda): \, & \begin{cases}
	{X_1}_R =&  A\sin(R), \\
	{Y_1}_R =&  B\sin(R),\\
\end{cases}\\
\label{eq:twoD_slowosc_caseII_O3}
O(\epsilon^{\lambda+1}):\, &\begin{cases}
	\epsilon^{q-\lambda-1}{X_2}_R+{X_0}_t =&  -\epsilon^{\beta-1}n(t)-\eta_3X_0-(1-\eta_3)Y_0, \\
	\epsilon^{q-\lambda-1}{Y_2}_R+{Y_0}_t =&  -\eta_1|\epsilon^{\lambda}|X_0+\epsilon^{\lambda-1}X_1|-Y_0,\\
\end{cases}
\end{align}

We learn in \eqref{eq:twoD_slowosc_caseII_O3} that $q=\lambda+1$ prevents terms from becoming trivial or unbalanced and thus we choose this. We also find $\beta =1$ that prevents triviality contrary to Case I where we chose the value of $\beta$ for convenience. In \eqref{eq:twoD_slowosc_caseII_O1} we find that the leading order behavior for this case is purely slow, $X_0=X_0(t)$ and $Y_0=Y_0(t)$, thus giving the slow domination of this case. We are able to extract the oscillatory forcing into $X_1$ and $Y_1$ with \eqref{eq:twoD_slowosc_caseII_O2} and since the slow behavior in $X_1$ and $Y_1$ are just next order corrections to the purely slow $X_0$ and $Y_0$, without loss of generality we allow the slow behavior to be expressed by $X_0$ and $Y_0$; thus we have purely oscillatory corrections, $X_1=-A\cos(R)$ and $Y_1=-B\cos(R)$. Applying Fredholm \eqref{eq:Fredholm} to \eqref{eq:twoD_slowosc_caseII_O3} then gives

\begin{equation}\label{eq:twoD_slowosc_caseII_fullinner}
\begin{aligned}
{X_0}_t =& -n(t) -\eta_3 X_0- (1-\eta_3)Y_0,\\
{Y_0}_t =& -\frac{\eta_1}{2\pi}\int_0^{2\pi}|X_0(t)-\epsilon^{\lambda-1}A\cos(R)|\,dR -Y_0,\\
n_t=& -1.
\end{aligned}
\end{equation}

From Case I, we used the pseudo-equilibrium of $Q_0$ for an integral of this type regardless of the size of the oscillations to find a solvable equation. Here we expect \eqref{eq:twoD_slowosc_caseII_fullinner} to have some kind of quadratic form like in Case I, so we choose a priori to reduce this into a one-dimensional problem by assuming the equation for $Y_0$ is in it's pseudo-equilibrium with

\begin{equation}\label{eq:twoD_slowosc_caseII_equilreduction}
{Y_0}(X_0)= -\frac{\eta_1}{2\pi}\int_0^{2\pi}|X_0-\epsilon^{\lambda-1}A\cos(R)|\,dR.
\end{equation}

We find the resulting reduced one-dimensional equation by introducing \eqref{eq:twoD_slowosc_caseII_equilreduction} into the full inner equation \eqref{eq:twoD_slowosc_caseII_fullinner} with

\begin{equation}\label{eq:twoD_slowosc_caseII_reducedeq}
{X_0}_t = -n(t)-\eta_3 X_0+\frac{\eta_1(1-\eta_3)}{2\pi}\int_0^{2\pi}|X_0(t)-\epsilon^{\lambda-1}A\cos(R)|\,dR.
\end{equation}

Where in \eqref{eq:twoD_slowosc_caseII_reducedeq} the behavior is very similar to Case I as long as the amplitude of oscillations inside the integral are still manageable with our assumptions from that case (i.e $\epsilon^{\lambda-1}A \sim O(1)$). This indicates that $\lambda \approx 1$ to see mixed behavior of Case I and to see this similarity, we once more follow the method of \autoref{sec:oneD_slowosc}. Our assumption on the size of the oscillations allow us to integrate \eqref{eq:twoD_slowosc_caseII_reducedeq} with $R_1= \arccos(x_0/\epsilon^{\lambda-1}A)$ and $R_2 = 2\pi - \arccos(x_0/\epsilon^{\lambda-1}A)$. Another application of a quadratic Taylor approximation then yields

\begin{equation}\label{eq:twoD_slowosc_caseII_taylor}
\begin{aligned}
{X_0}_t = -n(t) +\epsilon^{\lambda-1}\frac{2\eta_1(1-\eta_3)|A|}{\pi}-\eta_3 X_0 +\epsilon^{1-\lambda}\frac{\eta_1(1-\eta_3)}{\pi |A|}X_0^2.
\end{aligned}
\end{equation}

Where we again find a form that we apply the result from Zhu \& Kuske in \eqref{eq:intro_Zhuresult} to. Thus we find the tipping for the scaled parameter $n$ and then transform back into the original coordinates for tipping in $\eta_2$ with

\begin{equation}
\begin{aligned}
n_{\text{mixed}}=&-\epsilon^{(\lambda-1)/3}\left(\frac{\pi|A|}{\eta_1(1-\eta_3)}\right)^{1/3}(2.33810)+\epsilon^{\lambda-1}\frac{\eta_1(1-\eta_3)|A|}{\pi}\left(2-\left(\frac{\pi\eta_3}{2\eta_1(1-\eta_3)}\right)^2\right),\\
{\eta_2}_{\text{mixed}}=& \epsilon^{(\lambda-1)/3}\left(\frac{\pi|A|}{\eta_1(1-\eta_3)}\right)^{1/3}\mu_{\text{smooth}}+{\eta_2}_{\text{osc}}.
\end{aligned}
\end{equation}

On the other hand, when $\lambda$ becomes large, these oscillations become more insignificant, and inside the integral in \eqref{eq:twoD_slowosc_innergeneral} their contribution gets weaker. For sufficiently large $\lambda$, here these values range from approximately $2\le\lambda \le 4$ depending on other model parameters, \eqref{eq:twoD_slowosc_caseII_fullinner} begins to act like

\begin{equation}\label{eq:twoD_slowosc_caseII_sloweq}
\begin{aligned}
{X_0}_t =& -n(t)-\eta_3 X_0 -(1-\eta_3)Y_0,\\
{Y_0}_t =&-\eta_3|X_0|-Y_0,\\
n_t  =&-1.
\end{aligned}
\end{equation}

Where \eqref{eq:twoD_slowosc_caseII_sloweq} is the same system as the purely slow model in \autoref{sec:twoD_slow}. Since this is the same inner behavior and we still have slow variation, we are able to use the approximation found there for the tipping \eqref{eq:twoD_slow_tipping}. This indicates that the oscillations decay to a point where only the slow variation effects the tipping of the Stommel model.

With both Case I and Case II, we have the tipping behavior for any choice in $\lambda$. For $\lambda\le1$, we found similar averaging between the oscillatory bifurcation and the smooth tipping as in \autoref{sec:oneD_slowosc}. This behavior continued even past $\lambda=1$ but the oscillatory contribution begins to contribute less to the over all tipping. Once $\lambda$ was sufficiently large, the oscillations all but die off in their contribution and we recover the purely slow tipping. This gives us an effective description of the tipping for the most general version of the Stommel model and we recap this in the following table.

\begin{table}[H]
\begin{center}
\begin{tabular}{|c|c|}
\hline 
 \multicolumn{2}{|c|}{Two-Dimensional Tipping} \\ 
\hline
Slow: & ${\eta_2}_{\text{slow}}=\min(\eta_1\eta_3 -\epsilon\log(\epsilon)/\lambda_i)$ for $i\in\{1,2\}$ \\ 
\hline 
High Freq. Osc: & ${\eta_2}_{\text{osc}}=\eta_1\eta_3+\frac{\eta_1(1-\eta_3)|A|}{\pi\Omega}\left(2-\left(\frac{\pi\eta_3}{2\eta_1(1-\eta_3)}\right)^2\right)$ \\ 
\hline 
Slowly Oscillatory $\lambda\le 1$: & ${\eta_2}_{\text{mixed}}=\epsilon^{(\lambda-1)/3}\left(\frac{\pi |A|}{\eta_1(1-\eta_3)}\right)^{1/3} \mu_{\text{smooth}}+{\eta_2}_{\text{osc}}$ \\ 
\hline 
Slowly Oscillatory $\lambda >1$ and $\lambda \approx 1$: &${\eta_2}_{\text{mixed}}=\epsilon^{(\lambda-1)/3}\left(\frac{\pi |A|}{\eta_1(1-\eta_3)}\right)^{1/3} \mu_{\text{smooth}}+{\eta_2}_{\text{osc}}$ \\ 
\hline 
Slowly Oscillatory $\lambda>1$:
 & ${\eta_2}_{\text{slow}}=\min(\eta_1\eta_3 -\epsilon\log(\epsilon)/\lambda_i)$ for $i\in\{1,2\}$ \\
\hline
\end{tabular} 
\caption{The tipping of the two-dimensional model for each mechanism and case.}
\end{center}
\end{table}

In figure~\ref{fig:twoD_slowosc_Vnumerics_small}, we see an example of the numerical solution of $V$ to the canonical system \eqref{eq:twoD_canonical} with slow variation and oscillatory forcing. This example has tipping occurring in Case I due to $\lambda\in (\frac{1}{2},1]$ producing a mixed effect from both the slow variation and oscillations on the tipping. The vertical lines are the tipping, black solid for the numerical and blue dotted for the approximation for this case \eqref{eq:twoD_slowosc_subcaseII_tipping}. Although there is a mixture of effects, the tipping still is occurring in the region near the oscillatory bifurcation. This tells us that for these choices in the model parameters that the strongest effect is the oscillatory forcing. This also is shown in figure \eqref{fig:twoD_slowosc_Tnumerics_small} which has the numerical solution of $T$ plotted against $V$. Here we see that due to the early tipping in $V$, the solution for $T$ also never achieves it's maximum and there is early tipping here as well, which agrees with the assumptions we had made of considering $T$ responding to $V$. We also see that there are areas of the stable branch that is skipped over from the early tipping, and even the oscillations cross the unstable branch frequently near the tipping, which is consistent with the crossing behavior in the numerical solution for $V$.

\begin{figure}[H]
\centering
\begin{subfigure}{.5\textwidth}
  \centering
  \includegraphics[width=\linewidth]{twoD/slowosc_bif_diagram_small.jpg}
  \caption{}
\end{subfigure}%
\begin{subfigure}{.5\textwidth}
  \centering
  \includegraphics[width=\linewidth]{twoD/slowosc_bif_diagram_small_zoom.jpg}
  \caption{}
\end{subfigure}
\caption{Model values are $\lambda=.8$, $\epsilon=.01$ with $A=B=2$. In (a) the numerical solution (black dotted line) to \eqref{eq:twoD_canonical} is given with $\eta_1=4$, $\eta_3=.375$. In (b) a zoom in closer to the non-smooth bifurcation region where the blue vertical line is the tipping prediction against the black dotted vertical line which is the numerical bifurcation.}
\label{fig:twoD_slowosc_Vnumerics_small}
\end{figure}

\begin{figure}[H]
\centering
\begin{subfigure}{.5\textwidth}
  \centering
  \includegraphics[width=\linewidth]{twoD/slowosc_Tplot_small.jpg}
  \caption{}
\end{subfigure}%
\begin{subfigure}{.5\textwidth}
  \centering
  \includegraphics[width=\linewidth]{twoD/slowosc_Tplot_small_zoom.jpg}
  \caption{}
\end{subfigure}
\caption{Model values are $\lambda=.8$, $\epsilon=.01$ with $A=B=2$. In (a) we have the numerical solution (black dotted) over the standard equilibrium plot for $V$ vs. $T$. In (b) a zoom of the bifurcation area.}
\label{fig:twoD_slowosc_Tnumerics_small}
\end{figure}

In figure~\ref{fig:twoD_slowosc_Vnumerics_medium} we have chosen a value of $\lambda$ that causes the tipping to fall into Case II as $\lambda>1$ but this choice is close to the boundary and thus we see comparable behavior to Case I with the addition that the slow variation is now dominant as in \autoref{sec:twoD_slow}. Upon a zoom in, it is apparent that oscillations are still present and this is where we see the mixture of effects that cause a similar tipping to take place. We've plotted the purely slow tipping as the green vertical dotted line for comparison. As the tipping occurs near the purely slow tipping this confirms that the slow variation is indeed dominating the tipping. In figure~\ref{fig:twoD_slowosc_Tnumerics_medium} we again see very similar behavior to the purely slow model in \autoref{sec:twoD_slow} but the zoom in further reveals the oscillations are present and have minor influence by forcing the solution to cross the unstable branch near the tipping.

\begin{figure}[H]
\centering
\begin{subfigure}{.5\textwidth}
  \centering
  \includegraphics[width=\linewidth]{twoD/slowosc_bif_diagram_medium.jpg}
  \caption{}
\end{subfigure}%
\begin{subfigure}{.5\textwidth}
  \centering
  \includegraphics[width=\linewidth]{twoD/slowosc_bif_diagram_medium_zoom.jpg}
  \caption{}
\end{subfigure}
\caption{Model values are $\lambda=1.3$, $\epsilon=.01$ with $A=B=2$. In (a) the numerical solution (black dotted line) to \eqref{eq:twoD_canonical} is given with $\eta_1=4$ and $\eta_3=.375$. In (b) a zoom in closer to the non-smooth bifurcation region where the blue dotted vertical line is the prediction against the black solid vertical line which is the numerical bifurcation.}
\label{fig:twoD_slowosc_Vnumerics_medium}
\end{figure}

\begin{figure}[H]
\centering
\begin{subfigure}{.5\textwidth}
  \centering
  \includegraphics[width=\linewidth]{twoD/slowosc_Tplot_medium.jpg}
  \caption{}
\end{subfigure}%
\begin{subfigure}{.5\textwidth}
  \centering
  \includegraphics[width=\linewidth]{twoD/slowosc_Tplot_medium_zoom.jpg}
  \caption{}
\end{subfigure}
\caption{Model values are $\lambda=1.3$, $\epsilon=.01$ with $A=B=2$. In (a) we have the numerical solution (black dotted) over the standard equilibrium plot for $V$ vs. $T$. In (b) a zoom of the bifurcation area.}
\label{fig:twoD_slowosc_Tnumerics_medium}
\end{figure}

In figure~\ref{fig:twoD_slowosc_Vnumerics_large} we see the numerics for a $\lambda$ that is large enough to force the problem to behave like the purely slow model like in \autoref{sec:twoD_slow}. Even upon a zoom it is almost impossible to notice that oscillations are occurring in this solution. The green dotted vertical line is our purely slow tipping estimate \eqref{eq:twoD_slow_tipping} where the blue dotted is the same mixed approximation \eqref{eq:twoD_slowosc_subcaseII_tipping}. Further evidence is seen in figure~\ref{fig:twoD_slowosc_Tnumerics_large} where this compares very similar to \eqref{fig:twoD_slow_Tnumerics}.

\begin{figure}[H]
\centering
\begin{subfigure}{.5\textwidth}
  \centering
  \includegraphics[width=\linewidth]{twoD/slowosc_bif_diagram_large.jpg}
  \caption{}
\end{subfigure}%
\begin{subfigure}{.5\textwidth}
  \centering
  \includegraphics[width=\linewidth]{twoD/slowosc_bif_diagram_large_zoom.jpg}
  \caption{}
\end{subfigure}
\caption{ Model values are $\lambda=2$, $\epsilon=.01$ with $A=B=2$. In (a) the numerical solution (black dotted line) to \eqref{eq:twoD_canonical} is given with $\eta_1=4$ and $\eta_3=.375$. In (b) a zoom in closer to the non-smooth bifurcation region where the blue vertical line is the prediction against the black dotted vertical line which is the numerical bifurcation.}
\label{fig:twoD_slowosc_Vnumerics_large}
\end{figure}

\begin{figure}[H]
\centering
\begin{subfigure}{.5\textwidth}
  \centering
  \includegraphics[width=\linewidth]{twoD/slowosc_Tplot_large.jpg}
  \caption{}
\end{subfigure}%
\begin{subfigure}{.5\textwidth}
  \centering
  \includegraphics[width=\linewidth]{twoD/slowosc_Tplot_large_zoom.jpg}
  \caption{}
\end{subfigure}
\caption{Model values are $\lambda=2$, $\epsilon=.01$ with $A=B=2$. In (a) we have the numerical solution (black dotted) over the standard equilibrium plot for $V$ vs. $T$. In (b) a zoom of the bifurcation area.}
\label{fig:twoD_slowosc_Tnumerics_large}
\end{figure}

Although the figures above show that we have classified the behavior appropriately for the various cases in $\lambda$ and relative solution sized, performance of our approximate tipping needs to be evaluated to verify our analysis had good results. In figure~\ref{fig:twoD_slowosc_lambdacomp} we compare the tipping between Case I and Case II with the numerical tipping across $\lambda$ with a fixed $\epsilon$. For smaller $\lambda$, the frequency $\Omega$ gets smaller and the Case I tipping becomes more predominant. But for the analysis performed in this section, $\Omega\gg 1$ and for $\lambda<\frac{1}{2}$ we have $\Omega\sim O(1)$. We will not consider low frequency corresponding to $\lambda<\frac{1}{2}$ in this section. The larger $\lambda$ becomes, the less effect we see from the oscillatory forcing until it is negligible for some $\lambda>1$. We notice that our one-dimensional reduction tipping approximation, there is some bias due to this being a reduced approach, there is likely a slightly bigger coefficient and this becomes apparent for $\lambda>1$. Although we use a one-dimensional reduced equation to get these approximations, they seem to be performing quite well across all $\lambda$ hence validating the method developed for tipping with varying $\lambda$.

\begin{figure}[H]
\centering
\includegraphics[width=0.7\textwidth]{twoD/slowosc_lambdacomp.jpg}
\caption{An example of numerical tipping (red stars) as the numerical solution to \eqref{eq:oneD_canonical} passes $x=.2$ for the last time. Parameter values are $\epsilon=.01$ and $A=1$. The lines are the Case I tipping estimate (black solid line) and the Case II tipping estimate (blue dotted line).}
\label{fig:twoD_slowosc_lambdacomp}
\end{figure} 

We also are interested in the performance of the tipping approximations across values of $\epsilon$ while leaving $\lambda$ fixed. The performance of each estimate is seen in figure~\ref{fig:twoD_slowosc_epscomp}. For Case I tipping, the range of appropriate $\epsilon$ is highly dependent on the choice in $\lambda$. Often, the range is very small to get accurate estimates. Once this range is left, there are interesting phase effects for the tipping which causes oscillations in the numeric tipping points. For Case II tipping, we that there is a transition happening towards the purely slow tipping, similarly to \autoref{sec:twoD_slow}


\begin{figure}[H]
\centering
\begin{subfigure}{.5\textwidth}
  \centering
  \includegraphics[width=\linewidth]{twoD/slowosc_epscomp_mixed.jpg}
  \caption{$\lambda=.8$}
\end{subfigure}%
\begin{subfigure}{.5\textwidth}
  \centering
  \includegraphics[width=\linewidth]{twoD/slowosc_epscomp_slow.jpg}
  \caption{$\lambda=1.3$}
\end{subfigure}
\caption{The numerical tipping (red stars) follows the appropriate case depending on $\lambda$ for $\epsilon=0.005$. The Case I tipping estimate (black solid line) and purely slow tipping estimate (blue dotted line) are shown.}
\label{fig:twoD_slowosc_epscomp}
\end{figure}

With the numerical results agreeing with our results, we may finally conclude that this method is both useful for analyzing the non-smooth behavior in the Stommel model but also results in an approximation that is more accurate in the extremes of the model (i.e $\Omega \gg 1$ or $\epsilon \ll 1$). This gives us a very accessible means of extracting the tipping in the full two-dimensional without needing to solve difficult Riccati equations or other complex systems that appear from the full problem. All that is needed to verify is our solutions remain stable until we arrive at the region of tipping.

\subsection{Stability}

\subsubsection{Case I: $\lambda\le 1$}

From the analysis, we had discovered the inner equations that govern the behavior of the solution for this range of $\lambda$ are

\begin{equation}\label{eq:twoD_slowosc_stability_caseI_full}
\begin{aligned}
{P_0}_t =& -\epsilon^{1-\lambda} n(t)-\eta_2 P_0 -(1-\eta_3)Q_0,\\
{Q_0}_t =& -\frac{\eta_1}{2\pi}\int_0^{2\pi}|P_0-A\cos(R)|\,dR - Q_0.
\end{aligned}
\end{equation}

But we also found that the relative size of $P_0(t)$ dictates the difficulty of \eqref{eq:twoD_slowosc_stability_caseI_full}. In the analysis we treat these as Sub-Case I: $P_0(t)\le-|A|$ and Sub-Case II: $|P_0(t)|<|A|$ which each require a separate analysis due to their differing behavior.

\subsubsection{Sub-Case I: $P_0(t)\le-|A|$}

We called this the entirely below-axis sub-case due to the solution remaining below the axis and hence predictable under these conditions thus we anticipate this sub-case to remain stable. The equation \eqref{eq:twoD_slowosc_stability_caseI_full} simplifies for this sub-case to

\begin{equation}\label{eq:twoD_slowosc_stability_subcaseI_full}
\begin{aligned}
{P_0}_t =& -\epsilon^{1-\lambda} n(t)-\eta_2 P_0 -(1-\eta_3)Q_0,\\
{Q_0}_t =& \eta_1 P_0 - Q_0.
\end{aligned}
\end{equation}

Which from the analysis we choose to reduce \eqref{eq:twoD_slowosc_stability_subcaseI_full} with the pseudo-equilibria $Q_0(P_0)=\eta_1 P_0$. This gives the following one-dimensional equation with it's pseudo-equilibria as 

\begin{equation}\label{eq:twoD_slowosc_stability_subcaseI_reduced}
\begin{aligned}
{P_0}_t =& -\epsilon^{1-\lambda}n(t)-(\eta_3+\eta_1(1-\eta_3))P_0=f(t,P_0),\\ Z^0(t) =& -\epsilon^{1-\lambda}\frac{n(t)}{\eta_3+\eta_1(1-\eta_3)}.
\end{aligned}
\end{equation}

We adopt a similar strategy for analyzing the stability from the one-dimensional model in \autoref{sec:oneD_slowosc} due to \eqref{eq:twoD_slowosc_stability_subcaseI_reduced} being a one-dimensional equation. Hence we take a simple linear perturbation about the pseudo-equilibrium with $P_0(t)=Z^0(t)+U(t)$ and $\lVert U(t)\rVert \ll 1$. Taking special care to note that $Z^0(t)$ also varies in time, we find the Taylor approximation

\begin{equation}\label{eq:twoD_slowosc_subcaseI_perturb}
\begin{aligned}
{P_0}_t=&f(t,Z^0) +f_{P_0}(t,Z^0)(P_0(t)-Z^0(t))+O(\lVert (P_0(t)-Z^0(t))^2\rVert^2),\\
U_t+Z^0_t =& -(\eta_3+\eta_1(1-\eta_3))U,\\
U_t =& -\epsilon^{1-\lambda}\frac{1}{\eta_3+\eta_1(1-\eta_3)}-(\eta_3+\eta_1(1-\eta_3))U.
\end{aligned}
\end{equation}

With \eqref{eq:twoD_slowosc_subcaseI_perturb} we find that the perturbations decay exponentially to a nearby equilibrium. This indicates the solution for this sub-case is hyperbolically stable and further agrees that no tipping will happen for this size of the solution.

\subsubsection{Sub-Case II: $|P_0(t)|<-|A|$}

We called this the crossing case and from the analysis we anticipate the tipping to occur as the solution is gradually becoming uncontrollable with the crossing. Under the conditions of this sub-case, we integrate \eqref{eq:twoD_slowosc_stability_caseI_full} with the $R_1$ and $R_2$ from the analysis and take the same Taylor approximation to find 

\begin{equation}\label{eq:twoD_slowosc_stability_subcaseII,full}
\begin{aligned}
{P_0}_t =& -n(t)-\eta_3 P_0-(1-\eta_3)Q_0,\\
{Q_0}_t =&-\epsilon^{\lambda-1}\frac{2\eta_1|A|}{\pi}-\epsilon^{1-\lambda}\frac{\eta_1(1-\eta_3)}{\pi|A|}P_0^2-Q_0.
\end{aligned}
\end{equation}

Once more, we assume that the equation for $Q_0$ is in pseudo-equilibrium 
to reduce to the following one-dimensional inner equation with equilibrium where here we let $a=\frac{\eta_1(1-\eta_3)}{\pi|A|}$ for simplicity

\begin{equation}\label{eq:twoD_slowosc_stability_subcaseII,reduced}
\begin{aligned}
{P_0}_t =& -\epsilon^{1-\lambda}n(t)+\frac{2\eta_1(1-\eta_3)|A|}{\pi}-\eta_3 P_0+aP_0^2=f(t,P_0),\\
Z^0(t) =& \frac{1}{2a}\left(\eta_3-\sqrt{4a(\epsilon^{1-\lambda}n(t)-n_{\text{osc}})}\right).
\end{aligned}
\end{equation}

Where we choose to write the argument of the square root in terms of the bifurcation found in \eqref{eq:twoD_osc_bifurcation}. We then consider the linear perturbation about the pseudo-equilibrium $P_0(t)= Z^0(t)+U(t)$ with $\lVert U\rVert \ll 1$. We take a Taylor expansion here to find the dynamics of the perturbation, but recall that we have contributions to the derivative from both the perturbation as well as the pseudo-equilibrium. This is seen with

\begin{equation}
\begin{aligned}
{P_0}_t =& Z^0_t+U_t,\\
Z^0_t=&\begin{cases}
\frac{\epsilon^{1-\lambda}}{\sqrt{4a(\epsilon^{1-\lambda}n(t)-n_{\text{osc}})}} & \epsilon^{1-\lambda}n(t)>n_{\text{osc}},\\
0 & \epsilon^{1-\lambda}n(t)=n_{\text{osc}}.
\end{cases}
\end{aligned}
\end{equation}

Thus we find the following Taylor expansion for the perturbations

\begin{equation}\label{eq:twoD_slowosc_stability_subcaseII_perturb}
\begin{aligned}
{P_0}_t =& f(t,Z^0)+f_{P_0}(t,Z^0)(P_0-Z^0)+O(\lVert P_0-Z^0 \rVert^2),\\
U_t+Z^0_t=& -\sqrt{4a(\epsilon^{1-\lambda}n(t)-n_{\text{osc}})} U,\\
 U_t = & \begin{cases}
\frac{\epsilon^{1-\lambda}}{\sqrt{4a(\epsilon^{1-\lambda}n(t)-n_{\text{osc}})}}-\left(\sqrt{4a(\epsilon^{1-\lambda}n(t)-n_{\text{osc}})}\right) U & \epsilon^{1-\lambda}n(t)>n_{\text{osc}},\\
0 & \epsilon^{1-\lambda}n(t)=n_{\text{osc}}.
\end{cases}
\end{aligned}
\end{equation}

From \eqref{eq:twoD_slowosc_stability_subcaseII_perturb} we find exponentially decaying perturbations that give asymptotic stability until we get to the oscillatory bifurcation. The oscillatory bifurcation corresponds to non-hyperbolic behavior and we lose stability shortly after which indicates of the tipping to occur after the bifurcation.

\subsubsection{Case II: $\lambda>1$}

From the analysis we determined this to be the slowly dominant case and we had discovered the inner equations that govern the behavior of the solution for this range of $\lambda$ to be

\begin{equation}\label{eq:twoD_slowosc_caseII_full}
\begin{aligned}
{P_0}_t =& - n(t)-\eta_2 P_0 -(1-\eta_3)Q_0,\\
{Q_0}_t =& -\frac{\eta_1}{2\pi}\int_0^{2\pi}|P_0-\epsilon^{\lambda-1} A\cos(R)|\,dR - Q_0.
\end{aligned}
\end{equation}

The behavior of this case when $\lambda\sim 1$ is very similar to Case I, thus we anticipate the stability to behave similarly as well. Hence we consider the behavior when $|P_0(t)|<\epsilon^{\lambda-1}|A|$ which is where we found tipping to occur in the analysis. As long as we have $\epsilon^{\lambda-1}A\sim O(1)$, we are follow the same approach as Case I where we integrate \eqref{eq:twoD_slowosc_caseII_full} with a similar $R_1$ and $R_2$ and use a Taylor approximation to get

\begin{equation*}
\begin{aligned}
{P_0}_t =& - n(t)-\eta_2 P_0 -(1-\eta_3)Q_0,\\
{Q_0}_t =& -\epsilon^{\lambda-1}\frac{2\eta_1(1-\eta_3)|A|}{\pi}-\epsilon^{1-\lambda}\frac{\eta_1}{\pi|A|}P_0^2- Q_0.
\end{aligned}
\end{equation*}

The analysis gave sufficient reason to reduce the inner equations to a one-dimensional model and thus like in Case I w find the inner equation with pseudo-equilibrium where here we let $a=\frac{\eta_1(1-\eta_3)}{\pi|A|}$ for simplicity

\begin{equation*}
\begin{aligned}
{P_0}_t =& -n(t)+\epsilon^{\lambda-1}\frac{2\eta_1(1-\eta_3)|A|}{\pi}-\eta_3 P_0+\epsilon^{1-\lambda}aP_0^2,\\
Z^0(t) =& \frac{1}{2a}\left(\epsilon^{\lambda-1}\eta_3-\sqrt{\epsilon^{\lambda-1}4a(n(t)-\epsilon^{\lambda-1}n_{\text{osc}})}\right)
\end{aligned}
\end{equation*}

Where we consider the linear perturbation about the pseudo-equilibrium $P_0(t)= Z^0(t)_U(t)$ with $\lVert U\rVert \ll 1$. We take a Taylor expansion here to find the dynamics of the perturbation, but recall that we have contributions to the derivative from both the perturbation as well as the pseudo-equilibrium. This is seen with

\begin{equation}
\begin{aligned}
{P_0}_t =& Z^0_t+U_t,\\
Z^0_t=&\begin{cases}
\frac{\epsilon^{(\lambda-1)/2}}{\sqrt{4a(n(t)-\epsilon^{\lambda-1}n_{\text{osc}})}} & n(t)>\epsilon^{\lambda-1}n_{\text{osc}},\\
0 & n(t)=\epsilon^{\lambda-1}n_{\text{osc}}.
\end{cases}
\end{aligned}
\end{equation}

Thus we find the following Taylor expansion for the perturbations

\begin{equation}\label{eq:twoD_slowosc_stability_caseII_perturb}
\begin{aligned}
{P_0}_t =& f(t,Z^0)+f_{P_0}(t,Z^0)(P_0-Z^0)+O(\lVert P_0-Z^0 \rVert^2),\\
U_t+Z^0_t=& -\left(\sqrt{\epsilon^{1-\lambda}4a(n(t)-\epsilon^{\lambda-1}n_{\text{osc}})}\right) U,\\
U_t = & \begin{cases}
\frac{\epsilon^{(\lambda-1)/2}}{\sqrt{4a(n(t)-\epsilon^{\lambda-1}n_{\text{osc}})}}-\left(\sqrt{\epsilon^{1-\lambda}4a(n(t)-n_{\text{osc}})}\right) U & n(t)>\epsilon^{\lambda-1}n_{\text{osc}},\\
0 & n(t)=\epsilon^{\lambda-1}n_{\text{osc}}.
\end{cases}
\end{aligned}
\end{equation}

From \eqref{eq:twoD_slowosc_stability_caseII_perturb} we find that the perturbations decay exponentially and we have asymptotic stability until we get to the oscillatory bifurcation. The bifurcation found in \eqref{eq:twoD_osc_bifurcation} corresponds to non-hyperbolic behavior and we then lose the stability which indicates of the tipping to occur after the bifurcation. Comparing this to Case I, we see there is small nuances between these perturbations, although the overall stability remains the same. As for when $\lambda$ grows, we already established this behaves like the purely slow model and hence we can use the stability from that section to conclude that our solution is still stable until the slow tipping point, at which we lose stability as anticipated.


Thus the stability for both Case I and Case II agrees with the results found in the analysis. We have that the behavior of the solution is stable from the outer solution, stability holds before the solution begins to cross the axis $V=0$ and once the crossing begins to happen we lose stability at the location of the oscillatory bifurcation. Because there is slow variation in this model, there is still delayed behavior and thus the tipping happens shorty after the oscillatory bifurcation. In both cases we discovered that the pseudo-equilibrium has a contribution to the derivative and this in turn causes the perturbations to decay towards a small constant. This means that there is a small region around the pseudo-equilibrium that attracts the solution and this is seen in the numerical results. 


\chapter{Summary and Future Work}
\label{chap:conclusion}
With the results found in this thesis, we have accurately described what kind of behavior is present about the non-smooth bifurcation when new mechanisms are introduced in both the one-dimensional model \eqref{eq:oneD_canonical} and Stommel model \eqref{eq:twoD_canonical}. The novelty of this work comes from the link between delayed tipping methods to the non-smooth Stommel model. This then paves the way for a more general approach to the broader class of non-smooth dynamical systems.

To describe a large class of observable behaviors, we considered the mixture of early bifurcation due to high oscillatory forcing with frequency $\Omega\gg 1$ and amplitude $A\sim O(1)$ as well as delayed tipping due to slow variation in the bifurcating parameter at rate $\epsilon\ll 1$. The main result being that these mechanisms have opposite effects on the tipping point and do mix with a kind of weighted average to produce an effective tipping approximation. These results give insight into the hysteresis behavior of the Stommel model and the less studied realm of non-smooth dynamics. The main approach used asymptotic expansions as well as the methods of multiple scales to identify reduced equations and to find asymptotic solutions to the models. We found that with oscillatory forcing, the reduced equations had a different expression depending on the relative size of the solution to the amplitude of oscillation. We also discovered that linking the slow variation $\epsilon$ and the frequency $\Omega$ gives important insight into how the system behaves. 

The methods developed for finding tipping points in the one-dimensional model \eqref{eq:oneD_canonical} gave good approximations with the numerical results. Due to the many similarities to the two-dimensional system \eqref{eq:twoD_canonical} we were able to modify the same analysis to find the tipping points here as well.

Future work would need to be done on cases where $\Omega\sim O(1)$ or smaller. This case is qualitatively different as slow oscillations may have dramatic contributions to the dynamics. This is also seen from the analysis where low frequency oscillations no longer allow for asymptotic expansions in terms of $\Omega^{-1}$ and no longer fall under our assumptions to integrate with $T_1$ and $T_2$. Thus this case behaves fundamentally different and can influence tipping in a way we hadn't explored here. This is shown in figure~\ref{fig:low_freq}. Also, large amplitude behavior $A>O(1)$ can force an additional rescaling before any familiar approaches hold. This is seen in figure~\ref{fig:large_amp}. These cases were mentioned but have yet to be performed on this model, although both have been studied around the smooth case in \cite{zhu2015tipping}. It is possible that they could have some surprising results in the non-smooth case. Together, these could help to further classify the tipping behavior for the variety of cases in real world ocean dynamics.

\begin{figure}[H]
\centering
\begin{subfigure}{.5\textwidth}
 \centering
 \includegraphics[width=\linewidth]{conclusion/low_freq_V.jpg}
 \caption{}
\end{subfigure}%
\begin{subfigure}{.5\textwidth}
 \centering
 \includegraphics[width=\linewidth]{conclusion/low_freq_T.jpg}
 \caption{}
\end{subfigure}
\caption{Low Frequency: Model parameters are $\epsilon=.01$, $A=B=1$ and $\Omega=3$.}
\label{fig:low_freq}
\end{figure}

\begin{figure}[H]
\centering
\begin{subfigure}{.5\textwidth}
 \centering
 \includegraphics[width=\linewidth]{conclusion/large_amp_V.jpg}
 \caption{}
\end{subfigure}%
\begin{subfigure}{.5\textwidth}
 \centering
 \includegraphics[width=\linewidth]{conclusion/large_amp_T.jpg}
 \caption{}
\end{subfigure}
\caption{Large Amplitude: Model parameters are $\epsilon=.01$, $A=B=300$ and $\Omega=1000$.}
\label{fig:large_amp}
\end{figure}

Lastly, we considered only deterministic behavior throughout this analysis but there are many reasons to incorporate stochastic elements into the Stommel model as well, see \cite{lorenzo2012role}. From \cite{zhu2015tipping} it is concluded that stochastic forcing has elements of both early bifurcations and delayed tipping and thus a natural follow-up to the analysis in this thesis. We could consider stochastic forcing with

\begin{equation}\label{eq:stochastic}
\begin{aligned}
\dot{V} & = \eta_1-\eta_2+\eta_3(T-V)-T-V|V|+A\xi_1(t), \\
  \dot{T}   & = \eta_1-T(1+|V|)+B\xi_2(t), \\
 \dot{\eta_2} & = -\epsilon\\
 V(0)=&V^0,\quad T(0)=T^0, \quad \eta_2(0)={\eta_2}^0,
\end{aligned}
\end{equation}

where $\xi_i(t)$ is standard Gaussian noise with mean 0 and variance $t$ and initial conditions centered on the lower branch. This is shown in figure~\ref{fig:stochastic} and it is clear a completely separate analysis is needed.

\begin{figure}[H]
\centering
\begin{subfigure}{.5\textwidth}
 \centering
 \includegraphics[width=\linewidth]{conclusion/stochastic_V.jpg}
 \caption{}
\end{subfigure}%
\begin{subfigure}{.5\textwidth}
 \centering
 \includegraphics[width=\linewidth]{conclusion/stochastic_V_zoom.jpg}
 \caption{}
\end{subfigure}
\begin{subfigure}{.5\textwidth}
 \centering
 \includegraphics[width=\linewidth]{conclusion/stochastic_T.jpg}
 \caption{}
\end{subfigure}%
\begin{subfigure}{.5\textwidth}
 \centering
 \includegraphics[width=\linewidth]{conclusion/stochastic_T_zoom.jpg}
 \caption{}
\end{subfigure}
\caption{Stochastic: In (a) many realizations of the numerical solution for $V$ in \eqref{eq:stochastic} is given with model parameters $\eta_1=4$, $\eta_3=.375$, $\epsilon=.01$ and $A=B=.7$. In (b) a zoom in closer to the non-smooth bifurcation region. In (c) we have the realizations over the standard equilibrium plot for $V$ vs. $T$. In (d) a zoom of the bifurcation area.}
\label{fig:stochastic}
\end{figure}

%%% Note: the bibliography must come before the appendices.
\bibliographystyle{plain}
\bibliography{thesis_bib}

%% Use this to reset the appendix counter.  Note that the FoGS
%% requires that the word ``Appendices'' appear in the table of
%% contents either before each appendix lable or as a division
%% denoting the start of the appendices.  We take the latter option
%% here.  This is ensured by making the \texttt{appendicestoc} option
%% a default option to the UBC thesis class.

%%% If you only have one appendix, please uncomment the following line.
% \renewcommand{\appendicesname}{Appendix}
\appendix
\chapter{The Stommel Model}
\label{app:stommel}
\input{chapters/stommelappendix.tex}

\chapter{One-Dimensional}
\label{app:oneD}
\section*{High Frequency Oscillatory Forcing}

Here we continue the analysis to explicitly find the solution to the outer equation for the purely oscillatory model. Recall we found $x_1 = v_1(t)-A\cos(T)$, we then apply the Fredholm alternative \eqref{eq:Fredholm} to the $O(\Omega^{-2})$ equation in \eqref{eq:oneD_osc_outerO3} to get

\begin{equation}\label{eq:oneD_app_osc}
\begin{aligned}
0=&\frac{1}{2\pi}\int_0^{2\pi}-{x_1}_t-2x_1+2x_0x_1\, dT,\\
{v_1}_t=& -2v_1+2(1-\sqrt{1+\mu})v_1,\\
{v_1}_t=& -2\sqrt{1+\mu}v_1
\end{aligned}
\end{equation}

We search for the equilibrium to find stable behavior on this order but since \eqref{eq:oneD_app_osc} has a very simple form, the equilibrium is $v_1(t)\equiv 0$ and thus we find the correction term to only have oscillatory behavior, $x_1=-A\cos(T)$.

\section*{Slow Variation and Oscillatory Forcing}

Here we continue to find the terms of the outer solution for the slow varying and oscillatory forcing model. We have thus far found $x_0=x_0(\tau)$ and we have equations at $O(\epsilon^\lambda)$ and $O(\epsilon^{2\lambda})$ that give information about $x_0$ and $x_1$ respectively. From the $O(\epsilon^\lambda)$ equation \eqref{eq:oneD_slowosc_outerO2}, we apply the Fredholm alternative \eqref{eq:Fredholm} to find

\begin{equation}\label{eq:oneD_app_slowosc1}
\begin{aligned}
0 = & \frac{1}{2\pi}\int_0^{2\pi} -\mu(\tau) -2x_0(\tau)+x_0(\tau)^2+A\sin(T)\,dT,\\
0=&-\mu(\tau)-2x_0(\tau)+x_0(\tau)^2,\\
x_0(\tau) =& 1-\sqrt{1+\mu(\tau)},\\
{x_1}_T = & A\sin(T)
\end{aligned}
\end{equation}

From \eqref{eq:oneD_app_slowosc1} we find that $x_1=v_1(\tau)-A\cos(T)$, which gives us access to solving the next order equation. Thus we now do the same for the $O(\epsilon^{2\lambda})$ equation \eqref{eq:oneD_slowosc_outerO3} to find

\begin{equation}
\begin{aligned}
0=&\frac{1}{2\pi}\int_0^{2\pi}-\epsilon^{1-\lambda}{x_0}_\tau -2x_1+2x_0x_1\,dT,\\
\epsilon^{1-\lambda}{x_0}_\tau=& -2v_1+2(1-\sqrt{1+\mu(\tau)})v_1,\\
v_1(\tau) =& -\epsilon^{1-\lambda}\frac{{x_0}_\tau}{2\sqrt{1+\mu(\tau)}}.
\end{aligned}
\end{equation}

Where we recall that $\mu_\tau=-1$ and that ${x_0}_\tau = -\frac{{\mu}_\tau}{2\sqrt{1+\mu(\tau)}}=\frac{1}{2\sqrt{1+\mu(\tau)}}$ thus we have the form of the next order term in the expansion as

\begin{equation}
x_1(\tau,T) = -\epsilon^{1-\lambda}\frac{1}{4(1+\mu(\tau))}-A\cos(T).
\end{equation}

\chapter{Two-Dimensional}
\label{app:twoD}
\section*{High Frequency Oscillatory Forcing}

Here we show that the correction term for the outer solution is purely oscillatory. From the analysis, we found both leading order terms to be purely slow dependent $V_0=V_0(\tau)$ and $T_0=T_0(\tau)$. To find the explicit form for these, we apply Fredholm \eqref{eq:Fredholm} to the $O(\Omega^{-1})$ equations in \eqref{eq:twoD_osc_outerO2} to find

\begin{equation}\label{eq:twoD_app_osc1}
\begin{aligned}
&\begin{cases}
	0 = & \frac{1}{2\pi}\int_0^{2\pi}-{V_0}_t+\eta_1-\eta_2+\eta_3(T_0-V_0)-T_0+V_0^2+A\sin(R)\,dR,\\
	 0 =& \frac{1}{2\pi}\int_0^{2\pi}-{T_0}_t+ \eta_1-T_0(1-V_0)+B\sin(R)\,dR,
\end{cases}\\
&\begin{cases}
	{V_0}_t = & \eta_1-\eta_2+\eta_3(T_0-V_0)-T_0+V_0^2,\\
	 {T_0}_t =&  \eta_1-T_0(1-V_0),
\end{cases}\\
& {V_1}_R = A\sin(R),\quad {T_1}_R = B\cos(R).
\end{aligned}
\end{equation}

Since we have a fixed parameter $\eta_2$, we find the following equilibria both $V_0$ and $T_0$ as well as the form of the correction terms

\begin{equation*}
\begin{aligned}
T_0&(V_0) = \frac{\eta_1}{1-V_0},\\
0= & \eta_1-\eta_2 +\eta_3(T_0(V_0)-V_0)-T_0(V_0)+V_0^2,\\
V_1 =& X_1(t)-A\cos(R),  \quad T_1 = Y_1(t)-B\cos(R).
\end{aligned}
\end{equation*}

Where we note these equilibria to be the same as the static problem with no forcing from the introduction. But with the form of the correction terms, we now solve the equation at $O(\Omega^{-2})$ \eqref{eq:twoD_osc_outerO3} by applying Fredholm \eqref{eq:Fredholm} again. This results in

\begin{equation}\label{eq:twoD_app_osc2}
\begin{aligned}
&\begin{cases}
	0 = & \frac{1}{2\pi}\int_0^{2\pi}-{V_1}_t+\eta_3(T_1-V_1)-T_1+2V_0V_1\,dR,\\
	 0 =& \frac{1}{2\pi}\int_0^{2\pi}-{T_1}_t+ T_1(1-V_0)+T_0V_1\,dR,
\end{cases}\\
&\begin{cases}
	{X_1}_t = & \eta_3(Y_1-X_1)-Y_1+2X_0X_1,\\
	 {Y_1}_t =&  Y_1(1-X_0)+Y_0X_1,
\end{cases}
\end{aligned}
\end{equation}

We then search for the equilibria of \eqref{eq:twoD_app_osc2} to find

\begin{equation*}
\begin{aligned}
Y_1&(X_1) =-\frac{Y_0X_1}{1-X_0},\\
0 =& \left(\eta_3\left(\frac{Y_0}{1-X_0}-1\right)-\frac{Y_0}{1-X_0}+2X_0\right)X_1.
\end{aligned}
\end{equation*}

Thus we find that the correction terms are purely oscillatory since $X_1\equiv 0$ and $Y_1 \equiv 0$. Thus we have $V_1=-A\cos(R)$ and $T_1=-B\cos(R)$.

\section*{Slow Variation and Oscillatory Forcing}

Here we continue to find terms of the outer solution by working through the equations \eqref{eq:twoD_slowosc_outerO2}-\eqref{eq:twoD_slowosc_outerO3}. From the analysis, we had already determined that the leading order terms are purely slow, $V_0=V_0(\tau)$ and $T_0=T_0(\tau)$. To find their exact form, we apply Fredholm \eqref{eq:Fredholm} to the $O(\epsilon^\lambda)$ equation \eqref{eq:twoD_slowosc_outerO2} to find 

\begin{equation}\label{eq:twoD_app_slowosc1}
\begin{aligned}
&\begin{cases}
	0 = & \frac{1}{2\pi}\int_0^{2\pi}\eta_1-\eta_2(\tau)+\eta_3(T_0-V_0)-T_0+V_0^2+A\sin(R)\,dR,\\
	 0 =& \frac{1}{2\pi}\int_0^{2\pi} \eta_1-T_0(1-V_0)+B\sin(R)\,dR,
\end{cases}\\
&\begin{cases}
	0 = & \eta_1-\eta_2(\tau)+\eta_3(T_0-V_0)-T_0+V_0^2,\\
	 0 =&  \eta_1-T_0(1-V_0),
\end{cases}\\
& {V_1}_R = A\sin(R),\quad {T_1}_R = B\cos(R).
\end{aligned}
\end{equation}

The leading order solution to \eqref{eq:twoD_app_slowosc1} is the same as the purely slow problem in \autoref{sec:twoD_slow} with

\begin{equation*}
\begin{aligned}
T_0&(V_0)=\frac{\eta_3}{1-V_0},\\
0=& \eta_1-\eta_2(\tau)+\eta_3(T_0(V_0)-V_0)-T_0(V_0)+V_0^2.
\end{aligned}
\end{equation*}

But we also find the form of the correction terms, $V_1 = X_1(\tau) -A\cos(R)$ and $T_1 = Y_1(\tau)-B\cos(R)$, which allow us to solve the $O(\epsilon^{2\lambda})$ equation \eqref{eq:twoD_slowosc_outerO3}. Applying Fredholm \eqref{eq:Fredholm} here results in 

\begin{equation}\label{eq:twoD_app_slowosc2}
\begin{aligned}
&\begin{cases}
	0 = & \frac{1}{2\pi}\int_0^{2\pi}-\epsilon^{1-\lambda}{V_0}_\tau+\eta_3(T_1-V_1)-T_1+2V_0V_1+A\sin(R)\,dR,\\
	 0 =& \frac{1}{2\pi}\int_0^{2\pi} \epsilon^{1-\lambda}{T_0}_\tau-T_1(1-V_0)+T_0V_1\,dR,
\end{cases}\\
&\begin{cases}
	\epsilon^{1-\lambda}{V_0}_\tau=&\eta_3(Y_1-X_1)-Y_1+2V_0X_1, \\
	 \epsilon^{1-\lambda}{T_0}_\tau =&  Y_1(1-V_0)+T_0X_1,
\end{cases}
\end{aligned}
\end{equation}

We recall that ${\eta_2}_\tau=-1$ and solving \eqref{eq:twoD_app_slowosc2} requires the derivatives of $V_0$ and $T_0$ which are solvable explicitly as

\begin{equation*}
\begin{aligned}
{T_0}_\tau&({V_0}_\tau) = -\frac{\eta_1{V_0}_\tau}{1-V_0},\\
{V_0}_\tau &=\frac{(1-V_0)}{\eta_1+\eta_3(\eta_1+1-V_0)-2V_0(1-V_0)}. 
\end{aligned}
\end{equation*}

With everything put together, we find the solution to \eqref{eq:twoD_app_slowosc2} as 

\begin{equation*}
\begin{aligned}
Y_1(X_1) =& \frac{\epsilon^{1-\lambda}{T_0}_\tau-T_0X_1}{1-V_0},\\
X_1=&\frac{\epsilon^{1-\lambda}({V_0}_\tau(1-V_0)+(1-\eta_3){T_0}_\tau)}{(1-\eta_3)T_0+(2V_0-\eta_3)(1-V_0)}.
\end{aligned}
\end{equation*}

Where we now have the first correction term as

\begin{equation*}
\begin{aligned}
V_1(\tau,R)=& \frac{\epsilon^{1-\lambda}({V_0}_\tau(1-V_0)+(1-\eta_3){T_0}_\tau)}{(1-\eta_3)T_0+((2V_0-\eta_3)(1-V_0)}-A\cos(R),\\
T_1(\tau,R)=&\frac{\epsilon^{1-\lambda}{T_0}_\tau}{1-V_0}-\frac{\epsilon^{1-\lambda}T_0({V_0}_\tau(1-V_0)+(1-\eta_3){T_0}_\tau)}{(1-\eta_3)T_0(1-V_0)+(2V_0-\eta_3)(1-V_0)^2}-B\cos(R).
\end{aligned}
\end{equation*}

\backmatter

\end{document}
\endinput
%%
%% End of file `ubcsample.tex'.
